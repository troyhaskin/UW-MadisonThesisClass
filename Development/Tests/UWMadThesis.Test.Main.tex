\documentclass{UWMadThesis}

\usepackage{lipsum}

% ========================== %
%     Thesis Information     %
% ========================== %
\Title{UWMadThesis Example PDF}
\Author{Troy C. Haskin}
\Program{Nuclear Engineering--Engineering Physics}
\Doctorate
\DefenseDate{July 14, 2014}
\University{University of Wisconsin--Madison}
\Dissertation


\MakeLinksBlack
\UWMadSetup {
    LayoutStyle / {
        make-links-red = true,
        paragraph-style = indent
    }
}


\Advisor{Michael L. Corradini}       {Professor}{Engineering Physics}
\CommitteeMember{Robert J. Witt}     {Professor}{Engineering Physics}
\CommitteeMember{Gregory A. Moses}   {Professor}{Engineering Physics}
\CommitteeMember{Jake Blanchard}     {Professor}{Engineering Physics}
\CommitteeMember{Christopher Rutland}{Professor}{Mechanical Engineering}

\begin{document}

% ========================== %
%       Thesis Committee     %
% ========================== %


% =========================================================================== %
%                              SpecialPages Tests                             %
% =========================================================================== %
\MakeTitlePage{}
\begin{LicensePage}
    \CreativeCommons
    \Attribution
    \NonCommercial
    \ShareAlike
\end{LicensePage}



% =========================================================================== %
%                              Front Matter Tests                             %
% =========================================================================== %

\acknowledgments{}
\lipsum[1]

\TableOfContents

\abstract
\lipsum[2-3]




% =========================================================================== %
%                               Nomenclature Tests                            %
% =========================================================================== %
\begin{Nomenclature}[Symbol Table]%
    \Entry{rho}{Density}%
    \Entry{b}{Traditions surrounding maiden speeches vary from country to country. In many Westminster system governments, there is a convention that maiden speeches should be relatively uncontroversial, often consisting of a general statement of the politician's beliefs and background rather than a partisan comment on a current topic.}%
    \Entry{bbbbbbbbbbbbbbbbbbbbb}{The Scaly Yellowfish (Labeobarbus natalensis) is a fish found in the Umzimkulu River System as well as in the Umgeni, Umkomazi, Tukhela and the Umfolozi. It is a common endemic species in KwaZulu-Natal Province and it lives in different habitats between the Drakensberg foothills and the coastal lowlands.}%
\end{Nomenclature}


% =========================================================================== %
%                            RelativeDirectory Tests                          %
% =========================================================================== %
%
%   This section of input aims to test the RelativeDirectory module of the
%   \UWMadClass{}.  It switches naming conventions mid-document to achieve
%   this goal efficiently.  It is advised to NOT CHANGE naming modes as this
%   example does.
%
%
%
%   RelativeDirectory Test:
%       o Chapter directory prefix {Chapter-} in increment mode
%       o Empty section *-fixes to test the stack-pushing conditions
\UWMadSetup {
    RelativeDirectory / {
        chapter-directory-prefix = Chapter-,
        chapter-directory-name   = increment
    }
}
\IncludeChapter{Main}
    \IncludeSection{Section-1}
    \IncludeSection{Section-2}
%
%
%   RelativeDirectory Test:
%       o Chapter directory prefix {Chapter-} in increment mode
%       o Section directory prefix {Section-} in increment mode
%       o All same name to test the stack-pushing conditions and file in 
%           subfolders.
\UWMadSetup { RelativeDirectory / {
    section-directory-prefix    = Section-,
    section-directory-name      = increment,
    subsection-directory-prefix = Subsection-,
    subsection-directory-name   = increment,
    graphics-directory-name     = Graphics
}}
\IncludeChapter{Main}
%
%
%   RelativeDirectory Test:
%       o Chapter    directory prefix {} and suffix {-Chapter}    in same mode
%       o Section    directory prefix {} and suffix {-Section}    in same mode
%       o Subsection directory prefix {} and suffix {-Subsection} in same mode
%       o All same name-mode and same file name to test the stack-pushing.
\UWMadSetup{
    RelativeDirectory / {
        chapter-directory-prefix    = {},
        chapter-directory-name      = same,
        chapter-directory-suffix    = {-Chapter},
        section-directory-prefix    = {},
        section-directory-name      = same,
        section-directory-suffix    = {-Section},
        subsection-directory-prefix = {},
        subsection-directory-name   = same,
        subsection-directory-suffix = {-Subsection},
        the-only-graphics-directory = Graphics
    }
}
\IncludeChapter{SameNameTest}
    \IncludeSection{SameNameTest}
        \IncludeSubsection{SameNameTest}
%
%
%   RelativeDirectory Test:
%       o Chapter    directory is user-supplied: OptionalDirectoryName.
%       o Section continues the name-mode from above (same) to test chaining.
\IncludeChapter[OptionalDirectoryName]{Optional}
    \IncludeSection{ChainTest}
%
%
%   RelativeDirectory Test:
%       o Chapter    directory is user-supplied: OptionalDirectoryName.
%       o Section continues the name-mode from above (same) to test chaining.
\UWMadSetup { RelativeDirectory / {
    chapter-directory-prefix    = SingleSectionDirectoryTest,
    chapter-directory-name      = none,
    chapter-directory-suffix    = {},
    section-directory-prefix    = SectionsAndSubsections,
    section-directory-name      = none,
    section-directory-suffix    = {},
    subsection-directory-prefix = {},
    subsection-directory-name   = none,
    subsection-directory-suffix = {}
}}
\IncludeChapter{Chapter}
    \IncludeSection{Section}
        \IncludeSubsection{Subsection}


\appendix{First Appendix}
\lipsum[7-9]


\appendix{Second Appendix}
\lipsum[10-15]


\end{document}