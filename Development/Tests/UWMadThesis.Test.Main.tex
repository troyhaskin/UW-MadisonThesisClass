\documentclass[12pt]{UWMadThesis}

\usepackage{lipsum}

% ========================== %
%     Thesis Information     %
% ========================== %
\Title{UWMadThesis Example PDF}
\Author{Troy C. Haskin}
\Program{Nuclear Engineering--Engineering Physics}
\Doctorate
\DefenseDate{July 14, 2014}
\University{University of Wisconsin--Madison}
\Dissertation


\MakeLinksBlack


% ========================== %
%       Thesis Committee     %
% ========================== %
\Advisor{Michael L. Corradini}       {Professor}{Engineering Physics}
\CommitteeMember{Robert J. Witt}     {Professor}{Engineering Physics}
\CommitteeMember{Gregory A. Moses}   {Professor}{Engineering Physics}
\CommitteeMember{Jake Blanchard}     {Professor}{Engineering Physics}
\CommitteeMember{Christopher Rutland}{Professor}{Mechanical Engineering}

\begin{document}

% =========================================================================== %
%                              SpecialPages Tests                             %
% =========================================================================== %
\MakeTitlePage{}
\begin{LicensePage}
    \CreativeCommons
    \Attribution
    \NonCommercial
    \ShareAlike
\end{LicensePage}



% =========================================================================== %
%                              Front Matter Tests                             %
% =========================================================================== %
\TableOfContents

\acknowledgments{}
\lipsum[1]

\abstract
\lipsum[2-3]




% =========================================================================== %
%                               Nomenclature Tests                            %
% =========================================================================== %
\begin{Nomenclature}
    \Entry{$\rho$}{Density}
\end{Nomenclature}


% =========================================================================== %
%                            RelativeDirectory Tests                          %
% =========================================================================== %
%
%   This section of input aims to test the RelativeDirectory module of the
%   \UWMadClass{}.  It switches naming conventions mid-document to achieve
%   this goal efficiently.  It is advised to NOT CHANGE naming modes as this
%   example does.
%
%
%
%   RelativeDirectory Test:
%       o Chapter directory prefix {Chapter-} in increment mode
%       o Empty section *-fixes to test the stack-pushing conditions
\RelativeDirectorySetup {
    chapter-directory-prefix = Chapter-,
    chapter-directory-name   = increment
}
\IncludeChapter{Main}
    \IncludeSection{Section-1}
    \IncludeSection{Section-2}
%
%
%   RelativeDirectory Test:
%       o Chapter directory prefix {Chapter-} in increment mode
%       o Section directory prefix {Section-} in increment mode
%       o All same name to test the stack-pushing conditions
\RelativeDirectorySetup {
    section-directory-prefix    = Section-,
    section-directory-name      = increment,
    subsection-directory-prefix = Subsection-,
    subsection-directory-name   = increment
}
\IncludeChapter{Main}
    \IncludeSection{Main}
        \IncludeSubsection{Main}
        \IncludeSubsection{Main}
    \IncludeSection{Main}
        \IncludeSubsection{Main}
        \RelativeDirectorySetup {
            graphics-directory-name = Graphics
        }
        \IncludeSubsection{Main}
%
%
%   RelativeDirectory Test:
%       o Chapter    directory prefix {} and suffix {-Chapter}    in same mode
%       o Section    directory prefix {} and suffix {-Section}    in same mode
%       o Subsection directory prefix {} and suffix {-Subsection} in same mode
%       o All same name-mode and same file name to test the stack-pushing.
\RelativeDirectorySetup {
    chapter-directory-prefix    = {},
    chapter-directory-name      = same,
    chapter-directory-suffix    = {-Chapter},
    section-directory-prefix    = {},
    section-directory-name      = same,
    section-directory-suffix    = {-Section},
    subsection-directory-prefix = {},
    subsection-directory-name   = same,
    subsection-directory-suffix = {-Subsection}
}
\IncludeChapter{SameNameTest}
    \IncludeSection{SameNameTest}
        \IncludeSubsection{SameNameTest}
%
%
%   RelativeDirectory Test:
%       o Chapter    directory is user-supplied: OptionalDirectoryName.
%       o Section continues the name-mode from above (same) to test chaining.
\IncludeChapter[OptionalDirectoryName]{Optional}
    \IncludeSection{ChainTest}


\appendix{First Appendix}
\lipsum[7-9]


\appendix{Second Appendix}
\lipsum[10-15]


\end{document}