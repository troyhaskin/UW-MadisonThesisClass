%<*UserGuide>

\UWFeature{List Environments}

The \UWMadClass{} has a special set of functions from creating list environments (called \texttt{ListOf} in the implementation).
The functions use queues and associative arrays to store and use data before it is typeset.
These data structures allow for operations to be carried out without writing external files or repeating compilation; of course, there is added memory usage which could lead to problems on older systems.

The primary motivation for such a system was the creation of a nomenclature environment and, subsequently, an acronym environment/system.
These two similar features are discussed here.

\UWSubFeature{Nomenclature}
The \texttt{Nomenclature} environment is, by default, a list of \texttt{(symbol, description)} entries.
There is a user option for changing the system to a list of \texttt{(symbol, units, description)} entries if a separate unit column is desired.
For every set of entries, the nomenclature system measures the width of the \texttt{symbol} and (if present) \texttt{units} to determine the maximum width of the \texttt{description} such that no text overflows into the margins of the page.

When first adding entries to a nomenclature, the symbols are part of the so-called Main group.
The Main group has a title and a section level associated with it.
By default, the Main group title is ``Nomenclature'' and the section is ``chapter''.
The entries can be put into two lower sectioned groups using the \cs{Group} and \cs{Subgroup} commands described below.
The grouping commands allows a set of symbols to be classified as ``Greek Symbols'' while another is ``Subscripts''.
The default titles for these lower groups are empty by default and the default section is ``section'' and ``subsection''.

All of these defaults can be changed by the \cs{NomenclatureSetup} command described below.

\UWSubSubFeature{Command Descriptions}

A sketch of the \texttt{Nomenclature} implementation would be:              \vskip1em
    \hspace{1em}\cs{begin{Nomenclature}}\cs{oarg}{section}\cs{marg}{title}  \vskip0em
    \hspace{3em}\cs{Entry}\cs{marg}{symbol}\cs{marg}{description}           \vskip0em
    \hspace{3em}\cs{Group}\cs{marg}{group title}                            \vskip0em
    \hspace{6em}\cs{Entry}\cs{marg}{symbol}\cs{marg}{description}           \vskip0em
    \hspace{6em}\cs{Subgroup}\cs{marg}{subgroup title}                      \vskip0em
    \hspace{9em}\cs{Entry}\cs{marg}{symbol}\cs{marg}{description}           \vskip0em
    \hspace{1em}\cs{end{Nomenclature}}                                      \vskip1em

The square brace-delimited \oarg{section} is \textsc{optional} and overrides the default Main group section.
The curly brace-delimited  \marg{title} is \textsc{optional} and overrides the default Main group title.

\begin{function}{\Entry}
    \begin{syntax}
        \cs{Entry}\marg{symbol}\marg{description}
        \cs{Entry}\marg{symbol}\marg{units}\marg{description}
    \end{syntax}
    Within the environment, entries are added to the nomenclature using the \cs{Entry} command above.
    All arguments are required.
    The second version above is if a units column is requested (see \hyperref[UG/Nomenclature/Customization]{Customization}).
\end{function}

\begin{function}{\Group,\Subgroup}
    \begin{syntax}
        \cs{Group}\marg{group title}
        \cs{Subgroup}\marg{subgroup title}
    \end{syntax}
    Creates a group or subgroup with the indicated title and using the default section.
    The default section can be changed by the user (see \hyperref[UG/Nomenclature/Customization]{Customization}).
\end{function}


\UWSubSubFeature{Examples}
As an example, the following input
\begin{verbatim}
    \begin{Nomenclature}[subsubsection]{Symbol Table}
        \Entry{$\rho$}{Density}
        \Entry{LongNotRealSymbol}{
            In publishing and graphic design, lorem ipsum is a placeholder
            text commonly used to demonstrate the graphic elements of a
            document or visual presentation. By replacing the distraction
            of meaningful content with filler text of scrambled Latin it
            allows viewers to focus on graphical elements such as font,
            typography, and layout.}
        \Entry{$\mu$}{Viscosity}
    \end{Nomenclature}
\end{verbatim}
would be typeset as:

\setcounter{section}{1}
\UWMadSetup{ Nomenclature } {include-in-toc = false}

\rule{\textwidth}{0.1em}
    \begin{Nomenclature}[subsection]{Symbol Table}
        \Entry{$\rho$}{Density}
        \Entry{LongNotRealSymbol}{
            In publishing and graphic design, lorem ipsum is a placeholder
            text commonly used to demonstrate the graphic elements of a
            document or visual presentation. By replacing the distraction
            of meaningful content with filler text of scrambled Latin it
            allows viewers to focus on graphical elements such as font,
            typography, and layout.}
        \Entry{$\mu$}{Viscosity}
    \end{Nomenclature}
\rule{\textwidth}{0.1em}
As can be seen, the symbol column is as wide as the widest symbol (plus some padding) and lengthy text can be put into the description without penalty.
Of course, this example is purposefully extreme.
We can tweak the example a bit more by putting the second two items under a group:

\rule{\textwidth}{0.1em}
    \begin{Nomenclature}[subsection]{Symbol Table}
        \Entry{$\rho$}{Density}
        
        \Group{Group 1 Title}
        \Entry{LongNotRealSymbol}{
            In publishing and graphic design, lorem ipsum is a placeholder
            text commonly used to demonstrate the graphic elements of a
            document or visual presentation. By replacing the distraction
            of meaningful content with filler text of scrambled Latin it
            allows viewers to focus on graphical elements such as font,
            typography, and layout.}
        \Entry{$\mu$}{Viscosity}
    \end{Nomenclature}
\rule{\textwidth}{0.1em}
By default, the section level used by \cs{Group} is one below that of the main nomenclature section; therefore, since the nomenclature's section level is defined as \texttt{subsection}, the \cs{Group} is a \texttt{subsubsection}.
Not shown: using \cs{Subgroup} would typeset the title as a \texttt{paragraph} in this example.


\UWSubSubFeature{Customization}\label{UG/Nomenclature/Customization}

As mentioned, there are several options available to the user for customizing the nomenclature.
These options are set by giving a comma-separate list of key-value pairs to the function \cs{UWMadSetup} with the module name \texttt{Nomenclature}:
    \begin{verbatim}
        \UWMadSetup { Nomenclature } {
            key-one = option,
            key-two = {option two},
            ...
            key-n =  {option n},
        }
    \end{verbatim}
A table of the keys, meaning, defaults, and allow value is given in \cref{Table:NomenclatureKeyValue}.


% ================================================================ %
%                               Acronym                            %
% ================================================================ %
\clearpage
\UWSubFeature{Acronym}

\UWSubSubFeature{Description}
The \texttt{Acronym} environment is a specialized extension of the \texttt{Nomenclature} environment.
It has the same basic syntax, but a \texttt{units} column is not supported.
Also, instead of \cs{Entry} taking \texttt{(symbol, description)} pairs, it takes \texttt{(acronym,meaning)} pairs.
Lastly, it comes equipped with a new command: \cs{Acro}.

\begin{function}{\Acro}
    \begin{syntax}
        \cs{Acro}\marg{acronym}
    \end{syntax}
    \cs{Acro} is meant to be used throughout the document to reference back to the \texttt{Acronym} environment where it was defined.
    If an \texttt{Acronym} environment contains the line \cs{Entry}\texttt{\{TBD\}\{To be determined\}}, the first usage of \cs{Arco}\texttt{\{TBD\}} will be typeset as `To be determined (TBD)' while subsequent uses will simply be `TBD'.
    Also, if links are not turned off (they are on by default), the acronym will be a link back to the original environment entry.
\end{function}

\UWSubSubFeature{Example}
The following input
\begin{verbatim}
    \UWMadSetup { Acronym } {
        main-section  = section,
        main-title = {Acronym Table},
        entry-padding = 1in
    }
    \begin{Acronym}
        \Entry{RCCS}{Reactor Cavity Cooling System}
        \Entry{NRC}{Nuclear Regulatory Commission}
    \end{Acronym}
\end{verbatim}
is typeset as

\rule{\textwidth}{0.1em}
    \AcronymSetup {
        main-section  = section,
        main-title = {Acronym Table},
        entry-padding = 1in
    }
    \begin{Acronym}
        \Entry{RCCS}{Reactor Cavity Cooling System}
        \Entry{NRC}{Nuclear Regulatory Commission}
    \end{Acronym}
\rule{\textwidth}{0.1em}

The first usage of \cs{Acro}\{NRC\} is `\Acro{NRC}' while the second usage is `\Acro{NRC}'.


\UWSubSubFeature{Acronym Customization}

Since this feature is an extension of the \texttt{Nomenclature} feature, it is customized in a similar fashion: using \cs{UWMadSetup} and the \texttt{Acronym} module name.
It shares all of the same keys with some additional ones outline in \cref{Table:AcronymKeyValue}.


\begin{table}[H]
\begin{center}
    \caption{List of key-value pairs for Nomenclature customization.}
    \label{Table:NomenclatureKeyValue}
    \begin{tabular}{c c c c}
        \toprule
        Key & Meaning & Default & Allowed value \\
        \midrule
        title-skip          & Vertical space following the printed title     & 0pt          & dimension \\[10pt]
        print-skip          & Vertical space following a printing of entries & 1em          & dimension \\[10pt]
        entry-margin-left   & Horizontal margin left of an entry             & 1em          & dimension \\[10pt]
        entry-margin-bottom & Vertical margin below a printed entry          & 0.25em       & dimension \\[10pt]
        entry-padding       & Horizontal space between columns               & 0.75em       & dimension \\[10pt]
        main-section        & Section level for Main group                   & chapter      &  section  \\[10pt]
        group-section       & Section level for \cs{Group} command           & section      &  section  \\[10pt]
        subgroup-section    & Section level for \cs{Subgroup} command        & subsection   &  section  \\[10pt]
        main-title          & Title for the nomenclature                     & Nomenclature &   text    \\[10pt]
        group-title         & Title for the \cs{Group} command               & ---          &   text    \\[10pt]
        subgroup-title      & Title for the \cs{Subgroup} command            & ---          &   text    \\[10pt]
        include-in-toc      & Include the nomenclature in the TOC            & true         & boolean   \\[10pt]
        with-units          & Include a units column                         & false        & boolean   \\
        \bottomrule
    \end{tabular}
\end{center}
\end{table}

\begin{table}[H]
\begin{center}
    \caption{Additional key-value pairs for Acronym environment.}
    \label{Table:AcronymKeyValue}
    \begin{tabular}{c c c c}
        \toprule
        Key & Meaning & Default & Allow value \\
        \midrule
        use-links           & Create hyperlink to Acronym entry  & true & boolean \\[10pt]
        link-color          & Color of hyperlink text            & blue & color   \\
        \bottomrule
    \end{tabular}
\end{center}
\end{table}


%</UserGuide>
%
%
%
%
%
%<*Implementation>
%<<VERBATIM
%   \iffalse
%<*Code>
%   \fi
%
%
%  \UWModule{ListOf}
%
%   The ListOf Module is a collection of commands that enables the easy
%   creation and typsetting of Lists.
%
%   Lists are taken to be any collection of entries that is to be typeset
%   with a particular style.  For example, a simple Nomenclature could be
%   considered a list of (symbol, description) entries to be typeset with a
%   fixed style for all entires.  The \texttt{ListOf} commands create a system
%   specifically for this scenario.
%
%   Of course, as the commands description will show, lists can be much more
%   complicated that two items.  For the \texttt{ListOf} system to function, an
%   author really only needs to define the \texttt{ListOf}, create a command to push
%   (enqueue) entries on to the \texttt{ListOf} queue, and at some point tell the
%   \texttt{ListOf} to typeset the entries it has stored (if display of the content
%   is desired).
%
%
%
%   \begin{macro}{\UWMad_ListOf_Define:n}
%   Define a new \texttt{ListOf} with \marg{ID}. This command creates the
%   commands to store the section commands and title for each group,
%   the booleans to indicate if the sections should be numbered and
%   if the sections should be included in the table of contentst
%   (regardless of numbering), a hash to hold of the user-defined
%   hooks for the \texttt{ListOf}, and a queue to store the entries for
%   typesetting.
%
%    \begin{macrocode}
\cs_new:Nn \UWMad_ListOf_Define:n {
    \tl_const:cn {c__UWMad_ListOf#1_IsDefined_tl}{}
%
    \tl_new:c {g__UWMad_ListOf#1_Section_Main_tl}
    \tl_new:c {g__UWMad_ListOf#1_Section_Group_tl}
    \tl_new:c {g__UWMad_ListOf#1_Section_Subgroup_tl}
%
    \tl_new:c {g__UWMad_ListOf#1_Title_Main_tl}
    \tl_new:c {g__UWMad_ListOf#1_Title_Group_tl}
    \tl_new:c {g__UWMad_ListOf#1_Title_Subgroup_tl}
%
    \bool_new:c       {g__UWMad_ListOf#1_ClearAfterPrint_bool}
    \bool_gset_true:c {g__UWMad_ListOf#1_ClearAfterPrint_bool}
    \bool_new:c       {g__UWMad_ListOf#1_IsNumbered_bool}
    \bool_gset_true:c {g__UWMad_ListOf#1_IsNumbered_bool}
    \bool_new:c       {g__UWMad_ListOf#1_IncludeInTOC_bool}
    \bool_gset_true:c {g__UWMad_ListOf#1_IncludeInTOC_bool}
    \UWMad_Queue_Define:n               {g__ListOf#1_EntryQueue}
    \UWMad_Hash_Define:n                {g__ListOf#1_Hook}
}
%    \end{macrocode}
%   \end{macro}
%
%
%
%   \begin{macro}{\UWMad_ListOf_Delete:n}
%   Simply undefines all of the commands created in the \texttt{Define} command
%   above for the given \marg{ID}.
%
%    \begin{macrocode}
\cs_new:Nn \UWMad_ListOf_Delete:n {
    \cs_undefine:c {c__UWMad_ListOf#1_IsDefined_tl}
%
    \cs_undefine:c {g__UWMad_ListOf#1_Section_Main_tl}
    \cs_undefine:c {g__UWMad_ListOf#1_Section_Group_tl}
    \cs_undefine:c {g__UWMad_ListOf#1_Section_Subgroup_tl}
%
    \cs_undefine:c {g__UWMad_ListOf#1_Title_Main_tl}
    \cs_undefine:c {g__UWMad_ListOf#1_Title_Group_tl}
    \cs_undefine:c {g__UWMad_ListOf#1_Title_Subgroup_tl}
%
    \cs_undefine:c {g__UWMad_ListOf#1_ClearAfterPrint_bool}
    \cs_undefine:c {g__UWMad_ListOf#1_IsNumbered_bool}
    \cs_undefine:c {g__UWMad_ListOf#1_IncludeInTOC_bool}
    \UWMad_Queue_Delete:n   {g__ListOf#1_EntryQueue}
    \UWMad_Hash_Delete:n    {g__ListOf#1_Hook}
}
%    \end{macrocode}
%   \end{macro}
%
%
%
%   \begin{macro}{\UWMad_ListOf_IfDefined:nT}
%   Checks to see if a \texttt{ListOf} with \marg{ID} has been created and
%   errors if not.
%
%    \begin{macrocode}
\cs_new:Nn \UWMad_ListOf_IfDefined:nT {
    \__UWMad_IfDefined:nnnnT
        {c__UWMad_ListOf}
        {#1}
        {_IsDefined_tl}
        {ListOf}
        {#2}
}
%    \end{macrocode}
%   \end{macro}
%
%
%
%   \begin{macro}{
%   \UWMad_ListOf_MakeNumbered:n,\UWMad_ListOf_MakeNotNumbered:n}
%   Makes the current section of the \texttt{ListOf} with \marg{ID} numbered
%   or unnumbered (i.e., a star version).
%
%    \begin{macrocode}
\cs_new:Nn \UWMad_ListOf_MakeNumbered:n {
    \UWMad_ListOf_IfDefined:nT {#1} {
        \bool_set_true:c {g__UWMad_ListOf#1_IsNumbered_bool}
    }
}
\cs_new:Nn \UWMad_ListOf_MakeNotNumbered:n {
    \UWMad_ListOf_IfDefined:nT {#1} {
        \bool_set_false:c {g__UWMad_ListOf#1_IsNumbered_bool}
    }
}
%    \end{macrocode}
%   \end{macro}
%
%
%
%   \begin{macro}{\UWMad_ListOf_IfNumbered:nTF}
%   Branches to \marg{True Code} or \marg{False Code} depending on whether
%   the \texttt{ListOf} with \marg{ID} is numbered or not.
%
%    \begin{macrocode}
\cs_new:Nn \UWMad_ListOf_IfNumbered:nTF {
    \UWMad_ListOf_IfDefined:nT {#1} {
        \bool_if:cTF {g__UWMad_ListOf#1_IsNumbered_bool} {
            #2
        }{
            #3
        }
    }
}
%    \end{macrocode}
%   \end{macro}
%
%
%
%   \begin{macro}{
%   \UWMad_ListOf_IncludeInTOC:n,\UWMad_ListOf_DoNotIncludeInTOC:n}
%   Makes the current section of the \texttt{ListOf} with \marg{ID} appear in
%   the Table of Contents (TOC)  or not, regardless of if it is
%   numbered/unnumbered.
%
%    \begin{macrocode}
\cs_new:Nn \UWMad_ListOf_IncludeInTOC:n {
    \UWMad_ListOf_IfDefined:nT {#1} {
        \bool_set_true:c {c__UWMad_ListOf#1_IncludeInTOC_bool}
    }
}
\cs_new:Nn \UWMad_ListOf_DoNotIncludeInTOC:n {
    \UWMad_ListOf_IfDefined:nT {#1} {
        \bool_set_false:c {c__UWMad_ListOf#1_IncludeInTOC_bool}
    }
}
%    \end{macrocode}
%   \end{macro}
%
%
%
%   \begin{macro}{\UWMad_ListOf_IfIncludeInTOC:n}
%   Branches to \marg{True Code} or \marg{False Code} depending on whether
%   the \texttt{ListOf} with \marg{ID} is to be included or not.
%
%    \begin{macrocode}
\cs_new:Nn \UWMad_ListOf_IfIncludeInTOC:nTF {
    \UWMad_ListOf_IfDefined:nT {#1} {
        \bool_if:cTF {c__UWMad_ListOf#1_IncludeInTOC_bool} {
            #2
        }{
            #3
        }
    }
}
%    \end{macrocode}
%   \end{macro}
%
%
%
%   \begin{function}{
%       \UWMad_ListOf_SetTitle_Main:nn,
%       \UWMad_ListOf_SetTitle_Group:nn,
%       \UWMad_ListOf_SetTitle_Subgroup:nn}
%   \begin{syntax}
%       \cs{UWMad_ListOf_SetTitle_Main:nn}\Arg{ID}\Arg{Title}
%       \cs{UWMad_ListOf_SetTitle_Group:nn}\Arg{ID}\Arg{Title}
%       \cs{UWMad_ListOf_SetTitle_Subgroup:nn}\Arg{ID}\Arg{Title}
%   \end{syntax}
%   Sets the value of the title of the sections to \marg{Title} for the
%   \texttt{ListOf} with \marg{ID}
%
%    \begin{macrocode}
\cs_new:Nn \UWMad_ListOf_SetTitle_Main:nn {
    \UWMad_ListOf_IfDefined:nT {#1} {
        \tl_set:cn {g__UWMad_ListOf#1_Title_Main_tl}{#2}
    }
}
\cs_new:Nn \UWMad_ListOf_SetTitle_Group:nn {
    \UWMad_ListOf_IfDefined:nT {#1} {
        \tl_set:cn {g__UWMad_ListOf#1_Title_Group_tl}{#2}
    }
}
\cs_new:Nn \UWMad_ListOf_SetTitle_Subgroup:nn {
    \UWMad_ListOf_IfDefined:nT {#1} {
        \tl_set:cn {g__UWMad_ListOf#1_Title_Subgroup_tl}{#2}
    }
}
%    \end{macrocode}
%   \end{function}
%
%
%
%   \begin{function}{
%       \UWMad_ListOf_GetTitle_Main:nn,
%       \UWMad_ListOf_GetTitle_Group:nn,
%       \UWMad_ListOf_GetTitle_Subgroup:nn}
%   \begin{syntax}
%       \cs{UWMad_ListOf_GetTitle_Main:n}    \Arg{ID}
%       \cs{UWMad_ListOf_GetTitle_Group:n}   \Arg{ID}
%       \cs{UWMad_ListOf_GetTitle_Subgroup:n}\Arg{ID}
%   \end{syntax}
%   Retrieces the value of the title of the section for the
%   \texttt{ListOf} with \marg{ID}.
%
%    \begin{macrocode}
\cs_new:Nn \UWMad_ListOf_GetTitle_Main:n {
    \UWMad_ListOf_IfDefined:nT {#1} {
        \tl_use:c {g__UWMad_ListOf#1_Title_Main_tl}
    }
}
\cs_new:Nn \UWMad_ListOf_GetTitle_Group:n {
    \UWMad_ListOf_IfDefined:nT {#1} {
        \tl_use:c {g__UWMad_ListOf#1_Title_Group_tl}
    }
}
\cs_new:Nn \UWMad_ListOf_GetTitle_Subgroup:n {
    \UWMad_ListOf_IfDefined:nT {#1} {
        \tl_use:c {g__UWMad_ListOf#1_Title_Subgroup_tl}
    }
}
%    \end{macrocode}
%   \end{function}
%
%
%
%   \begin{function}{
%       \UWMad_ListOf_SetSection_Main:nn,
%       \UWMad_ListOf_SetSection_Group:nn,
%       \UWMad_ListOf_SetSection_Subgroup:nn}
%   \begin{syntax}
%       \cs{UWMad_ListOf_SetSection_Main:nn}\Arg{ID}\Arg{Section}
%       \cs{UWMad_ListOf_SetSection_Group:nn}\Arg{ID}\Arg{Section}
%       \cs{UWMad_ListOf_SetSection_Subgroup:nn}\Arg{ID}\Arg{Section}
%   \end{syntax}
%   Sets the value of the section level and (currently) the sectioning
%   command for a particular group to \Arg{Section} of the \texttt{ListOf}
%   with \Arg{ID}.
%
%    \begin{macrocode}
\cs_new:Nn \UWMad_ListOf_SetSection_Main:nn {
    \UWMad_ListOf_IfDefined:nT {#1} {
        \tl_set:cn {g__UWMad_ListOf#1_Section_Main_tl}{#2}
    }
}
\cs_new:Nn \UWMad_ListOf_SetSection_Group:nn {
    \UWMad_ListOf_IfDefined:nT {#1} {
        \tl_set:cn {g__UWMad_ListOf#1_Section_Group_tl}{#2}
    }
}
\cs_new:Nn \UWMad_ListOf_SetSection_Subgroup:nn {
    \UWMad_ListOf_IfDefined:nT {#1} {
        \tl_set:cn {g__UWMad_ListOf#1_Section_Subgroup_tl}{#2}
    }
}
%    \end{macrocode}
%   \end{function}
%
%
%
%   \begin{function}{
%       \UWMad_ListOf_GetSection_Main:n,
%       \UWMad_ListOf_GetSection_Group:n,
%       \UWMad_ListOf_GetSection_Subgroup:n}
%   \begin{syntax}
%       \cs{UWMad_ListOf_GetSection_Main:n}\Arg{ID}
%       \cs{UWMad_ListOf_GetSection_Group:n}\Arg{ID}
%       \cs{UWMad_ListOf_GetSection_Subgroup:n}\Arg{ID}
%   \end{syntax}
%   Gets the value of the section level for a particular group of the
%   \texttt{ListOf} with \Arg{ID}.
%
%    \begin{macrocode}
\cs_new:Nn \UWMad_ListOf_GetSection_Main:n {
    \UWMad_ListOf_IfDefined:nT {#1} {
        \tl_use:c {g__UWMad_ListOf#1_Section_Main_tl}
    }
}
\cs_new:Nn \UWMad_ListOf_GetSection_Group:n {
    \UWMad_ListOf_IfDefined:nT {#1} {
        \tl_use:c {g__UWMad_ListOf#1_Section_Group_tl}
    }
}
\cs_new:Nn \UWMad_ListOf_GetSection_Subgroup:n {
    \UWMad_ListOf_IfDefined:nT {#1} {
        \tl_use:c {g__UWMad_ListOf#1_Section_Subgroup_tl}
    }
}
%    \end{macrocode}
%   \end{function}
%
%
%
%   \begin{function}{\UWMad_ListOf_SetHook:nnn}
%   \begin{syntax}
%       \cs{UWMad_ListOf_SetHook:nnn}\Arg{ID}\Arg{Hook name}\Arg{Hook code}
%   \end{syntax}
%   Sets \Arg{Hook name} to \Arg{Hook code} for the \texttt{ListOf}
%   with \Arg{ID}.
%   The current hooks used are: \texttt{PrePush}, \texttt{PostPush},
%   \texttt{PrePrint}, and \texttt{PostPrint}.
%
%    \begin{macrocode}
\cs_new:Nn \UWMad_ListOf_SetHook:nnn {
    \UWMad_Hash_Set:nnn{g__ListOf#1_Hook}{#2}{#3}
}
%    \end{macrocode}
%   \end{function}
%
%
%
%   \begin{function}{\UWMad_ListOf_PushEntry:nn}
%   \begin{syntax}
%       \cs{UWMad_ListOf_PushEntry:nn} \Arg{ID}\Arg{Entry}
%   \end{syntax}
%   Pushes \Arg{Entry} on to the entry queue of the \texttt{ListOf} with \Arg{ID}.
%
%    \begin{macrocode}
\cs_new:Nn \UWMad_ListOf_PushEntry:nn {
    \UWMad_Hash_Get:nn   {g__ListOf#1_Hook}{PrePush}
    \UWMad_Queue_Push:nn {g__ListOf#1_EntryQueue}{#2}
    \UWMad_Hash_Get:nn   {g__ListOf#1_Hook}{PostPush}
}
%    \end{macrocode}
%   \end{function}
%
%
%
%   \begin{function}{\UWMad_ListOf_PrintEntries:n}
%   \begin{syntax}
%       \cs{UWMad_ListOf_PrintEntries:n}\Arg{ID}
%   \end{syntax}
%   Prints all entries currently in the \texttt{ListOf} queue with \marg{ID} and
%   clears the queue.  The \texttt{PrePrint} and \texttt{PostPrint} hooks
%   are also called here.
%
%    \begin{macrocode}
\cs_new:Nn \UWMad_ListOf_PrintEntries:n {
    \UWMad_Hash_Get:nn   {g__ListOf#1_Hook}{PrePrint}
    \UWMad_Queue_Walk:nn {g__ListOf#1_EntryQueue}{##1}
    \UWMad_Queue_Clear:n {g__ListOf#1_EntryQueue}
    \UWMad_Hash_Get:nn   {g__ListOf#1_Hook}{PostPrint}
}
%    \end{macrocode}
%   \end{function}
%
%
%
%   \begin{function}{\UWMad_ListOf_PrintTitle:nn}
%   \begin{syntax}
%       \cs{UWMad_ListOf_PrintTitle:nn}\Arg{ID}\Arg{Group}
%   \end{syntax}
%   Prints the title for the \Arg{Group} of the \texttt{ListOf} with
%   \Arg{ID} at the section indicated by its associated token
%   list.
%   Numbering and table of contents adding is done according to the
%   current values of their respective booleans.
%
%    \begin{macrocode}
\cs_new:Nn \__UWMad_ListOf_CurrentSectioningCommmand:n {}
\cs_new:Nn \UWMad_ListOf_PrintTitle:nn {

    \cs_set_eq:Nc
        \__UWMad_ListOf_CurrentSectioningCommmand:n
        {\tl_use:c{g__UWMad_ListOf#1_Section_#2_tl}}

    \UWMad_ListOf_IfNumbered:nTF {#1} {

        \tl_if_eq:nnTF {#2} {Main} {
            \UWMad_ListOf_IfIncludeInTOC:nTF {#1} { } {
                \int_set_eq:NN \l_tmpa_int \c@tocdepth
                \setcounter{tocdepth}{-1}
            }
        } { 
            \int_set_eq:NN \l_tmpa_int \c@tocdepth
            \setcounter{tocdepth}{-1}
        }


        \__UWMad_ListOf_CurrentSectioningCommmand:n
            {\tl_use:c {g__UWMad_ListOf#1_Title_#2_tl}}


        \tl_if_eq:nnTF #2 {Main} {
            \UWMad_ListOf_IfIncludeInTOC:nTF {#1} { } {
                \setcounter{tocdepth}{\l_tmpa_int}
            }
        } { 
            \setcounter{tocdepth}{\l_tmpa_int}
        }

    } {
    
        \cs_generate_variant:Nn \tl_if_eq:nnTF {onTF}
        \tl_set:Nn \l_tmpa_tl {Main}
    
        \phantomsection
        \__UWMad_ListOf_CurrentSectioningCommmand:n*
        {\tl_use:c {g__UWMad_ListOf#1_Title_#2_tl}}
        \tl_if_eq:onTF {#2} {Main} {
            \UWMad_ListOf_IfIncludeInTOC:nTF {#1} {
                \addcontentsline
                    {toc}
                    {\tl_use:c {g__UWMad_ListOf#1_Section_#2_tl}}
                    {\tl_use:c {g__UWMad_ListOf#1_Title_#2_tl}}
            } { }
        } { }
    }

}
%    \end{macrocode}
%   \end{function}
%
%
%
%   \begin{function}{\UWMad_ListOf_StartGroup:nn}
%   \begin{syntax}
%       \cs{UWMad_ListOf_StartGroup:n}\Arg{ID}\Arg{Group}
%   \end{syntax}
%   A shortcut command that prints the entires in the current queue
%   and then starts the next section by printing the title.
%
%    \begin{macrocode}
\cs_new:Nn \UWMad_ListOf_StartGroup:nn {
    \UWMad_ListOf_PrintEntries:n{#1}
    \UWMad_ListOf_PrintTitle:nn {#1}{#2}
}
%    \end{macrocode}
%   \end{function}
%
%
%
%
%
%   \UWSubModule{Nomenclature}
%   Dimensions that are calculated are declared first.
%    \begin{macrocode}
\dim_new:N \l__UWMad_Nomenclature_WidestSymbol_dim
\dim_new:N \l__UWMad_Nomenclature_WidestUnit_dim
\dim_new:N \l__UWMad_Nomenclature_Entry_WidthSymbol_dim
\dim_new:N \l__UWMad_Nomenclature_Entry_WidthUnits_dim
\dim_new:N \l__UWMad_Nomenclature_Entry_WidthDescription_dim
%    \end{macrocode}
%
%   Then user-adjustable dimensions are declared.
%    \begin{macrocode}
\dim_new:N \l__UWMad_Nomenclature_TitleSkip_dim
\dim_new:N \l__UWMad_Nomenclature_PrintSkip_dim
\dim_new:N \l__UWMad_Nomenclature_Entry_MarginLeft_dim
\dim_new:N \l__UWMad_Nomenclature_Entry_MarginBottom_dim
\dim_new:N \l__UWMad_Nomenclature_Entry_Padding_dim
%    \end{macrocode}
%
%   The token lists that hold the section and title of the
%   groups are declared
%    \begin{macrocode}
\tl_new:N \l__UWMad_Nomenclature_Section_Main_tl
\tl_new:N \l__UWMad_Nomenclature_Section_Group_tl
\tl_new:N \l__UWMad_Nomenclature_Section_Subgroup_tl
\tl_new:N \l__UWMad_Nomenclature_Title_Main_tl
\tl_new:N \l__UWMad_Nomenclature_Title_Group_tl
\tl_new:N \l__UWMad_Nomenclature_Title_Subgroup_tl
%    \end{macrocode}
%
%   Now the keys for user-customization are defined:
%    \begin{macrocode}
\clist_new:N   \g__UWMad_Nomenclature_KeyValuePairs_clist
\clist_gset:Nn \g__UWMad_Nomenclature_KeyValuePairs_clist {
%    \end{macrocode}
%
%   Adjustable dimensions:
%    \begin{macrocode}
    title-skip .dim_set:N = \l__UWMad_Nomenclature_TitleSkip_dim,
    print-skip .dim_set:N = \l__UWMad_Nomenclature_PrintSkip_dim,
    entry-margin-left .dim_set:N =
        \l__UWMad_Nomenclature_Entry_MarginLeft_dim,
    entry-margin-bottom .dim_set:N =
        \l__UWMad_Nomenclature_Entry_MarginBottom_dim,
    entry-padding .dim_set:N =
        \l__UWMad_Nomenclature_Entry_Padding_dim,
%    \end{macrocode}
%
%   Adjustable dimension defaults:
%    \begin{macrocode}
    title-skip          .default:n  = {0.00pt},
    print-skip          .default:n  = {1.00em},
    entry-margin-left   .default:n  = {1.00em},
    entry-margin-bottom .default:n  = {0.25em},
    entry-padding       .default:n  = {0.75em},
%    \end{macrocode}
%
%   Group section adjustments:
%    \begin{macrocode}
    main-section .code:n = {
        \tl_set:Nn
            \l__UWMad_Nomenclature_Section_Main_tl {#1}
    },
    group-section .code:n = {
        \tl_set:Nn
            \l__UWMad_Nomenclature_Section_Group_tl {#1}
    },
    subgroup-section .code:n = {
        \tl_set:Nn
            \l__UWMad_Nomenclature_Section_Subgroup_tl {#1}
    },
%    \end{macrocode}
%
%   The default nomenclature section is chapter.
%   Since the other two groups of empty by default, the Nomenclature
%   environment will handle them.
%    \begin{macrocode}
    main-section .default:n = chapter,
%    \end{macrocode}
%
%   Group title adjustments:
%    \begin{macrocode}
    main-title .code:n = {
        \tl_set:Nn
            \l__UWMad_Nomenclature_Title_Main_tl {#1}
    },
    group-title .code:n = {
        \tl_set:Nn
            \l__UWMad_Nomenclature_Title_Group_tl {#1}
    },
    subgroup-title .code:n = {
        \tl_set:Nn
            \l__UWMad_Nomenclature_Title_Subgroup_tl {#1}
    },
%    \end{macrocode}
%
%   Group title default for main group only:
%    \begin{macrocode}
    main-title .default:n = Nomenclature,
%    \end{macrocode}
%
%   Miscellaneous options:
%    \begin{macrocode}
    numbered       .bool_gset:N = 
        \g__UWMad_Nomenclature_IsNumbered_bool,
    include-in-toc .bool_gset:N =
        \g__UWMad_Nomenclature_IncludeInTOC_bool,
    with-units .bool_gset:N =
        \g__UWMad_Nomenclature_IncludeUnitsColumn_bool,
%    \end{macrocode}
%
%   Miscellaneous option defaults:
%    \begin{macrocode}
      numbered     .default:n = false,
    include-in-toc .default:n = true,
     with-units    .default:n = false
}
\exp_args:Nnf
    \keys_define:nn
    { UWMadThesis / Nomenclature }
    {
        \clist_use:Nn \g__UWMad_Nomenclature_KeyValuePairs_clist {,}
    }
%    \end{macrocode}
%
%   And the defaults for all keys are now set.
%    \begin{macrocode}
\keys_set:nn { UWMadThesis / Nomenclature } {
    title-skip          ,
    print-skip          ,
    entry-margin-left   ,
    entry-margin-bottom ,
    entry-padding       ,
    numbered            ,
    include-in-toc      ,
    with-units          ,
    main-section        ,
    main-title
}
%    \end{macrocode}
%
%
%
%   And the defaults for all keys are now set.
%   \begin{function} {
%       \UWMad_Nomenclature_UpdateWidest:Nn,
%       \UWMad_Nomenclature_UpdateWidest_Symbol:n,
%       \UWMad_Nomenclature_UpdateWidest_Units:n,
%       }
%       \begin{syntax}    
%           \cs{UWMad_Nomenclature_UpdateWidest:Nn}\meta{dim}\marg{object}
%           \cs{UWMad_Nomenclature_UpdateWidest_Symbol:n}\marg{symbol}
%           \cs{UWMad_Nomenclature_UpdateWidest_Units:n}\marg{units}
%       \end{syntax}
%       These commands update the widest symbol and widest unit lengths.
%    \begin{macrocode}
\cs_new:Nn \UWMad_Nomenclature_UpdateWidest:Nn {
    \hbox_set:Nn \l_tmpa_box {#2}
    \dim_set:Nn  \l_tmpa_dim {\box_wd:N \l_tmpa_box}
    \dim_compare:nNnTF {#1} < {\l_tmpa_dim} {
        \dim_set:Nn #1 {\l_tmpa_dim}
    } { }
}
\cs_new:Nn \UWMad_Nomenclature_UpdateWidest_Symbol:n {
    \UWMad_Nomenclature_UpdateWidest:Nn
        \l__UWMad_Nomenclature_WidestSymbol_dim {#1}
}
%
\cs_new:Nn \UWMad_Nomenclature_UpdateWidest_Units:n {
    \UWMad_Nomenclature_UpdateWidest:Nn
        \l__UWMad_Nomenclature_WidestUnit_dim {#1}
}
%
%
\cs_new:Nn \UWMad_Nomenclature_ZeroWidest_Symbol: {
    \dim_set:Nn \l__UWMad_Nomenclature_WidestSymbol_dim {0pt}
}
\cs_new:Nn \UWMad_Nomenclature_ZeroWidest_Unit: {
    \dim_set:Nn \l__UWMad_Nomenclature_WidestUnit_dim {0pt}
}
%    \end{macrocode}
%   \end{function}
%    \begin{macrocode}
%
%
% ======================================================================= %
%                             Set Entry Widths                            %
% ======================================================================= %
\cs_new:Nn \UWMad_Nomenclature_SetEntryWidths_NoUnits: {
    %
    % Define symbol width
    \dim_set:Nn \l__UWMad_Nomenclature_Entry_WidthSymbol_dim {
        1.01\l__UWMad_Nomenclature_WidestSymbol_dim
    }
    %
    % Define description width
    \dim_set:Nn \l__UWMad_Nomenclature_Entry_WidthDescription_dim {
        0.995\textwidth -
        \l__UWMad_Nomenclature_Entry_MarginLeft_dim  -
        \l__UWMad_Nomenclature_Entry_WidthSymbol_dim -
        \l__UWMad_Nomenclature_Entry_Padding_dim
    }
}
%
\cs_new:Nn \UWMad_Nomenclature_SetEntryWidths_Units: {
    %
    % Define symbol width
    \dim_set:Nn \l__UWMad_Nomenclature_Entry_WidthSymbol_dim {
        1.01\l__UWMad_Nomenclature_WidestSymbol_dim
    }
    %
    % Define unit width
    \dim_set:Nn \l__UWMad_Nomenclature_Entry_WidthUnit_dim {
        1.01\l__UWMad_Nomenclature_WidestUnit_dim
    }
    %
    % Define description width
    \dim_set:Nn \l__UWMad_Nomenclature_Entry_WidthDescription_dim {
        0.995\textwidth -
         \l__UWMad_Nomenclature_Entry_MarginLeft_dim  -
         \l__UWMad_Nomenclature_Entry_WidthSymbol_dim -
         \l__UWMad_Nomenclature_Entry_WidthUnit_dim   -
        2\l__UWMad_Nomenclature_Entry_Padding_dim
    }
}
%
\cs_new:Nn \UWMad_Nomenclature_SetEntryWidths: {
    \bool_if:NTF \g__UWMad_Nomenclature_IncludeUnitsColumn_bool {
        \UWMad_Nomenclature_SetEntryWidths_Units:
    } {
        \UWMad_Nomenclature_SetEntryWidths_NoUnits:
    }
}
%
%
%
%
% ======================================================================= %
%                              Typeset Entry                              %
% ======================================================================= %
%
\coffin_new:N \l_tmpc_coffin
%
\cs_new:Nn \UWMad_Nomenclature_SetEntry_NoUnits:nn {
% Set the entry material in the temporary coffins
    \vcoffin_set:Nnn
        \l_tmpa_coffin
        {\l__UWMad_Nomenclature_Entry_WidthSymbol_dim}
        {#1}
    \vcoffin_set:Nnn
        \l_tmpb_coffin
        {\l__UWMad_Nomenclature_Entry_WidthDescription_dim}
        {#2}
%
% Typeset the material and skips
    \group_begin:
        \setstretch{1.1}
        \skip_horizontal:n {\l__UWMad_Nomenclature_Entry_MarginLeft_dim}
        \coffin_typeset:Nnnnn \l_tmpa_coffin {l}{t}{0pt}{0pt}
        \skip_horizontal:n {\l__UWMad_Nomenclature_Entry_Padding_dim}
        \coffin_typeset:Nnnnn \l_tmpb_coffin {l}{t}{0pt}{0pt}
        \skip_vertical:n {\l__UWMad_Nomenclature_Entry_MarginBottom_dim}
    \group_end:
}
%
\cs_new:Nn \UWMad_Nomenclature_SetEntry_Units:nnn {
% Set the entry material in the temporary coffins
    \vcoffin_set:Nnn
        \l_tmpa_coffin
        {\l__UWMad_Nomenclature_Entry_WidthSymbol_dim}
        {#1}
    \vcoffin_set:Nnn
        \l_tmpb_coffin
        {\l__UWMad_Nomenclature_Entry_WidthUnit_dim}
        {#2}
    \vcoffin_set:Nnn
        \l_tmpc_coffin
        {\l__UWMad_Nomenclature_Entry_WidthDescription_dim}
        {#3}
%
% Typeset the material and skips
    \group_begin:
        \setstretch{1.1}
        \skip_horizontal:n {\l__UWMad_Nomenclature_Entry_MarginLeft_dim}
        \coffin_typeset:Nnnnn \l_tmpa_coffin {l}{t}{0pt}{0pt}
        \skip_horizontal:n {\l__UWMad_Nomenclature_Entry_Padding_dim}
        \coffin_typeset:Nnnnn \l_tmpb_coffin {l}{t}{0pt}{0pt}
        \skip_horizontal:n {\l__UWMad_Nomenclature_Entry_Padding_dim}
        \coffin_typeset:Nnnnn \l_tmpc_coffin {l}{t}{0pt}{0pt}
        \skip_vertical:n {\l__UWMad_Nomenclature_Entry_MarginBottom_dim}
    \group_end:
}
%
\cs_new:Nn \UWMad_Nomenclature_SetEntry: {
    \bool_if:NTF \g__UWMad_Nomenclature_IncludeUnitsColumn_bool {
        \UWMad_Nomenclature_SetEntry_Units:
    } {
        \UWMad_Nomenclature_SetEntry_NoUnits:
    }
}
%
%
%
%
%
% ======================================================================= %
%                              User Front-Ends                            %
% ======================================================================= %
\DeclareDocumentCommand \NomenclatureSetup { m } {
    \keys_set:nn { UWMadThesis / Nomenclature } { #1 }
}
%
%
\DeclareDocumentEnvironment {Nomenclature} { o g } {
%
%   Create the ListOf
    \UWMad_ListOf_Define:n {Nomenclature}
%
%
%   Check for an optional section declaration and
%   set Main section token list.
    \IfNoValueTF {#1} { } {
        \UWMad_IfSectionExists:nT {#1} { }
        \tl_set:Nn \l__UWMad_Nomenclature_Section_Main_tl {#1}
    }
%
%
%   Check for an optional section declaration and
%   set Main section token list.
    \IfNoValueTF {#2} { } {
        \tl_set:Nf
            \l__UWMad_Nomenclature_Title_Main_tl {#2}
    }
    \UWMad_ListOf_SetTitle_Main:nn {Nomenclature}
        {\l__UWMad_Nomenclature_Title_Main_tl}
%
%
   \bool_if:NTF \g__UWMad_Nomenclature_IsNumbered_bool {
        \UWMad_ListOf_MakeNumbered:n    {Nomenclature}
    }{
        \UWMad_ListOf_MakeNotNumbered:n {Nomenclature}
    }
%
%
%
%   If Group section token list is empty, set it to the following
%   section after Main in the LaTeX sectioning hierarchy.
%   Otherwise, take the value at its word.
    \tl_if_empty:NTF \l__UWMad_Nomenclature_Section_Group_tl {
        \tl_set:Nf \l__UWMad_Nomenclature_Section_Group_tl {
            \UWMad_NextSection:n {
                \l__UWMad_Nomenclature_Section_Main_tl
            }
        }
    } { }
%
%   If Subgroup section token list is empty, set it to the following
%   section after Group in the LaTeX sectioning hierarchy.
%   Otherwise, take the value at its word.
    \tl_if_empty:NTF \l__UWMad_Nomenclature_Section_Subgroup_tl {
        \tl_set:Nf \l__UWMad_Nomenclature_Section_Subgroup_tl {
            \UWMad_NextSection:n {
                \l__UWMad_Nomenclature_Section_Group_tl
            }
        }
    } { }
%
%   Set the sections with the Nomenclature ListOf instance
    \UWMad_ListOf_SetSection_Main:nn
        {Nomenclature} {\l__UWMad_Nomenclature_Section_Main_tl}
    \UWMad_ListOf_SetSection_Group:nn
        {Nomenclature} {\l__UWMad_Nomenclature_Section_Group_tl}
    \UWMad_ListOf_SetSection_Subgroup:nn
        {Nomenclature} {\l__UWMad_Nomenclature_Section_Subgroup_tl}
%
%
%   Determine if this nomenclature should be in the Table of Contents
    \bool_if:NTF \g__UWMad_Nomenclature_IncludeInTOC_bool {
        \UWMad_ListOf_IncludeInTOC:n {Nomenclature}
    } {
        \UWMad_ListOf_DoNotIncludeInTOC:n {Nomenclature}
    }
%
%   Set some hooks in the Nomenclature ListOf instance
    \UWMad_ListOf_SetHook:nnn {Nomenclature} {PrePrint} {
        \UWMad_Nomenclature_SetEntryWidths:
    }
    \UWMad_ListOf_SetHook:nnn {Nomenclature} {PostPrint} {
        \UWMad_Nomenclature_ZeroWidest_Symbol:
    }
%
%
%   User front-end for creating a Group
    \DeclareDocumentCommand \Group { G{} } {
        \IfNoValueTF {##1} { } {
            \tl_set:Nn
                \l__UWMad_Nomenclature_Title_Group_tl {##1}
        }
        \UWMad_ListOf_SetTitle_Group:nn {Nomenclature}
            {\l__UWMad_Nomenclature_Title_Group_tl}
        \UWMad_ListOf_StartGroup:nn{Nomenclature}{Group}
    }
%
%   User front-end for creating a Subgroup
    \DeclareDocumentCommand \Subgroup { G{} } {
        \IfNoValueTF {##1} { } {
            \tl_set:Nn
                \l__UWMad_Nomenclature_Title_Subgroup_tl {##1}
        }
        \UWMad_ListOf_SetTitle_Subgroup:nn {Nomenclature}
            {\l__UWMad_Nomenclature_Title_Subgroup_tl}
        \UWMad_ListOf_StartGroup:nn{Nomenclature}{Subgroup}
    }
%
%   User front-end for creating an entry
    \bool_if:NTF \g__UWMad_Nomenclature_IncludeUnitsColumn_bool {
        \DeclareDocumentCommand \Entry { m m m } {
            \UWMad_ListOf_PushEntry:nn {Nomenclature} {
                \UWMad_Nomenclature_SetEntry_Units:nnn
                    {##1} {##2} {##3}
            }
            \UWMad_Nomenclature_UpdateWidest_Symbol:n{##1}
        }
    } {
        \DeclareDocumentCommand \Entry { m m } {
            \UWMad_ListOf_PushEntry:nn {Nomenclature} {
                \UWMad_Nomenclature_SetEntry_NoUnits:nn
                    {##1} {##2}
            }
            \UWMad_Nomenclature_UpdateWidest_Symbol:n{##1}
        }
    }
%
%   User front-end for reseting the column width
    \DeclareDocumentCommand  \ResetColumnWidth { } {
        \UWMad_Nomenclature_ZeroWidest_Symbol:
        \UWMad_Nomenclature_ZeroWidest_Unit:
    }
%
%
%   Print the main section title
    \UWMad_ListOf_PrintTitle:nn {Nomenclature}{Main}
%
} {
%   Flush the remaining entries from the ListOf queue and
%   delete the Nomenclature ListOf instance.
    \UWMad_ListOf_PrintEntries:n {Nomenclature}
    \UWMad_ListOf_Delete:n{Nomenclature}
}
%
%
%
%
\DeclareDocumentEnvironment {Acronym} { o G{Acronym} } {

    \begin{Nomenclature}[#1]{#2}
%
%
    \UWMad_Hash_Define:n{Acronyms}
    \UWMad_Hash_Define:n{AcronymMeanings}
%
%
    \cs_undefine:N \Entry
    \DeclareDocumentCommand \Entry { o m m } {
        \IfNoValueTF {##1} {

            \UWMad_Hash_Set:nnn{Acronyms}       {##2}{##2}
            \UWMad_Hash_Set:nnn{AcronymMeanings}{##2}{##3}
            \bool_new:c {g__UWMad_Acronym_WasSet_##2_bool}
            %
            \UWMad_ListOf_PushEntry:nn {Nomenclature} {
                \hypertarget{Acronym:##2}{}
                \UWMad_Nomenclature_SetEntry_NoUnits:nn
                    {##2} {##3}
            }

        } {

            \UWMad_Hash_Set:nnn{Acronyms}       {##1}{##2}
            \UWMad_Hash_Set:nnn{AcronymMeanings}{##1}{##3}
            \bool_new:c {g__UWMad_Acronym_WasSet_##1_bool}
            %
            \UWMad_ListOf_PushEntry:nn {Nomenclature} {
                \hypertarget{Acronym:##1}{}
                \UWMad_Nomenclature_SetEntry_NoUnits:nn
                    {##2} {##3}
            }

        }
        \UWMad_Nomenclature_UpdateWidest_Symbol:n{##2}
    }
} {

    \end{Nomenclature}

}

\DeclareDocumentCommand \Acro { m } {
    \UWMad_Hash_IfKeySet:nnTF {Acronyms} {#1} {
        \bool_if:cTF {g__UWMad_Acronym_WasSet_#1_bool} {
            \hyperlink{Acronym:#1}{
                \UWMad_Hash_Get:nn{Acronyms}{#1}
            }
        } {
            \UWMad_Hash_Get:nn{AcronymMeanings}{#1}~
                (
                    \UWMad_Hash_Get:nn{Acronyms}{#1}
                )
            \bool_gset_true:c {g__UWMad_Acronym_WasSet_#1_bool}
        }
    } { }
}
%
%
\cs_new_eq:NN \AcronymSetup \NomenclatureSetup
%
%
%
%    \end{macrocode}
%
%
%
%
%
%   \iffalse
%</Code>
%   \fi
%VERBATIM
%</Implementation>