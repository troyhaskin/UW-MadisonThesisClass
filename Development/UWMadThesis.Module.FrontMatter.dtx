%<*UserGuide>


\UWFeature{Getting Started}

\UWSubFeature{Options On-Load}
\UWSubFeature{Feature Options}
\UWSubFeature{Identification Commands}



%</UserGuide>
%
%
%
%
%
%
%<*Implementation>
%<<Verbatim
%   \iffalse
%<*Code>
%   \fi
%
%
%   \UWPart{Implementation}
%   \UWModule{Front Matter}
%
%   Much of this class is written using the \LaTeX3 Programming Layer;
%   this will be denoted as \LaTeXPL{}.  The \LaTeXPL{} is the first
%   piece of a new system designed to succeed \LaTeXe{} in the future.
%   However, while the programming layer is solid and remarkable,
%   a lot of presentation work still needs to be done.  Therefore,
%   this class uses \LaTeXe{} code where necessary and will hopefully
%   be slowly pulled out as needed.  The good news is that since everything
%   is more-or-less an abstraction of \TeX{}, it should work together well.
%
%   \UWSubModule{expl3 Package and Identification}
%   The |expl3| package loads the \LaTeXPL{} and is therefore required.
%   If the package is not recent enough, the class aborts and requests
%   the user update.
%    \begin{macrocode}
\RequirePackage{expl3}[2013/07/28]
\@ifpackagelater{expl3}{2013/07/28} {} {%
    \PackageError{UWMadThesis}{Version of l3kernel is too old}
      {%
        Please install an up to date version of l3kernel\MessageBreak
        using your TeX package manager or from CTAN.
      }%
    \endinput
}%
%    \end{macrocode}
%
%
%    \begin{macrocode}
\ExplSyntaxOn
%    \end{macrocode}
%
%
%   \UWSubModule{Identification and Defaults}
%
%   If the |expl3| package is recent enoughw, define some identification
%   variables (token lists).
%    \begin{macrocode}
\tl_const:Nn \c__UWMad_Class_Name_tl        {UWMadThesis}
\tl_const:Nn \c__UWMad_Class_Version_tl     {1.0}
\tl_const:Nn \c__UWMad_Class_Date_tl        {2014/04/01}
\tl_const:Nn \c__UWMad_Class_Description_tl {
    LaTeX2e+~Thesis~Class~for~UW-Madison
}
\tl_const:Nn \c__UWMad_UniversityLong_tl   {University~of~Wisconsin--Madison}
\tl_const:Nn \c__UWMad_UniversityShort_tl  {UW--Madison}
%    \end{macrocode}
%
%   Assuming the the |expl3| package is recent enough, we provide the class
%   using the \LaTeXPL{}'s provide function.
%    \begin{macrocode}
\ProvidesExplClass
    {\c__UWMad_Class_Name_tl}   {\c__UWMad_Class_Date_tl}
    {\c__UWMad_Class_Version_tl}{\c__UWMad_Class_Description_tl}
%    \end{macrocode}
%
%
%
%   In an effort to allow the thesis class to adapt to new underlying classes,
%   the class that the \UWMadClass{} loads is decalred as a mutable
%   token list.  The default is the \LaTeX{} base class \texttt{report}.
%    \begin{macrocode}
\tl_new:N   \g_UWMad_ParentClass_tl
\tl_gset:Nn \g_UWMad_ParentClass_tl {report}
%    \end{macrocode}
%
%
%
%   \UWSubModule{Options}
%
%
%   First, a command is created to throw a warning if an option that
%   violates \UWMadLong{}'s dissertation guidelines is chosen.
%    \begin{macrocode}
\msg_new:nnn{ UWMadThesis }{ Options / StyleViolation }{
    Option~'#1'~violates~\c_UWMadUniversityShort_tl{}~
    Dissertation~Guidelines;~consider~omission
}
\cs_new:Nn \__UWMad_FrontMatter_StyleWarning:n {
    \msg_warning:nnn { UWMadThesis }{ Options / StyleViolation } { #1 }
   \PassOptionsToClass{#1}{\g_UWMad_ParentClass_tl}
}
%    \end{macrocode}
%
%
%
%   Now, declare booleans for the option processing.  All new booleans
%   are false by default.
%    \begin{macrocode}
\bool_new:N       \g__UWMad_MathTweaks_bool
\bool_gset_true:N \g__UWMad_MathTweaks_bool
%    \end{macrocode}
%
%   Declare the options.
%    \begin{macrocode}
\DeclareOption{NoMath} {
    \bool_gset_false:N \g__UWMad_MathTweaks_bool
}
\DeclareOption{Quiet} {
    \msg_redirect_module:nnn { UWMadThesis } { warning } { none }
}
%    \end{macrocode}
%
%   Catch the couple of default options that violate the requirements: 
%   8.5 by 11 paper for single-sided printing.
%    \begin{macrocode}
\DeclareOption{a4paper} {
    \__UWMad_FrontMatter_StyleWarning:n {\CurrentOption}
}
\DeclareOption{twoside} {
    \__UWMad_FrontMatter_StyleWarning:n {\CurrentOption}
}
%    \end{macrocode}
%
%   These options change the default report class to the
%   ones indicated.
%    \begin{macrocode}
\DeclareOption{article} {
    \tl_gset:Nn \g_UWMad_ParentClass_tl {article}
}
%    \end{macrocode}
%
%   This is a special class option for generating the documentation.
%   Users should not use this unless they know what they're doing.
%   The line below the \texttt{ParentClass} class prevents the \pkg{thumbpdf}
%   package from being loaded.
%    \begin{macrocode}
\DeclareOption{l3doc} {
    \tl_gset:Nn \g_UWMad_ParentClass_tl {l3doc}
    \tl_const:cn {ver@thumbpdf.sty} {}
}
%    \end{macrocode}
%
%   Pass all remaining options to the base class.
%    \begin{macrocode}
\DeclareOption*{
    \PassOptionsToClass
        {\CurrentOption}{\g_UWMad_ParentClass_tl}
}
%    \end{macrocode}
%
%   Process the options with some defaults and load the base class.
%    \begin{macrocode}
\ExecuteOptions{oneside,12pt}
\ProcessOptions\relax
\LoadClass{\g_UWMad_ParentClass_tl}[1995/12/01]
%    \end{macrocode}
%
%
%
%   \UWSubModule{Package Loads}
%   Load some packages that give nice features and are not
%   hyperlink sensitive.
%    \begin{macrocode}
\RequirePackage{xparse}
\RequirePackage{fixltx2e}
\RequirePackage{microtype}
\RequirePackage{array}
\RequirePackage{float}
\RequirePackage{graphicx}
\RequirePackage{setspace}
\RequirePackage{geometry}
%    \end{macrocode}
%
%
%   Load the \AmS{} suite.
%    \begin{macrocode}
\RequirePackage{amsmath}
\RequirePackage{amsfonts}
\RequirePackage{amssymb}
\RequirePackage{mathtools}
%    \end{macrocode}
%
%   And now we load some packages that give nice features and are
%   hyperlink sensitive.
%    \begin{macrocode}
\RequirePackage[noabbrev,nameinlink]{cleveref}
\RequirePackage[usenames,dvipsnames,svgnames,table,hyperref]{xcolor}
\RequirePackage{caption}
\RequirePackage{subcaption}
%    \end{macrocode}
%
%
%   Conditionally load either the \pkg{polyglossia} or \pkg{babel}
%   language packages depending on the engine in use.
%    \begin{macrocode}
\bool_if:nTF {\xetex_if_engine_p: || \luatex_if_engine_p:} {
    \RequirePackage{fontspec}
    \defaultfontfeatures{Ligatures={TeX}}
    \setmainfont
        [SmallCapsFont = {Latin~Modern~Roman~Caps}]
        {Latin~Modern~Roman}
%
    \RequirePackage{polyglossia}
    \setmainlanguage[variant = usmax]{english}
} {
    \RequirePackage{lmodern}
    \RequirePackage[T1]{fontenc}
%
    \RequirePackage[english]{babel}
}
%    \end{macrocode}
%
%
%   If links were not negated by the options, \pkg{bookmark} and 
%   \pkg{hyperref} are loaded.
%    \begin{macrocode}
\RequirePackage{hyperref}
\RequirePackage{bookmark}
%    \end{macrocode}
%
%
%
%
%   And since these identifications may be desired in typsetting more,
%   where \LaTeXPL{}'s syntax will be turned off, we define some aliases.
%    \begin{macrocode}
\DeclareDocumentCommand \UWMadClass { } {
    \texttt{\c__UWMad_Class_Name_tl}~class
}
\DeclareDocumentCommand \UWMadClassVersion { } {
    \c__UWMad_Class_Version_tl
}
\DeclareDocumentCommand \UWMadClassDate { } {
    \c__UWMad_Class_Date_tl
}
\DeclareDocumentCommand \UWMadLong { } {
    \c__UWMad_UniversityLong_tl
}
\DeclareDocumentCommand \UWMadShort { } {
    \c__UWMad_UniversityShort_tl
}
%    \end{macrocode}
%
%
%
%
%
%
%
%
%   \UWSubModule{Key-Value Interface}
%
%   \begin{function}{\UWMadSetup}
%       \begin{syntax}
%           \cs{UWMadSetup}\Arg{module name}\Arg{key-value csv}
%       \end{syntax}
%   This simple command creates a user interface for the key-value system used
%   for several feature set options.
%    \begin{macrocode}
\DeclareDocumentCommand \UWMadSetup { >{\TrimSpaces} m m } {
    \keys_set:nn
        { UWMadThesis / #1 }
        { #2 }
}
%    \end{macrocode}
%   \end{function}
%
%   \iffalse
%</Code>
%   \fi
%Verbatim
%</Implementation>