%<*UserGuide>

\UWFeature{Layout And Style}

The \UWMadClass{} has several default styling differences from the standard \LaTeXe{} class it is based on.
Some of these changes exist to abide by the \UWMadShort{} dissertation guidelines and others are based on the author's preferences.
They are, however, readily changeable using the facilities of the packages used to make the changes.
The defaults and methods for changing the styles are list in this section or the references manuals.


\UWSubFeature{Captions}
The \UWMadClass{} uses the \pkg{caption} and \pkg{subcaption} packages to style float captions and subcaptions.
It is possible to adjust the defaults showcased below by using the packages' utilities outlined in their respective manuals.

\vskip1em
{
    \captionof {figure} {Here is an example of a figure caption.
        The default style for the \UWMadClass{} is a slanted font (abbrev. \enquote{sl}) and small capitals (abbrev. \enquote{sc}) for the float label.
        Notice that long captions, like this, are indented such that the caption text is visibly separated from the float label.
    }

    \captionof {table} {Here is a shorter example of a table caption.
        The default styling is identical to the figure caption.
    }
}


\UWSubFeature{Links}
The \UWMadClass{} loads the \pkg{hyperref} and \pkg{bookmark} packages to create hyperlinks and a clickable documents.
The default color for document links is \textcolor{blue}{blue}, for urls is \textcolor{violet}{violet}, and for citations is \textcolor{UWMadGreen}{UWMadGreen} (a darker version of \textcolor{green}{green}).
These defaults can be changed using the commands below or the facilities of the \pkg{hyperref} package as described in its manual.
New colors can be created using the facilities of the \pkg{xcolor} package as described in its manual.


\UWSubSubFeature{Link Colors}
To more easily facilitate color changes to links, several user interface commands have been defined.

\begin{function} {\MakeLinksTheseColors}
    \begin{syntax}
        \cs{MakeLinksTheseColors}\Arg{link color}\Arg{cite color}\Arg{url color}
    \end{syntax}
    Redefines the colors used for (internal) links, cites, and URLs.
    Any valid color, including those defined by the \pkg{xcolor} package, is allowed for all three, required arguments.
\end{function}
\begin{function} {\MakeLinksThisColor}
    \begin{syntax}
        \cs{MakeLinksThisColor}\Arg{color}
    \end{syntax}
    Redefines the colors used for (internal) links, cites, and URLs to be the single indicated color.
    Any valid color, including those defined by the \pkg{xcolor} package, is allowed for the one required arguments.
\end{function}

\begin{function} {\MakeLinksBlack,\MakeLinksBlue,\MakeLinksRed}
    \begin{syntax}
        \cs{MakeLinksBlack}
        \cs{MakeLinksBlue}
        \cs{MakeLinksRed}
    \end{syntax}
    These commands take no argument and define all links to have the color indicated in the command name.
\end{function}


\UWSubSubFeature{References}
References may be handled by the \pkg{hyperref} package using \cs{autocite} or by the \pkg{cleveref} package using \cs{cref}/\cs{Cref} (the latter producing a capital letter for the reference type).
The user is referred to their respective manuals for more options and feature descriptions.




\UWSubFeature{Paragraph Spacing}
In general, there are two dominant methods for indicating separate paragraphs: no indentation with extra space between paragraphs (compared to between lines) and indentation with no extra space between paragraphs.
The default of the \UWMadClass{} is the former but some may prefer the latter.
To facilitate either, two commands have been created.

\begin{function} {\PadParagraphs}
    \begin{syntax}
        \cs{PadParagraphs}
    \end{syntax}
    This command adds 1em of vertical space between paragraphs with no indentation.
    This is the default style of this class.
\end{function}

\begin{function} {\IndentParagraphs}
    \begin{syntax}
        \cs{IndentParagraphs}
    \end{syntax}
    This command adds 1.5em of indent at the beginning of paragraphs (save those that follow section heads) with no extra vertical space.
\end{function}

%</UserGuide>
%
%
%
%
%
%
%<*Implementation>
%<<Verbatim
%   \iffalse
%<*Code>
%   \fi
%
%
%   \UWModule{Layout And Styles}
%
%
%   \UWSubModule{Float Styles}
%
%   Make equation references of the form (\#).
%    \begin{macrocode}
\creflabelformat{equation}{#2#1#3}
%    \end{macrocode}
%
%   Default table style
%       \begin{macrocode}
\captionsetup [table] {
    format        = hang                ,
    labelsep      = colon               ,
    justification = justified           ,
    labelfont     = sc                  ,
    textfont      = sl                  ,
    font          = {normal,stretch=1.1},
    width         = 0.9\textwidth       ,
    position      = above               ,
    skip          = 0.50em
}
%    \end{macrocode}
%
%   Default figure style.
%    \begin{macrocode}
\captionsetup [figure] {
    format        = hang                ,
    labelsep      = colon               ,
    justification = justified           ,
    labelfont     = sc                  ,
    textfont      = sl                  ,
    font          = {normal,stretch=1.1},
    width         = 0.9\textwidth       ,
    position      = above               ,
    skip          = 0.5em
}
%    \end{macrocode}
%
%
%
%
%
%
%   \UWSubModule{Links}
%
%   Define a darker green than |green|.
%    \begin{macrocode}
\definecolor{UWMadGreen}{rgb}{0,0.7,0}
%    \end{macrocode}
%
%   Define the default colors for the (internal) links, cites, and URLs.
%    \begin{macrocode}
\tl_new:N  \l_UWMad_LayoutStyle_Color_Link_tl
\tl_set:Nn \l_UWMad_LayoutStyle_Color_Link_tl {blue}
\tl_new:N  \l_UWMad_LayoutStyle_Color_Cite_tl
\tl_set:Nn \l_UWMad_LayoutStyle_Color_Cite_tl {UWMadGreen}
\tl_new:N  \l_UWMad_LayoutStyle_Color_URL_tl
\tl_set:Nn \l_UWMad_LayoutStyle_Color_URL_tl  {violet}
%    \end{macrocode}
%
%   Define a new color and hyperlink defaults
%    \begin{macrocode}
\hypersetup {
    colorlinks         = true,
    linkcolor          = \l_UWMad_LayoutStyle_Color_Link_tl,
    citecolor          = \l_UWMad_LayoutStyle_Color_Cite_tl,
    urlcolor           = \l_UWMad_LayoutStyle_Color_URL_tl,
    pdfdisplaydoctitle = true,
    pdfview            = {FitH},
    pdfstartview       = {FitH},
    pdfpagelayout      = OneColumn,
    plainpages         = false,
    hypertexnames      = true,
    bookmarksopenlevel = 1,
    bookmarksopen      = true,
    unicode            = true
}
%    \end{macrocode}
%
%   Define a helper commands to redefine all of the \pkg{hyperref}
%   link colors using this class's color token lists.
%    \begin{macrocode}
\cs_new:Nn \UWMad_LayoutStyle_ResetLinkColors: {
    \hypersetup {
        colorlinks = true,
        linkcolor  = \l_UWMad_LayoutStyle_Color_Link_tl,
        citecolor  = \l_UWMad_LayoutStyle_Color_Cite_tl,
        urlcolor   = \l_UWMad_LayoutStyle_Color_URL_tl
    }
}
%    \end{macrocode}
%
%   Define user interfaces to setting link colors.
%    \begin{macrocode}
\DeclareDocumentCommand \MakeLinksTheseColors { m m m } {
    \tl_set:Nn \l_UWMad_LayoutStyle_Color_Link_tl {#1}
    \tl_set:Nn \l_UWMad_LayoutStyle_Color_Cite_tl {#2}
    \tl_set:Nn \l_UWMad_LayoutStyle_Color_URL_tl  {#3}
    \UWMad_LayoutStyle_ResetLinkColors:
}
\DeclareDocumentCommand \MakeLinksThisColor { m } {
    \tl_set:Nn \l_UWMad_LayoutStyle_Color_Link_tl {#1}
    \tl_set:Nn \l_UWMad_LayoutStyle_Color_Cite_tl {#1}
    \tl_set:Nn \l_UWMad_LayoutStyle_Color_URL_tl  {#1}
    \UWMad_LayoutStyle_ResetLinkColors:
}
%    \end{macrocode}
%
%   Define user interfaces to specialized color commands.
%    \begin{macrocode}
\DeclareDocumentCommand \MakeLinksBlack { } {
    \MakeLinksThisColor{black}
}
\DeclareDocumentCommand \MakeLinksBlue { } {
    \MakeLinksThisColor{blue}
}
\DeclareDocumentCommand \MakeLinksRed { } {
    \MakeLinksThisColor{red}
}
%    \end{macrocode}
%
%
%
%   \UWSubModule{Page Layout}
%
%   One inche magrins and letter (paper size) are set.
%    \begin{macrocode}
\geometry{
    includehead = true,
    margin      = 1.0in,
    paper       = letterpaper,
}
%    \end{macrocode}
%
%   Invoke `doublespacing' and set a warning in case any others invoke
%   the `not cool' commands according to the \UWMadShort{} Guidelines.
%    \begin{macrocode}
\doublespacing
\UWMad_Hook_Prepend:Nn \singlespacing {
    \__UWMad_FrontMatter_StyleWarning:n {
        University~guidelines~require~double-spacing.~
        If~this~is~for~temporary~use,~please~use~the~spacing~environment.
    }
}
\UWMad_Hook_Prepend:Nn \onehalfspacing {
    \__UWMad_FrontMatter_StyleWarning:n {
        University~guidelines~require~double-spacing.~
        If~this~is~for~temporary~use,~please~use~the~spacing~environment.
    }
}
%    \end{macrocode}
%
%   This setting puts the page numbers in the upper right-hand corner and
%   atleast one inch from the top and right sides of the page (per
%   the \UWMadShort{} guidelines).
%    \begin{macrocode}
\pagestyle{myheadings}
\setlength{\headsep}  {1.15em}
%    \end{macrocode}
%
%   Define user interface for defining different indentation styles.
%    \begin{macrocode}
\DeclareDocumentCommand \IndentParagraphs { } {
    \setlength{\parindent}{1.50em}
    \setlength{\parskip}  {0pt}

}
\DeclareDocumentCommand \PadParagraphs { } {
    \setlength{\parindent}{0pt}
    \setlength{\parskip}  {1em}
}
%    \end{macrocode}
%
%   Declare the default style of the \UWMadClass{} from the author:
%   no indentation with extra space between paragraphs.
%    \begin{macrocode}
\PadParagraphs
%    \end{macrocode}
%
%
%   \iffalse
%</Code>
%   \fi
%Verbatim
%</Implementation>






