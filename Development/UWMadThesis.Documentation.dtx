%   \iffalse
%<*PROTECT>
%   \fi
%   \iffalse
\documentclass{UWMadThesisL3Doc}
%
    \DeclareDocumentCommand \LaTeXPL { } {%
        \texttt{expl3}%
    }
%
%
    \setcounter{page}{-1000}
%
%
    \Title{UWMadThesis Class Manual}
    \Author{Troy Christopher Haskin}
%s
%
    \EnableCrossrefs
    \RecordChanges
    \CodelineIndex

    \begin{document}

% ============================================= %
%                   Title Page                  %
% ============================================= %
            \newgeometry{
                marginparwidth = 0in,
                left           = 0.25in,
                right          = 0.25in,
                bottom         = 1.00in,
                top            = 1.00in,
            }
            \thispagestyle{empty}
            \null\vfill
            \begin{center}
                {
                    \huge
                    \texttt{UWMadThesis} Class Manual
                }
                \vskip0.25in
                {
                    \Large
                    \TheAuthor{}
                }
                \vskip2in
                \vfill
            \end{center}
        \clearpage
        \restoregeometry

% ============================================= %
%                      TOC                      %
% ============================================= %
        \pagenumbering{roman}
        \TableOfContents
        \DocInput{\jobname.dtx}
    \end{document}
%
%   \fi
%
%
% \CheckSum{0}
%
%   \CharacterTable
%   {Upper-case \A\B\C\D\E\F\G\H\I\J\K\L\M\N\O\P\Q\R\S\T\U\V\W\X\Y\Z
%    Lower-case \a\b\c\d\e\f\g\h\i\j\k\l\m\n\o\p\q\r\s\t\u\v\w\x\y\z
%    Digits     \0\1\2\3\4\5\6\7\8\9
%    Exclamation   \!    Double quote \"     Hash (number) \#
%    Dollar        \$    Percent      \%     Ampersand     \&
%    Acute accent  \'    Left paren   \(     Right paren   \)
%    Asterisk      \*    Plus         \+     Comma         \,
%    Minus         \-    Point        \.     Solidus       \/
%    Colon         \:    Semicolon    \;     Less than     \<
%    Equals        \=    Greater than \>     Question mark \?
%    Commercial at \@    Left bracket \[     Backslash     \\
%    Right bracket \]    Circumflex   \^     Underscore    \_
%    Grave accent  \`    Left brace   \{     Vertical bar  \|
%    Right brace   \}    Tilde \~}
%
%   \input{UWMadThesis.Documentation.IndexExclusions.dtx}
%   \changes{1.0}{Hello}{Hello}
%   \pagenumbering{arabic}
\UWPart{User Guide}

The \UWMadClass{} is aimed at providing a \LaTeXe{} class that conforms to the style and format guidelines of the Graduate School of the \UWMadLong{}.
A copy of the current style guidelines and other associated PDFs are available

In addition to that primary goal, the class also loads a number of useful packages and defines or expands on a number of commands and utilities for creating a high-quality document.



\UWFeature{Thesis and PDF Information}

In order for the \RefSubFeature{Title Page} to function properly, a certain amount of information about the thesis must be given.
The \UWMadClass{} has a set of commands to provide both the thesis information and PDF metadata to \LaTeX{}.

It is highly encouraged to use all of these commands in the preamble such that any PDF metadata can be directly set before the document begins.
If the commands are used within the |document| environment, it will require another \LaTeX{} compilation to include the metadata since \UWMadClass{} will automatically write the information to an external file.

\UWSubFeature{Required}
    These commands are required.
    If any of these commands is not present, usage of the \RefSubFeature[title page]{Title Page} command will throw an error.
    It is encouraged to use these commands in the preamble of the document.

    \begin{function} {
        \Title,
        \Author,
        \Program,
    }
        \begin{syntax}
            \cs{Title}   \marg{title}
            \cs{Author}  \marg{author name}
            \cs{Program} \marg{program}
        \end{syntax}
        Each of these commands must be used once; if not, their respective variables will be empty and usage of the \
        They can, of course, be used more than once, but the additional uses would only redefine the value of the associated variable.
    \end{function}

    \begin{function}{
        \Degree,
        \Doctorate,
        \Masters,
        \Bachelors
    }
        \begin{syntax}
            \cs{Degree}  \marg{degree}
            \cs{Doctorate}
            \cs{Masters}
            \cs{Bachelors}
        \end{syntax}
        Only one of these commands is required to define the \marg{degree} variable.
        The generic \cs{Degree} function will accept any valid text or expandable content for defining the degree variable.

        The other three commands take no argument and are semantic commands for defining the degree variable:
            \begin{itemize}
            \item{\cs{Doctorate} sets \marg{degree} to \enquote{Doctor of Philosophy}}
            \item{\cs{Masters} sets \marg{degree} to \enquote{Master's}}
            \item{\cs{Bachelors} sets \marg{degree} to \enquote{Bachelor's}}
            \end{itemize}
    \end{function}

    \begin{function} {
        \DefenseDate,
        \DefenceDate
    }
        \begin{syntax}
            \cs{DefenseDate} \marg{defense date}
            \cs{DefenceDate} \marg{defense date}
        \end{syntax}
        Only one of these commands is needed since they all point to the same variable \marg{defense date}.
        The aliases were created for personal preference only.

        Since \marg{defense date} has no parsing performed on it, any valid text or expandable argument may be entered and will be typeset as-entered.
    \end{function}

    \begin{function} {
        \Institution,
        \University
    }
        \begin{syntax}
            \cs{Institution} \marg{institution name}
            \cs{University}  \marg{institution name}
        \end{syntax}
        Only one of these commands is needed since they both point to the same variable \marg{institution name}.
        The aliases were created for personal preference only.
    \end{function}

    \begin{function} {
        \CommitteeMember,
        \Advisor,
        \Adviser
    }
        \begin{syntax}
            \cs{CommitteeMember} \marg{member name}  \marg{member position}  \marg{member program}
            \cs{Advisor}         \marg{advisor name} \marg{advisor position} \marg{advisor program}
            \cs{Adviser}         \marg{advisor name} \marg{advisor position} \marg{advisor program}
        \end{syntax}
        \cs{CommitteeMember} can be used as many times as required.
        However, if the list of members becomes too large, formatting of the \RefSubFeature[title page]{Title Page} will suffer.

        Using either the \cs{Advisor} or \cs{Adviser} commands automatically adds the advisor/adviser to the top of the committee list created by \cs{CommitteeMember}.
        Also, on the title page's committee list, the advisor/adviser is marked as such by ``(Advisor)'' or ``(Adviser)''.
        This is a rare exception where the choice of alias has a side-effect.
        Either of these commands are not required but semantic in nature.
    \end{function}

\UWSubFeature{Optional}
These commands are not required for the document to be typeset properly.
However, they do provide metadata for the PDF (e.g., keywords and document subject) that is convenient for searching and categorization.
It is encouraged to use these commands in the preamble of the document.

\begin{function} {
    \DocumentType,
    \Dissertation,
    \DoctoralThesis,
    \MastersThesis,
    \Thesis,
    \Prelim
    }
    \begin{syntax}
        \cs{DocumentType} \marg{document type}
        \cs{Dissertation}
        \cs{DoctoralThesis}
        \cs{MastersThesis}
        \cs{Thesis}
        \cs{Prelim}
    \end{syntax}
    By default, the \cs{MakeTitlePage} command prints the phrase \enquote{A \marg{document type} submitted in partial fulfillment of the requirements for the degree of'' on the title page}.
    The default \marg{document type} is \enquote{report}.
    This command sets the value to any valid text.

    To facilitate good semantic mark-up, some prepared commands to set the document type were made.
    These commands take no argument and set the value of \marg{document type} to something similar to their command name:
    \begin{itemize}
        \item{\cs{Dissertation} sets \marg{document type} to \enquote{dissertation}}
        \item{\cs{DoctoralThesis} sets \marg{document type} to \enquote{doctoral thesis}}
        \item{\cs{MastersThesis} sets \marg{document type} to \enquote{master's thesis}}
        \item{\cs{Thesis} sets \marg{document type} to \enquote{thesis}}
        \item{\cs{Prelim} sets \marg{document type} to \enquote{preliminary report}}
    \end{itemize}
\end{function}

\begin{function} {
    \Subject,
    \Keywords
    }
    \begin{syntax}
        \cs{Subject}  \marg{document subject}
        \cs{Keywords} \marg{list of keywords}
    \end{syntax}
    These commands set the subject and keyword portions of the PDF metadata.
    The \marg{document subject} is typically a one-ish line description of the document.
    The \marg{list of keywords} can be a long, punctuation-delimited list (e.g., comma or semicolon) of keywords.
\end{function}

\begin{function} {
    \Producer,
    \Creator
    }
    \begin{syntax}
        \cs{Producer} \marg{pdf producer}
        \cs{Creator}  \marg{pdf creator}
    \end{syntax}
    These commands set the PDF Producer and PDF Creator fields of the metadata.
    These fields are a little confusing in their intended usage.
    The best explanation found is
    \begin{description}
        \item[Creator]  {The application used to create the original document which became the PDF.}
        \item[Producer] {The application used to convert the original document into the PDF.}
    \end{description}
    These are very thin distinctions and complicated by the typical workflow of a \LaTeX{} document: installing a \TeX{} distribution, editing a text file in \TeX{}/\LaTeX{} editor, and running the document through a \TeX{} engine with the \LaTeX{} format.
    In order to give credit at all levels (while maintaining proper separation of the processes involved), it is recommended to state the editor and \TeX{} format used as the creator and state the engine and distribution used as the producer.
    For example, this document would declare the following:
    \begin{verbatim}
        \Creator{TeXnicCenter 2.02, LaTeX2e+}
        \Producer{pdfTeX 1.40.14, MiKTeX 2.9}
    \end{verbatim}
    But as stated before, this is all optional.
\end{function}

\UWSubFeature{Accessors}

\begin{function} {
    \TheTitle,
    \TheAuthor,
    \TheProgram,
    \TheDegree,
    \TheDefenseDate,
    \TheDefenceDate,
    \TheInstitution,
    \TheDocumentType,
    \TheAdvisor,
    \TheSubject,
    \TheKeywords,
    \TheProducer,
    \TheCreator
    }
If, for any reason, the thesis information or metadata registered with the document is required, these accessor commands exist to retrieve the stored value.
\end{function}


\UWFeature{Special Pages}

\UWSubFeature{Title Page}
This is a replacement for the default \cs{maketitle}.
Per the example provided by the \UWMadShort{} Graduate School's sample, the title page flows (in order): report title, author by-line, partial fulfillment clause, degree, program, university identification, oral defense date, and oral committee list.
The styles can be re-worked by redefining the commands as presented in the \RefSubModule{MakeTitlePage} implementation.
The formatting of the commands is standard \LaTeXe{} to facilitate customization.

\textsc{Note:} The \cs{MakeTitlePage} command needs the required thesis information from the commands described in the \RefSubFeature[Required subsection]{Required}.

\UWSubFeature{License Page}

There are two main licenses \UWMadClass{} supports: Copyright and Creative Commons.
If an author wishes to use these supported licenses to create a license page, all of the commands listed must be placed within a |LicensePage| environment, or the commands will not work (by design).

To declare a simple Copyright input
\begin{verbatim}
    \begin{LicensePage}
        \Copyright
    \end{LicensePage}
\end{verbatim}
To declare a simple Creative Commons input
\begin{verbatim}
    \begin{LicensePage}
        \CreativeCommons
    \end{LicensePage}
\end{verbatim}
There are more features for the Creative Commons license and are discussed below.

The above examples will automatically create a page using default values for license owner (the \RefSubFeature[thesis author]{Required}), year (the current year), and license specifics (outlined below).
If either is incorrect for the current usage, use the following commands:
\begin{function} {
        \LicenseOwner,
        \LicenseYear
}
    \begin{syntax}
        \vspace*{3pt}
        \setstretch{1.30}
        \cs{LicenseOwner} \marg{owner name}
        \cs{LicenseYear}  \marg{year}
    \end{syntax}
    These commands override the default values with the supplied, mandatory argument.
\end{function}

\UWSubSubFeature{Copyright}
The Copyright Act of 1976 (\href{http://www.copyright.gov/title17}{Title 17 of the United States Code}, section 106) lists the following six exclusive rights the owner of copyright and any other sanctioned parties have:
\begin{enumerate}
    \item{to reproduce the copyrighted work in copies or phonorecords}
    \item{to prepare derivative works based upon the copyrighted work}
    \item{  to distribute copies or phonorecords of the copyrighted work to the public by sale or other transfer of ownership, or by rental, lease, or lending}
    \item{in the case of literary, musical, dramatic, and choreographic works, pantomimes, and motion pictures and other audiovisual works, to perform the copyrighted work publicly}
    \item{in the case of literary, musical, dramatic, and choreographic works, pantomimes, and pictorial, graphic, or sculptural works, including the individual images of a motion picture or other audiovisual work, to display the copyrighted work publicly}
    \item{in the case of sound recordings, to perform the copyrighted work publicly by means of a digital audio transmission}
\end{enumerate}
There are a number of exceptions and limitations to these rights as outlined by subsequent sections (Title 17 of the United States Code, sections 107 -- 122), but these will not be discussed.
Under section 302 of the Copyright Act, the exclusive rights granted to a singular author of a work persist for 70 years following her death.

Section 401 of the Copyright Act requires a Form of Notice of copyright.
It consists of the elements: the copyright symbol \copyright{} (or the word ``Copyright''), the year of first publication (with more requirements for derivative works), and the name of the owner of the copyright (or some other designation).
All works containing this notice of copyright fall under the protection of the Copyright Law of the United States.

Section 408 of the Copyright Act states: for any work produced after 1978, ``the owner of copyright or of any exclusive right in the work may obtain registration of the copyright claim by delivering to the Copyright Office the deposit specified by this section, together with the application and fee''.
In others words, a copy of the work can be submitted to the Copyright Office and subsequently placed in the Library of Congress for official recognition of copyright.
However, registration is not compulsory since ``[s]uch registration is not a condition of copyright protection''.

\begin{function}{\Copyright}
    \begin{syntax}
        \cs{Copyright}
    \end{syntax}
    Using this command within a |LicensePage| environment will print a Copyright Notice at the bottom of a page and place a link in the table of contents.
\end{function}

An example of usage (along with a redefined owner and year) would be
\begin{verbatim}
    \begin{LicensePage}
        \LicenseOwner{Theodore Huxton}
        \LicenseYear{3001}
        \Copyright
    \end{LicensePage}
\end{verbatim}
This input would generate the following text at the bottom of a new page (with a link in the table of contents:

\ExplSyntaxOn
    \cs_gset_eq:NN \UWMadCopyrightPageText \__UWMad_Copyright_LicenseText:
\ExplSyntaxOff
\begin{center}
    \begin{LicensePage}
        \LicenseOwner{Theodore Huxton}
        \LicenseYear{3001}
        \UWMadCopyrightPageText
    \end{LicensePage}
\end{center}
\ExplSyntaxOn
    \cs_undefine:N \UWMadCopyrightPageText
\ExplSyntaxOff

\UWSubSubFeature{Creative Commons}
Creative Commons (CC) is a collective set of licenses that is most aptly described as ``some rights reserved''.
That is, while Copyright requires explicit permission of the author for a lot of uses, Creative Commons immediately waives those rights.
Why is this a good thing?
To quote from \href{http://creativecommons.org/about}{CreativeCommons.org}:
\begin{quote}
Creative Commons is a nonprofit organization that enables the sharing and use of creativity and knowledge through free legal tools. ...

If you want to give people the right to share, use, and even build upon a work you've created, you should consider publishing it under a Creative Commons license.
CC gives you flexibility (for example, you can choose to allow only non-commercial uses) and protects the people who use your work, so they don't have to worry about copyright infringement, as long as they abide by the conditions you have specified.
\end{quote}

Therefore, the goal of CC is to begin from the ``most free'' license of public domain (termed CC0) and then add on conditions for legal use of the material.
CC licenses are copyright licenses in that (aside from CC0) the author retains certain ownership rights, but a subset of the rights are relaxed or waived to encourage free sharing and extension of the work.
To this end, Creative Commons defines the following four conditions:
\begin{description}
    \item[Attribution]{
        Appropriate credit must be given to the original author, a link to the license provided, and indication of any changes that were made.
        This may be done in any reasonable manner, but not in any way that suggests the licensor endorses the new author or her use.
    }
    \item[ShareAlike] {
        If the work is remixed, transformed, or built upon the licensed material, the author of the new work \textsc{must distribute} the contributions under the same license as the original.
    }
    \item[NoDerivs] {
        If the work is remixed, transformed, or built upon the licensed material, the author of the new work \textsc{may not} distribute the modified material.
    }
    \item[NonCommercial] {
        The licensed work \textsc{may not} be used the material for commercial purposes.
    }
\end{description}
These conditions are then combined into six, non-contradictory licenses.
The licenses are ``layered'' into Legal Code (the official text determining the delineating usage), the License deed (non-legal text aimed to be non-lawyer readable), and machine readable code (the license put into an HTML-like style for search engines).
The CC licenses (and associated links) for the latest version are
\begin{description}
    \item[CC BY]{
        \hfill\\
        Attribution only (
        \href{http://creativecommons.org/licenses/by/4.0}{License Deed} $\vert$
        \href{http://creativecommons.org/licenses/by/4.0/legalcode}{Legal Code}
        ).
    }
    \item[CC BY-SA]{
        \hfill\\
        Attribution and ShareAlike (
        \href{http://creativecommons.org/licenses/by-sa/4.0}{License Deed} $\vert$
        \href{http://creativecommons.org/licenses/by-sa/4.0/legalcode}{Legal Code}
        ).
    }
    \item[CC BY-ND]{
        \hfill\\
        Attribution and NoDerivs (
        \href{http://creativecommons.org/licenses/by-nd/4.0}{License Deed} $\vert$
        \href{http://creativecommons.org/licenses/by-nd/4.0/legalcode}{Legal Code}
        ).
    }
    \item[CC BY-NC]{
        \hfill\\
        Attribution and NonCommerical (
        \href{http://creativecommons.org/licenses/by-nc/4.0}{License Deed} $\vert$
        \href{http://creativecommons.org/licenses/by-nc/4.0/legalcode}{Legal Code}
        ).
    }
    \item[CC BY-NC-SA]{
        \hfill\\
        Attribution, NonCommercial, and ShareAlike (
        \href{http://creativecommons.org/licenses/by-nc-sa/4.0}{License Deed} $\vert$
        \href{http://creativecommons.org/licenses/by/4.0/legalcode}{Legal Code}
        ).
    }
    \item[CC BY-NC-ND]{
        \hfill\\
        Attribution, NonCommercial, and NoDerivs (
        \href{http://creativecommons.org/licenses/by-nc-nd/4.0}{License Deed} $\vert$
        \href{http://creativecommons.org/licenses/by-nc-nd/4.0/legalcode}{Legal Code}
        ).
    }
\end{description}

Prior to version 4.0 (the current one), there were a number of ``ports'' of the licenses to particular locales to deal with the specifics of individual countries.
However, with the release of version 4.0 of the CC licenses, usage of the international version is highly encouraged as ports will be made ``\href{http://wiki.creativecommons.org/License_Versions#International_License_Development_Process}{only where a compelling need is demonstrated}''.
As such, version 4.0 International is the default license base for the \UWMadClass{}.
Of course, this choice can be circumvented.

\begin{function}{\CreativeCommons}
    \begin{syntax}
        \cs{CreativeCommons}
    \end{syntax}
    Using this command within a |LicensePage| environment will declare you have chosen a Creative Commons license.
    By default, the license will be ``Creative Commons Attribution 4.0 International''.
\end{function}

\begin{function}{
    \Attribution,
    \ShareAlike,
    \NonCommercial,
    \NoDerivs}
    \begin{syntax}
        \cs{Attribution}
        \cs{ShareAlike}
        \cs{NonCommercial}
        \cs{NoDerivs}
    \end{syntax}
    Using any of these commands (in any order) within a |LicensePage| environment will declare you have chosen to add the associated condition to the license of the work.
    However, since all six licensees require Attribution, it is always on by default but should be included for clarity.
\end{function}
An example of usage would be
\begin{verbatim}
    \begin{LicensePage}
        \CreativeCommons
        \Attribution
        \NonCommercial
        \ShareAlike
    \end{LicensePage}
\end{verbatim}
This input would generate the following text at the bottom of a new page (with a link in the table of contents):
\begin{center}
    \begin{LicensePage}
        \NonCommercial
        \ShareAlike
        \ExplSyntaxOn
            \__UWMad_CCLicense_CreateType:
            \__UWMad_CCLicense_LicenseText:
        \ExplSyntaxOff
    \end{LicensePage}
\end{center}
Notice that since neither the \cs{LicenseOwner} nor \cs{LicenseYear} commands were used, the author of this document and current year were used as defaults.

\begin{function}{
    \CCVersion,
    \CCPorting,
    \CCURL,
    \CCURLText}
    \begin{syntax}
        \cs{CCVersion}\marg{CC version}
        \cs{CCPorting}\marg{CC porting}
        \cs{CCURL}    \marg{CC link}
        \cs{CCURLText}\marg{CC link text}
    \end{syntax}
    These commands exist to override the default 4.0 International Creative Commons license.
    The link provided \textsc{should not} contain |http://| nor end with a |/|.
    Use these commands only if there is a compelling reason not to use the latest version of the license.
\end{function}
An example of usage would be
\begin{verbatim}
    \begin{LicensePage}
        \CreativeCommons
        \CCVersion{3.0}
        \CCPorting{United States}
        \CCURL{creativecommons.org/licenses/by/3.0/us}
        \CCURLText{Creative Commons Attribution 3.0 United States}
    \end{LicensePage}
\end{verbatim}
This input would generate the following text at the bottom of a new page (with a link in the table of contents):
\begin{center}
    \begin{LicensePage}
        \CCVersion{3.0}
        \CCPorting{United States}
        \CCURL{creativecommons.org/licenses/by/3.0/us}
        \CCURLText{Creative Commons Attribution 3.0 United States}
        \ExplSyntaxOn
            \__UWMad_CCLicense_LicenseText:
        \ExplSyntaxOff
    \end{LicensePage}
\end{center}


\UWFeature{Layout And Style}

The \UWMadClass{} has several default styling differences from the standard \LaTeXe{} class it is based on.
Some of these changes exist to abide by the \UWMadShort{} dissertation guidelines and others are based on the author's preferences.
They are, however, readily changeable using the facilities of the packages used to make the changes.
The defaults and methods for changing the styles are list in this section or the references manuals.

\UWSubFeature{Captions}
The \UWMadClass{} uses the \pkg{caption} and \pkg{subcaption} packages to style float captions and subcaptions.
It is possible to adjust the defaults showcased below by using the packages' utilities outlined in their respective manuals.

\vskip1em
{
    \captionof {figure} {Here is an example of a figure caption.
        The default style for the \UWMadClass{} is a slanted font (abbrev. \enquote{sl}) and small capitals (abbrev. \enquote{sc}) for the float label.
        Notice that long captions, like this, are indented such that the caption text is visibly separated from the float label.
    }

    \captionof {table} {Here is a shorter example of a table caption.
        The default styling is identical to the figure caption.
    }
}

\UWSubFeature{Links}
The \UWMadClass{} loads the \pkg{hyperref} and \pkg{bookmark} packages to create hyperlinks and a clickable documents.
The default color for document links is \textcolor{blue}{blue}, for urls is \textcolor{violet}{violet}, and for citations is \textcolor{UWMadGreen}{UWMadGreen} (a darker version of \textcolor{green}{green}).
These defaults can be changed using the commands below or the facilities of the \pkg{hyperref} package as described in its manual.
New colors can be created using the facilities of the \pkg{xcolor} package as described in its manual.

\UWSubSubFeature{Link Colors}
To more easily facilitate color changes to links, several user interface commands have been defined.

\begin{function} {\MakeLinksTheseColors}
    \begin{syntax}
        \cs{MakeLinksTheseColors}\Arg{link color}\Arg{cite color}\Arg{url color}
    \end{syntax}
    Redefines the colors used for (internal) links, cites, and URLs.
    Any valid color, including those defined by the \pkg{xcolor} package, is allowed for all three, required arguments.
\end{function}
\begin{function} {\MakeLinksThisColor}
    \begin{syntax}
        \cs{MakeLinksThisColor}\Arg{color}
    \end{syntax}
    Redefines the colors used for (internal) links, cites, and URLs to be the single indicated color.
    Any valid color, including those defined by the \pkg{xcolor} package, is allowed for the one required arguments.
\end{function}

\begin{function} {\MakeLinksBlack,\MakeLinksBlue,\MakeLinksRed}
    \begin{syntax}
        \cs{MakeLinksBlack}
        \cs{MakeLinksBlue}
        \cs{MakeLinksRed}
    \end{syntax}
    These commands take no argument and define all links to have the color indicated in the command name.
\end{function}

\UWSubSubFeature{References}
References may be handled by the \pkg{hyperref} package using \cs{autocite} or by the \pkg{cleveref} package using \cs{cref}/\cs{Cref} (the latter producing a capital letter for the reference type).
The user is referred to their respective manuals for more options and feature descriptions.

\UWSubFeature{Paragraph Spacing}
In general, there are two dominant methods for indicating separate paragraphs: no indentation with extra space between paragraphs (compared to between lines) and indentation with no extra space between paragraphs.
The default of the \UWMadClass{} is the former but some may prefer the latter.
To facilitate either, two commands have been created.

\begin{function} {\PadParagraphs}
    \begin{syntax}
        \cs{PadParagraphs}
    \end{syntax}
    This command adds 1em of vertical space between paragraphs with no indentation.
    This is the default style of this class.
\end{function}

\begin{function} {\IndentParagraphs}
    \begin{syntax}
        \cs{IndentParagraphs}
    \end{syntax}
    This command adds 1.5em of indent at the beginning of paragraphs (save those that follow section heads) with no extra vertical space.
\end{function}



\UWFeature{Sectioning}

Sectioning concerns the overall structure of your document into chunks called sections.
The default sections in \LaTeXe{} are |part|, |chapter|, |section|, |subsection|, |subsubsection|, |paragraph|, and |subparagraph|.
The \UWMadClass{} defines some new section commands and makes some other adjustments to the default commands.

\UWSubFeature{Front Matter}
Front Matter (or preliminary pages) is the whole-of-content that precedes the main document (i.e., the first unstarred chapter).
\UWMadShort{} requires that these pages are numbered in lower roman numerals and have that page number in the upper right-hand corner.
This requirement is automatically handled by the class.
The Front Matter commands are all semantically named and set as starred (unnumbered) chapters.

\begin{function} {
    \dedications,
    \acknowledgments,
    \abstract,
    \umiabstract,
    \preface
    }
    \begin{syntax}
        \cs{dedications}      \marg{title}
        \cs{acknowledgments} \marg{title}
        \cs{abstract}         \marg{title}
        \cs{umiabstract}      \marg{title}
        \cs{preface}          \marg{title}
    \end{syntax}
    The title \textsc{is optional}.
    If the title is omitted, the default is a capitalized version of the command's name.
    For example, |\dedications| will have the title \enquote{Dedications}.
\end{function}

\UWSubFeature{Appendix}
The standard method of including appendices in \LaTeX{} is calling for some initialization to be done by using the \cs{appendix} command and then using the \cs{chapter} command.
The \UWMadClass{} takes a different approach to encourage good semantic mark-up in \LaTeX{} documents and, therefore, redefines \cs{appendix}.

\begin{function} {
    \appendix
    }
    \begin{syntax}
        \cs{appendix} \oarg{short title}\marg{title}
        \cs{appendix}*\oarg{short title}\marg{title}
    \end{syntax}
    The appendix commands now acts like \cs{chapter} commands and are typeset in the Table of Contents as such.

    \textsc{Note}:
    The usage \cs{appendix} should be after all the chapter material is set since some of the \cs{chapter} internals are changed.
    Once the \cs{appendix} command is used, there is no mechanism to switch the internals back.
\end{function}

\UWSubFeature{Table of Contents Tweaks}
Invoking the Table of Contents, List of Tables, and List of Figures commands now puts the start of those sections into the Table of Contents as chapters.

\begin{function} {
    \TableOfContentsName,
    \ListOfTablesName,
    \ListOfFiguresName
    }
    \begin{syntax}
        \cs{TableOfContentsName}\marg{toc title}
        \cs{ListOfTablesName}   \marg{lot title}
        \cs{ListOfFiguresName}  \marg{lof title}
    \end{syntax}
    These commands redefine the title used in the associated sections.
    The defaults for the TOC, LOT, and LOF are, respectively, \enquote{Table of Contents}, \enquote{List of Tables}, and \enquote{List of Figures}.
\end{function}

\begin{function} {
    \TableOfContents,
    \ListOfTables,
    \ListOfFigures
    }
    \begin{syntax}
        \cs{TableOfContents}
        \cs{ListOfTables}
        \cs{ListOfFigures}
    \end{syntax}
    Camel-cased versions of the standard \LaTeX{} commands.
    These exist due to the preferences of the \UWMadClass{} author.
\end{function}


\UWFeature{List Environments}

The \UWMadClass{} has a special set of functions from creating list environments (called |ListOf| in the implementation).
The functions use queues and associative arrays to store and use data before it is typeset.
These data structures allow for operations to be carried out without writing external files or repeating compilation; of course, there is added memory usage which could lead to problems on older systems.

The primary motivation for such a system was the creation of a nomenclature environment and, subsequently, an acronym environment/system.
These two similar features are discussed here.

\UWSubFeature{Nomenclature}
The |Nomenclature| environment is, by default, a list of |(symbol, description)| entries.
There is a user option for changing the system to a list of |(symbol, units, description)| entries if a separate unit column is desired.
For every set of entries, the nomenclature system measures the width of the |symbol| and (if present) |units| to determine the maximum width of the |description| such that no text overflows into the margins of the page.

When first adding entries to a nomenclature, the symbols are part of the so-called Main group.
The Main group has a title and a section level associated with it.
By default, the Main group title is ``Nomenclature'' and the section is ``chapter''.
The entries can be put into two lower sectioned groups using the \cs{Group} and \cs{Subgroup} commands described below.
The grouping commands allows a set of symbols to be classified as ``Greek Symbols'' while another is ``Subscripts''.
The default titles for these lower groups are empty by default and the default section is ``section'' and ``subsection''.

All of these defaults can be changed by the \cs{NomenclatureSetup} command described below.

\UWSubSubFeature{Command Descriptions}

A sketch of the |Nomenclature| implementation would be: \vskip1em
    \hspace{1em}|\begin{Nomenclature}|\oarg{section}\marg{title}    \vskip0em
    \hspace{3em}|\Entry|\marg{symbol}\marg{description}             \vskip0em
    \hspace{3em}|\Group|\marg{group title}                          \vskip0em
    \hspace{6em}|\Entry|\marg{symbol}\marg{description}             \vskip0em
    \hspace{6em}|\Subgroup|\marg{subgroup title}                    \vskip0em
    \hspace{9em}|\Entry|\marg{symbol}\marg{description}             \vskip0em
    \hspace{1em}|\end{Nomenclature}|                                \vskip1em

The square brace-delimited \oarg{section} is \textsc{optional} and overrides the default Main group section.
The curly brace-delimited  \marg{title} is \textsc{optional} and overrides the default Main group title.

\begin{function}{\Entry}
    \begin{syntax}
        \cs{Entry}\marg{symbol}\marg{description}
        \cs{Entry}\marg{symbol}\marg{units}\marg{description}
    \end{syntax}
    Within the environment, entries are added to the nomenclature using the \cs{Entry} command above.
    All arguments are required.
    The second version above is if a units column is requested (see \RefSubSubFeature{Customization}).
\end{function}

\begin{function}{\Group,\Subgroup}
    \begin{syntax}
        \cs{Group}\marg{group title}
        \cs{Subgroup}\marg{subgroup title}
    \end{syntax}
    Creates a group or subgroup with the indicated title and using the default section.
    The default section can be changed by the user (see \RefSubSubFeature{Customization}).
\end{function}

\UWSubSubFeature{Examples}
As an example, the following input
\begin{verbatim}
    \begin{Nomenclature}[subsubsection]{Symbol Table}
        \Entry{$\rho$}{Density}
        \Entry{LongNotRealSymbol}{
            In publishing and graphic design, lorem ipsum is a placeholder
            text commonly used to demonstrate the graphic elements of a
            document or visual presentation. By replacing the distraction
            of meaningful content with filler text of scrambled Latin it
            allows viewers to focus on graphical elements such as font,
            typography, and layout.}
        \Entry{$\mu$}{Viscosity}
    \end{Nomenclature}
\end{verbatim}
would be typeset as:

\setcounter{section}{1}
\NomenclatureSetup{include-in-toc = false}

\rule{\textwidth}{0.1em}
    \begin{Nomenclature}[subsection]{Symbol Table}
        \Entry{$\rho$}{Density}
        \Entry{LongNotRealSymbol}{
            In publishing and graphic design, lorem ipsum is a placeholder
            text commonly used to demonstrate the graphic elements of a
            document or visual presentation. By replacing the distraction
            of meaningful content with filler text of scrambled Latin it
            allows viewers to focus on graphical elements such as font,
            typography, and layout.}
        \Entry{$\mu$}{Viscosity}
    \end{Nomenclature}
\rule{\textwidth}{0.1em}
As can be seen, the symbol column is as wide as the widest symbol (plus some padding) and lengthy text can be put into the description without penalty.
Of course, this example is purposefully extreme.
We can tweak the example a bit more by putting the second two items under a group:

\rule{\textwidth}{0.1em}
    \begin{Nomenclature}[subsection]{Symbol Table}
        \Entry{$\rho$}{Density}

        \Group{Group 1 Title}
        \Entry{LongNotRealSymbol}{
            In publishing and graphic design, lorem ipsum is a placeholder
            text commonly used to demonstrate the graphic elements of a
            document or visual presentation. By replacing the distraction
            of meaningful content with filler text of scrambled Latin it
            allows viewers to focus on graphical elements such as font,
            typography, and layout.}
        \Entry{$\mu$}{Viscosity}
    \end{Nomenclature}
\rule{\textwidth}{0.1em}
By default, the section level used by \cs{Group} is one below that of the main nomenclature section; therefore, since the nomenclature's section level is defined as |subsection|, the \cs{Group} is a |subsubsection|.
Not shown: using \cs{Subgroup} would typeset the title as a |paragraph| in this example.

\UWSubSubFeature{Customization}

As mentioned, there are several options available to the user for customizing the nomenclature.
These options are set by giving a comma-separate list of key-value pairs to the function \cs{NomenclatureSetup}

\begin{function}{\NomenclatureSetup}
    \begin{syntax}
        \cs{NomenclatureSetup}\marg{key-value CSV}
    \end{syntax}
    The format is more appropriately shown as
    \begin{verbatim}
        \NomenclatureSetup {
            key-one = option,
            key-two = {option two},
            ...
            key-n =  {option n},
        }
    \end{verbatim}
    A table of the keys, meaning, defaults, and allow value is given in \cref{Table:NomenclatureKeyValue}.
\end{function}

\clearpage
\UWSubFeature{Acronym}

\UWSubSubFeature{Description}
The |Acronym| environment is a specialized extension of the |Nomenclature| environment.
It has the same basic syntax, but a |units| column is not supported.
Also, instead of \cs{Entry} taking |(symbol, description)| pairs, it takes |(acronym,meaning)| pairs.
Lastly, it comes equipped with a new command: \cs{Acro}.

\begin{function}{\Acro}
    \begin{syntax}
        \cs{Acro}\marg{acronym}
    \end{syntax}
    \cs{Acro} is meant to be used throughout the document to reference back to the |Acronym| environment where it was defined.
    If an |Acronym| environment contains the line |\Entry{TBD}{To be determined}|, the first usage of |\Arco{TBD}| will be typeset as `To be determined (TBD)' while subsequent uses will simply be `TBD'.
    Also, if links are not turned off (they are on by default), the acronym will be a link back to the original environment entry.
\end{function}

\begin{function}{\AcronymSetup}
    \begin{syntax}
        \cs{AcronymSetup}\marg{key-value CSV}
    \end{syntax}
    An exact copy of \cs{NomenclatureSetup}.
\end{function}

\UWSubSubFeature{Example}
The following input
\begin{verbatim}
    \AcronymSetup {
        main-section  = section,
        main-title = {Acronym Table},
        entry-padding = 1in
    }
    \begin{Acronym}
        \Entry{RCCS}{Reactor Cavity Cooling System}
        \Entry{NRC}{Nuclear Regulatory Commission}
    \end{Acronym}
\end{verbatim}
is typeset as

\clearpage

\rule{\textwidth}{0.1em}
    \AcronymSetup {
        main-section  = section,
        main-title = {Acronym Table},
        entry-padding = 1in
    }
    \begin{Acronym}
        \Entry{RCCS}{Reactor Cavity Cooling System}
        \Entry{NRC}{Nuclear Regulatory Commission}
    \end{Acronym}
\rule{\textwidth}{0.1em}

The first usage of |\Acro{NRC}| is `\Acro{NRC}' while the second usage is `\Acro{NRC}'.

\begin{table}[H]
\begin{center}
    \caption{List of key-value pairs for Nomenclature customization.}
    \label{Table:NomenclatureKeyValue}
    \begin{tabular}{c c c c}
        \toprule
        Key & Meaning & Default & Allow value \\
        \midrule
        title-skip          & Vertical space following the printed title     & 0pt          & dimension \\[10pt]
        print-skip          & Vertical space following a printing of entries & 1em          & dimension \\[10pt]
        entry-margin-left   & Horizontal margin left of an entry             & 1em          & dimension \\[10pt]
        entry-margin-bottom & Vertical margin below a printed entry          & 0.25em       & dimension \\[10pt]
        entry-padding       & Horizontal space between columns               & 0.75em       & dimension \\[10pt]
        main-section        & Section level for Main group                   & chapter      &  section  \\[10pt]
        group-section       & Section level for \cs{Group} command           & section      &  section  \\[10pt]
        subgroup-section    & Section level for \cs{Subgroup} command        & subsection   &  section  \\[10pt]
        main-title          & Title for the nomenclature                     & Nomenclature &   text    \\[10pt]
        group-title         & Title for the \cs{Group} command               & ---          &   text    \\[10pt]
        subgroup-title      & Title for the \cs{Subgroup} command            & ---          &   text    \\[10pt]
        include-in-toc      & Include the nomenclature in the TOC            & true         & boolean   \\[10pt]
        with-units          & Include a units column                         & false        & boolean   \\
        \bottomrule
    \end{tabular}
\end{center}
\end{table}

\begin{table}[H]
\begin{center}
    \caption{Additional key-value pairs for Acronym environment.}
    \label{Table:AcronymKeyValue}
    \begin{tabular}{c c c c}
        \toprule
        Key & Meaning & Default & Allow value \\
        \midrule
        use-links           & Create hyperlink to Acronym entry  & true & boolean \\[10pt]
        link-color          & Color of hyperlink text            & blue & color   \\
        \bottomrule
    \end{tabular}
\end{center}
\end{table}

\UWFeature{Math}\label{UG:Math}
As the feature name may suggest, all of the commands in this section deal with mathematical typesetting.

\UWSubFeature{Derivative Commands}
These command set deal with quick and easy typesetting of derivatives.

\begin{function}{\deriv , \pderiv , \tderiv}
    \begin{syntax}
        \cs{deriv}  \marg{function} \marg{variable} \marg{order}    \\
        \cs{pderiv} \marg{function} \marg{variable} \marg{order}    \\
        \cs{tderiv} \marg{function} \marg{variable} \marg{order}
    \end{syntax}
    This function set is meant to typeset three different kinds of derivatives: ordinary, partial, and total (i.e., material or Lagragian).
    The only difference between them is the differential symbol: \cs{deriv} uses `$\derivSymbol$', \cs{pderiv} uses `$\pderivSymbol$', and \cs{tderiv} used `$\tderivSymbol$'.

    These commands typeset the derivative of a given \Arg{function} with respect to \Arg{variable} of $n$-th \Arg{order} using Leibniz's notation.
    The \Arg{order} is optional and defaults to empty (first derivative).
    For example, the input
    \begin{verbatim}
        \begin{align}
            \deriv{y}{x}{2} + \deriv{y}{x} + y(x)         &= 0    \\[0.50em]
            \pderiv{T}{t} - \alpha \pderiv{T}{z}{2}       &= 0    \\[0.50em]
            \tderiv{\rho{u}}{t} + \pderiv{P}{z}  - \rho g &= 0
        \end{align}
    \end{verbatim}
    and is typeset as
        \begin{align}
            \deriv{y}{x}{2} + \deriv{y}{x} + y(x)         &= 0    \\[0.50em]
            \pderiv{T}{t} - \alpha \pderiv{T}{z}{2}       &= 0    \\[0.50em]
            \tderiv{(\rho{u})}{t} + \pderiv{P}{z}  - \rho g &= 0
        \end{align}
\end{function}

\begin{function}{\derivbig,\pderivbig,\tderivbig}
    \begin{syntax}
        \cs{derivbig}   \oarg{left delim} \Arg{function} \oarg{right delim} \Arg{variable} \Arg{order} \\
        \cs{pderivbig}  \oarg{left delim} \Arg{function} \oarg{right delim} \Arg{variable} \Arg{order} \\
        \cs{tderivbig}  \oarg{left delim} \Arg{function} \oarg{right delim} \Arg{variable} \Arg{order}
    \end{syntax}
    This function set is identical to the non-|big| versions above, except that \Arg{function} is placed to the right of the derivative operator and wrapped by |\left| and |\right|.
    The default delimiters for the stretch commands are `[' and ']', and either can be individually overridden via the two optional arguments.
    For example, the input
    \begin{verbatim}
        \begin{align}
            -\derivbig{ p(x) \deriv{y}{x} }{x} +
                    q(x) (1 - \lambda) y(x)  &= 0 \\[0.50em]
            \tderivbig{ \rho{i} + \frac{1}{2} \rho u^2 }[(]{t} -
                    \pderivbig[\lvert]{ \kappa \pderiv{T}{z} }{z} &= 0
        \end{align}
    \end{verbatim}
    and is typeset as
        \begin{align}
            -\derivbig{ p(x) \deriv{y}{x} }{x} +
                    q(x) (1 - \lambda) y(x)  &= 0 \\[0.50em]
            \tderivbig{ \rho{i} + \frac{1}{2} \rho u^2 }[(]{t} -
                    \pderivbig[\lvert]{ \kappa \pderiv{T}{z} }{z} &= 0
        \end{align}
\end{function}

\begin{function}{\DerivativeGeneral,\DerivativeGeneralBig}
    \begin{syntax}
        \cs{DerivativeGeneral}     \Arg{function} \Arg{variable} \Arg{order} \Arg{symbol}
        \cs{DerivativeGeneralBig}  \Arg{function} \Arg{variable} \Arg{order} \Arg{symbol} \Arg{left delim} \Arg{right delim}
    \end{syntax}
    These commands are lower-level commands used by the |deriv| family above.
    All of the arguments are mandatory.
    If a change to the general style of the derivatives or another version of the |deriv| family is desire, these commands are available for usage.
\end{function}

\begin{function}{\derivSymbol,\pderivSymbol,\tderivSymbol}
    \begin{syntax}
        \cs{derivSymbol}
    \end{syntax}
    These commands take no arguments and expand to the current symbol used for the associated |deriv| command.
    The defaults require math mode to be typeset.
    Therefore, |$\pderivSymbol$| will be appear as $\pderivSymbol$.
\end{function}

\begin{function}{\derivSymbolChange,\pderivSymbolChange,\tderivSymbolChange}
    \begin{syntax}
        \cs{derivSymbolChange} \Arg{symbol} \\[0.50em]
    \end{syntax}
    These commands will \textsc{temporarily} change the symbol used by the associated |deriv| commands.
    The symbol will revert back to the original, default value after leaving the \TeX{} group where the switch was made (more often than not for \LaTeX{} users, this means ``upon exiting an environment'').
    For example:
    \begin{verbatim}
        \begin{equation}
            \deriv{U}{t} =
            \derivSymbolChange{\delta}
            \deriv{Q}{t} - \deriv{W}{t}
        \end{equation}
    \end{verbatim}
    typesets as
    \begin{equation}
            \deriv{U}{t} =
            \derivSymbolChange{\delta}
            \deriv{Q}{t} - \deriv{W}{t}
    \end{equation}
    and now, after the environment, the \cs{derivSymbol} is once again `$\derivSymbol$'.
\end{function}

\begin{function}{
    \derivSymbolChangeDefault,
    \pderivSymbolChangeDefault,
    \tderivSymbolChangeDefault}
    \begin{syntax}
        \cs{derivSymbolChangeDefault}  \Arg{symbol}
        \cs{pderivSymbolChangeDefault} \Arg{symbol}
        \cs{tderivSymbolChangeDefault} \Arg{symbol}
    \end{syntax}

    These commands will \textsc{permanently} change the symbol used by the associated |deriv| commands.
    For example:
    \begin{verbatim}
        \begin{equation}
            \deriv{U}{t} =
            \derivSymbolChangeDefault{\delta}
            \deriv{Q}{t} - \deriv{W}{t}
        \end{equation}
    \end{verbatim}
    typesets as
    \begin{equation}
            \deriv{U}{t} =
            \derivSymbolChangeDefault{\delta}
            \deriv{Q}{t} - \deriv{W}{t}
    \end{equation}
    and now, after the environment, the \cs{derivSymbol} is `$\derivSymbol$'.
\end{function}

\begin{function}{\DelimiterChangeDefault}
    \begin{syntax}
        \cs{DelimiterChangeDefault} \Arg{left delim} \Arg{right delim}
    \end{syntax}
    This command changes the default delimiters used by the |big| commands above.
    Any valid delimiters can be used.
    For example:
    \begin{verbatim}
        \DelimiterChangeDefault{(}{)}
        \begin{equation}
            -\derivbig{ p(x) \deriv{y}{x} }{x} +
                    q(x) (1 - \lambda) y(x) = 0 \\[0.50em]
        \end{equation}
    \end{verbatim}
    and is typeset as
    \DelimiterChangeDefault{(}{)}
    \begin{equation}
        -\derivbig{ p(x) \deriv{y}{x} }{x} +
                q(x) (1 - \lambda) y(x) = 0 \\[0.50em]
    \end{equation}
    and notice that the \cs{derivSymbol} is still $\derivSymbol$.
\end{function}

\UWSubFeature{Operators}
These operators are added to the standard set using the \AmS{} operator system.
Some are new while others are simply in a camel-cased versions of the standard ones.

\begin{function}{\Sup,\Inf}
    Supremum and Infinum operators using the math operator system.
    For example, the input
    \begin{verbatim}
        \begin{align}
            \Inf_{x \in \mathbb{R}} \{0 < x  < 1\} &= 0 \\[0.50em]
            \Sup_{x \in \mathbb{R}} \{0 < x  < 1\} &= 1
        \end{align}
    \end{verbatim}
    is typeset as
    \begin{align}
        \Inf_{x \in \mathbb{R}}
            \{0 \LessThan x \LessThan 1\} &= 0 \\[0.50em]
        \Sup_{x \in \mathbb{R}}
            \{0 \LessThan x \LessThan 1\} &= 1
    \end{align}
\end{function}

\begin{function}{\Lim}
    The limit operator:
    \begin{verbatim}
        \begin{equation}
            \Lim_{n \rightarrow \infty} \left(1 + \frac{1}{n}\right)^n = \mathrm{e}
        \end{equation}
    \end{verbatim}
    is typeset as
        \begin{equation}
            \Lim_{n \rightarrow \infty} \left(1 + \frac{1}{n}\right)^n = \mathrm{e}
        \end{equation}
\end{function}

\begin{function}{\Min,\Max}
    The maximum and minimum value operators
    \begin{verbatim}
        \begin{equation}
            \begin{align}
                \Min_{x \in \mathbb{R}} \Sin(x) &= -1 \\[0.50em]
                \Max_{x \in \mathbb{R}} \Sin(x) &= +1
            \end{align}
        \end{equation}
    \end{verbatim}
    is typeset as
    \begin{align}
        \Min_{x \in \mathbb{R}} \Sin(x) &= -1 \\[0.50em]
        \Max_{x \in \mathbb{R}} \Sin(x) &= +1
    \end{align}
\end{function}

\begin{function}{\ArgMin,\ArgMax}
    The maximum and minimum argument operators
    \begin{verbatim}
        \begin{equation}
            \begin{align}
                \ArgMin_{x \in \mathbb{R}} \Sin(x) &= \frac{3\pi}{2} + 2 \pi n \\[0.50em]
                \ArgMax_{x \in \mathbb{R}} \Sin(x) &= \frac{\pi}{2} + 2 \pi n
            \end{align}
        \end{equation}
    \end{verbatim}
    is typeset as
    \begin{align}
                \ArgMin_{x \in \mathbb{R}} \Sin(x) &= \frac{3\pi}{2} + 2 \pi n \\[0.50em]
                \ArgMax_{x \in \mathbb{R}} \Sin(x) &= \frac{\pi}{2} + 2 \pi n
    \end{align}
\end{function}

\begin{function}{\Abs,\Ln,\Log,\Exp}
    Common set of operators in uppercase form.
\end{function}

\begin{function}{\Cos,\Sin,\Tan,\Sec,\Csc,\Cot}
    Standard trigonometric functions and their reciprocals.
\end{function}
\begin{function}{\Cosh,\Sinh,\Tanh,\Sech,\Csch,\Coth}
    Hyperbolic trigonometric functions and their reciprocals.
\end{function}
\begin{function}{\ArcCos,\ArcSin,\ArcTan,\ArcSec,\ArcCsc,\ArcCot}
    Standard inverse trigonometric functions and their reciprocals.
\end{function}
\begin{function}{\ArcCosh,\ArcSinh,\ArcTanh,\ArcSech,\ArcCsch,\ArcCoth}
    Hyperbolic inverse trigonometric functions and their reciprocals.
\end{function}

\UWSubFeature{Miscellaneous Commands}

\begin{function}{\Sqrt}
    \begin{syntax}
        \cs{Sqrt} \oarg{n} \Arg{argument}
    \end{syntax}
    This command typesets the \oarg{n}-th root of a given \Arg{argument} with a closing tail.
    This command differs from the default \cs{sqrt} in appearance only:
    \begin{equation}
        \sqrt[3]{\frac{f(x)}{g(x)}} = \Sqrt[3]{\frac{f(x)}{g(x)}}
    \end{equation}
\end{function}

\begin{function}{\IfMathModeTF}
    \begin{syntax}
        \cs{IfMathModeTF} \Arg{math mode code} \Arg{text mode code}
    \end{syntax}
    This is an abstraction of |expl3|'s |\mode_if_math:TF| function.
    It was added to give more control on the following \cs{subs} and \cs{sups} commands since |expl3|'s syntax is disabled to make |_| a subscript shift and not a letter.
\end{function}

\begin{function}{\subs,\sups,\subsups}
    \begin{syntax}
        \cs{subs}    \oarg{space} \Arg{text subscript} \\[0.50em]
        \cs{sups}    \oarg{space} \Arg{text superscript} \\[0.50em]
        \cs{subsups} \oarg{subscript space} \Arg{text subscript} \oarg{superscript space} \Arg{text superscript}
    \end{syntax}
    These command typeset a subscript or superscript \textsc{in text mode}.
    They are useful if the subscript or superscript are not variable, and therefore should be in non-math text, or for making subscripts or superscripts in text mode.
    The optional argument \oarg{space} is meant for adjusting the spacing of the scripts and exists in \textsc{in math mode}, so technically, any valid math statement can be used.
    However, it is encouraged to only use this argument for spacing.
    For example, the input |`T\subs{P}, $T\subs{P}$, $T_P$'| is typeset as `T\subs{P}, $T\subs{P}$, $T_P$', and the input |`T\subs[\!]{P}, T\subs[\:]{P}'| is typeset as `T\subs[\!]{P}, T\subs[\:]{P}'.
    T\sups{P}
\end{function}

\begin{function}{\OneOver,\oneo}
    \begin{syntax}
        \cs{OneOver}  \Arg{denominator}
    \end{syntax}
    A simple command the typesets a fraction whose numerator is always one.
    For example, the input
    \begin{verbatim}
        \begin{equation}
            \OneOver{\Sqrt{x^2 + 1}}
        \end{equation}
    \end{verbatim}
    is typeset as
        \begin{equation}
            \OneOver{\Sqrt{x^2 + 1}}
        \end{equation}
\end{function}

\begin{function}{\dd}
    \begin{syntax}
        \cs{dd}  \Arg{variable}
    \end{syntax}
    A simple command the typesets a non-math `d' in math mode and is meant to be used for differentials.
    For example, the input
    \begin{verbatim}
        \derivSymbolChangeDefault{\mathrm{d}}
        \begin{equation}
            f(b) - f(a) = \int_a^b \deriv{f}{t} \dd{t}
        \end{equation}
    \end{verbatim}
    is typeset as
        \derivSymbolChangeDefault{\mathrm{d}}
        \begin{equation}
            f(b) - f(a) = \int_a^b \deriv{f}{t} \dd{t}
        \end{equation}
\end{function}

\begin{function}{\dprime,\tprime}
    \begin{syntax}
        \cs{dprime}
    \end{syntax}
    These commands take no arguments and simply mean `double prime' and `triple prime'.
    For example, the input
    \begin{verbatim}
        \begin{equation}
            q^\prime = q^\dprime 2\pi{R} = q^\tprime \pi{R^2}
        \end{equation}
    \end{verbatim}
    is typeset as
        \begin{equation}
            q^\prime = q^\dprime 2\pi{R} = q^\tprime \pi{R^2}
        \end{equation}
\end{function}

\UWFeature{Programming}
The \RefModule[Implementation section]{Programming} for this module outlines the programming layer used for the class.
There is a user-facing API but is not documented here as it is experimental.

\iffalse
\UWSubFeature{Utility Commands}

\begin{function}{\LoadUtilityAPI}
    \begin{syntax}
        \cs{LoadUtilityAPI}
    \end{syntax}
    In order to use the following commands, this command must be used first to load them.
\end{function}

\begin{function}{
    \IfCommandExistsTF,
    \IfCommandDoesNotExistTF}
    \begin{syntax}
        \cs{IfCommandExistsTF}      \marg{command name}\marg{true code}\marg{false code}
        \cs{IfCommandDoesNotExistTF}\marg{command name}\marg{true code}\marg{false code}
    \end{syntax}
    These commands test if a command with \marg{command name} exists or doesn't and branches to the appropriate code.
\end{function}

\begin{function}{\IfEmptyTF}
    \begin{syntax}
        \cs{IfEmptyTF} \marg{object}\marg{true code}\marg{false code}
    \end{syntax}
    This commands takes an argument, performs a full expansion of it, and branches to the appropriate code if the expansion is blank or not (just spaces or not).
\end{function}

\begin{function}{\IfCommandEmptyTF}
    \begin{syntax}
        \cs{IfCommandEmptyTF}\marg{command}\marg{true code}\marg{false code}
    \end{syntax}
    This function takes a command, expands it once, and branches to the appropriate code if the expansion is blank or not (just spaces or not).
\end{function}

\begin{function}{}
    \begin{syntax}
    \end{syntax}
\end{function}

\begin{function}{}
    \begin{syntax}
    \end{syntax}
\end{function}

\begin{function}{}
    \begin{syntax}
    \end{syntax}
\end{function}

\begin{function}{}
    \begin{syntax}
    \end{syntax}
\end{function}
\fi



%
%   \StopEventually{
%       \clearpage
%       \renewcommand{\glossaryname}{Change History}
%       \PrintChanges
%       \clearpage
%       \newgeometry{
%           marginparwidth = 0in,
%           left           = 0.25in,
%           right          = 0.25in,
%           bottom         = 1.00in,
%           top            = 1.00in,
%       }
%       \PrintIndex
%   }
%
%
%
%
%
%^^A ==================================================== !
%^^A                                                      !
%^^A                    BEGIN IMPLEMENTATION              !
%^^A                                                      !
%^^A ==================================================== !
%
%   \iffalse
%</PROTECT>
%   \fi
%   \iffalse
%<*Code>
%   \fi
%
%
%   \UWPart{Implementation}
%   \UWModule{Front Matter}
%
%   Much of this class is written using the \LaTeX3 Programming Layer;
%   this will be denoted as \LaTeXPL{}.  The \LaTeXPL{} is the first
%   piece of a new system designed to succeed \LaTeXe{} in the future.
%   However, while the programming layer is solid and remarkable,
%   a lot of presentation work still needs to be done.  Therefore,
%   this class uses \LaTeXe{} code where necessary and will hopefully
%   be slowly pulled out as needed.  The good news is that since everything
%   is more-or-less an abstraction of \TeX{}, it should work together well.
%
%   \UWSubModule{expl3 Package and Identification}
%   The |expl3| package loads the \LaTeXPL{} and is therefore required.
%   If the package is not recent enough, the class aborts and requests
%   the user update.
%    \begin{macrocode}
\RequirePackage{expl3}[2013/07/28]
\@ifpackagelater{expl3}{2013/07/28} {} {%
    \PackageError{UWMadThesis}{Version of l3kernel is too old}
      {%
        Please install an up to date version of l3kernel\MessageBreak
        using your TeX package manager or from CTAN.
      }%
    \endinput
}%
%    \end{macrocode}
%
%
%    \begin{macrocode}
\ExplSyntaxOn
%    \end{macrocode}
%
%
%   \UWSubModule{Identification and Defaults}
%
%   If the |expl3| package is recent enoughw, define some identification
%   variables (token lists).
%    \begin{macrocode}
\tl_const:Nn \c__UWMad_Class_Name_tl        {UWMadThesis}
\tl_const:Nn \c__UWMad_Class_Version_tl     {1.0}
\tl_const:Nn \c__UWMad_Class_Date_tl        {2014/04/01}
\tl_const:Nn \c__UWMad_Class_Description_tl {
    LaTeX2e+~Thesis~Class~for~UW-Madison
}
\tl_const:Nn \c__UWMad_UniversityLong_tl   {University~of~Wisconsin--Madison}
\tl_const:Nn \c__UWMad_UniversityShort_tl  {UW--Madison}
%    \end{macrocode}
%
%   Assuming the the |expl3| package is recent enough, we provide the class
%   using the \LaTeXPL{}'s provide function.
%    \begin{macrocode}
\ProvidesExplClass
    {\c__UWMad_Class_Name_tl}   {\c__UWMad_Class_Date_tl}
    {\c__UWMad_Class_Version_tl}{\c__UWMad_Class_Description_tl}
%    \end{macrocode}
%
%
%
%   In an effort to allow the thesis class to adapt to new underlying classes,
%   the class that the \UWMadClass{} loads is decalred as a mutable
%   token list.  The default is the \LaTeX{} base class \texttt{report}.
%    \begin{macrocode}
\tl_new:N   \g_UWMad_ParentClass_tl
\tl_gset:Nn \g_UWMad_ParentClass_tl {report}
%    \end{macrocode}
%
%
%
%   \UWSubModule{Options}
%
%
%   First, a command is created to throw a warning if an option that
%   violates \UWMadLong{}'s dissertation guidelines is chosen.
%    \begin{macrocode}
\msg_new:nnn{ UWMadThesis }{ Options / StyleViolation }{
    Option~'#1'~violates~\c_UWMadUniversityShort_tl{}~
    Dissertation~Guidelines;~consider~omission
}
\cs_new:Nn \__UWMad_FrontMatter_StyleWarning:n {
    \msg_warning:nnn { UWMadThesis }{ Options / StyleViolation } { #1 }
   \PassOptionsToClass{#1}{\g_UWMad_ParentClass_tl}
}
%    \end{macrocode}
%
%
%
%   Now, declare booleans for the option processing.  All new booleans
%   are false by default.
%    \begin{macrocode}
\bool_new:N       \g__UWMad_MathTweaks_bool
\bool_gset_true:N \g__UWMad_MathTweaks_bool
%    \end{macrocode}
%
%   Declare the options.
%    \begin{macrocode}
\DeclareOption{NoMath} {
    \bool_gset_false:N \g__UWMad_MathTweaks_bool
}
\DeclareOption{Quiet} {
    \msg_redirect_module:nnn { UWMadThesis } { warning } { none }
}
%    \end{macrocode}
%
%   Catch the couple of default options that violate the requirements:
%   8.5 by 11 paper for single-sided printing.
%    \begin{macrocode}
\DeclareOption{a4paper} {
    \__UWMad_FrontMatter_StyleWarning:n {\CurrentOption}
}
\DeclareOption{twoside} {
    \__UWMad_FrontMatter_StyleWarning:n {\CurrentOption}
}
%    \end{macrocode}
%
%   These options change the default report class to the
%   ones indicated.
%    \begin{macrocode}
\DeclareOption{article} {
    \tl_gset:Nn \g_UWMad_ParentClass_tl {article}
}
%    \end{macrocode}
%
%   This is a special class option for generating the documentation.
%   Users should not use this unless they know what they're doing.
%   The line below the \texttt{ParentClass} class prevents the \pkg{thumbpdf}
%   package from being loaded.
%    \begin{macrocode}
\DeclareOption{l3doc} {
    \tl_gset:Nn \g_UWMad_ParentClass_tl {l3doc}
    \tl_const:cn {ver@thumbpdf.sty} {}
}
%    \end{macrocode}
%
%   Pass all remaining options to the base class.
%    \begin{macrocode}
\DeclareOption*{
    \PassOptionsToClass
        {\CurrentOption}{\g_UWMad_ParentClass_tl}
}
%    \end{macrocode}
%
%   Process the options with some defaults and load the base class.
%    \begin{macrocode}
\ExecuteOptions{oneside,12pt}
\ProcessOptions\relax
\LoadClass{\g_UWMad_ParentClass_tl}[1995/12/01]
%    \end{macrocode}
%
%
%
%   \UWSubModule{Package Loads}
%   Load some packages that give nice features and are not
%   hyperlink sensitive.
%    \begin{macrocode}
\RequirePackage{xparse}
\RequirePackage{fixltx2e}
\RequirePackage{microtype}
\RequirePackage{array}
\RequirePackage{float}
\RequirePackage{graphicx}
\RequirePackage{setspace}
\RequirePackage{geometry}
%    \end{macrocode}
%
%
%   Load the \AmS{} suite.
%    \begin{macrocode}
\RequirePackage{amsmath}
\RequirePackage{amsfonts}
\RequirePackage{amssymb}
\RequirePackage{mathtools}
%    \end{macrocode}
%
%   And now we load some packages that give nice features and are
%   hyperlink sensitive.
%    \begin{macrocode}
\RequirePackage[noabbrev,nameinlink]{cleveref}
\RequirePackage[usenames,dvipsnames,svgnames,table,hyperref]{xcolor}
\RequirePackage{caption}
\RequirePackage{subcaption}
%    \end{macrocode}
%
%
%   Conditionally load either the \pkg{polyglossia} or \pkg{babel}
%   language packages depending on the engine in use.
%    \begin{macrocode}
\bool_if:nTF {\xetex_if_engine_p: || \luatex_if_engine_p:} {
    \RequirePackage{fontspec}
    \defaultfontfeatures{Ligatures={TeX}}
    \setmainfont
        [SmallCapsFont = {Latin~Modern~Roman~Caps}]
        {Latin~Modern~Roman}
%
    \RequirePackage{polyglossia}
    \setmainlanguage[variant = usmax]{english}
} {
    \RequirePackage{lmodern}
    \RequirePackage[T1]{fontenc}
%
    \RequirePackage[english]{babel}
}
%    \end{macrocode}
%
%
%   If links were not negated by the options, \pkg{bookmark} and
%   \pkg{hyperref} are loaded.
%    \begin{macrocode}
\RequirePackage{hyperref}
\RequirePackage{bookmark}
%    \end{macrocode}
%
%
%
%
%   And since these identifications may be desired in typsetting more,
%   where \LaTeXPL{}'s syntax will be turned off, we define some aliases.
%    \begin{macrocode}
\DeclareDocumentCommand \UWMadClass { } {
    \texttt{\c__UWMad_Class_Name_tl}~class
}
\DeclareDocumentCommand \UWMadClassVersion { } {
    \c__UWMad_Class_Version_tl
}
\DeclareDocumentCommand \UWMadClassDate { } {
    \c__UWMad_Class_Date_tl
}
\DeclareDocumentCommand \UWMadLong { } {
    \c__UWMad_UniversityLong_tl
}
\DeclareDocumentCommand \UWMadShort { } {
    \c__UWMad_UniversityShort_tl
}
%    \end{macrocode}
%
%
%
%
%
%
%
%
%   \UWSubModule{Key-Value Interface}
%
%   \begin{function}{\UWMadSetup}
%       \begin{syntax}
%           \cs{UWMadSetup}\Arg{module name}\Arg{key-value csv}
%       \end{syntax}
%   This simple command creates a user interface for the key-value system used
%   for several feature set options.
%    \begin{macrocode}
\DeclareDocumentCommand \UWMadSetup { >{\TrimSpaces} m m } {
    \keys_set:nn
        { UWMadThesis / #1 }
        { #2 }
}
%    \end{macrocode}
%   \end{function}
%
%   \iffalse
%</Code>
%   \fi
%   \iffalse
%<*Code>
%   \fi
%
%   \UWModule{Programming}\label{Imp/Programming}
%   This section outlines the Programming module for the \UWMadClass{}.
%   It encompasses thin abstractions from the standard \LaTeXPL{}'s type
%   and collection systems and provides \LaTeXe{} abstractions for
%   several other features.
%
%
%
%   \UWSubModule{Utility Commands}
%
%   Define some messages for the rest of the module.
%    \begin{macrocode}
\msg_new:nnn {UWMadThesis} {Programming/UnregisteredVariable} {
    `#1'~is~not~a~registered~#2.~~The~#2~must~be~defined~
    before~usage~by~the~function~\string\UWMad_#2_DefineLocal:n~or~
    \string\UWMad_#2_DefineGlobal:n.
}
\msg_new:nnn {UWMadThesis} {Programming/Undefined} {
    The~#2~`#1'~is~undefined.~~The~#2~must~be~defined~
    before~usage~by~the~function~\string\UWMad_#2_Define:n.
}
\msg_new:nnn {UWMadThesis} {Programming/Defined} {
    The~#2~`#1'~is~already~defined~and~will~not~altered.
}
%    \end{macrocode}
%
%
%
%   \begin{function} {
%       \UWMad_Hook_Prepend:cn,
%       \UWMad_Hook_Prepend:Nn}
%       \begin{syntax}
%           \cs{UWMad_Hook_Prepend:cn} \Arg{command name} \Arg{prepend code}
%           \cs{UWMad_Hook_Prepend:Nn} \meta{command} \Arg{prepend code}
%       \end{syntax}
%
%   These commands allow additional code to be prepended to a
%   specified command.
%
%    \begin{macrocode}
\cs_new:Nn \UWMad_Hook_Prepend:cn {
    \cs_new_eq:cc  {#1-Default:} {#1}
    \cs_gset:cn    {#1:}         {#2 \cs:w #1-Default:\cs_end:}
    \cs_undefine:c {#1}
    \cs_new_eq:cc  {#1}          {#1:}
}
\cs_new:Nn \UWMad_Hook_Prepend:Nn {
    \cs_new_eq:cN  {\string#1-Default:} #1
    \cs_gset:cn    {\string#1:}         {#2 \cs:w\string#1-Default:\cs_end:}
    \cs_undefine:N  #1
    \cs_new_eq:Nc   #1          {\string#1:}
}
%    \end{macrocode}
%   \end{function}
%
%
%   \begin{function} {
%       \UWMad_Hook_Append:cn,
%       \UWMad_Hook_Append:Nn}
%       \begin{syntax}
%           \cs{UWMad_Hook_Append:cn} \Arg{command name} \Arg{append code}
%           \cs{UWMad_Hook_Append:Nn} \meta{command} \Arg{append code}
%       \end{syntax}
%
%   These commands allow additional code to be appended to a
%   specified command.
%
%    \begin{macrocode}
\cs_new:Nn \UWMad_Hook_Append:cn {
    \cs_new_eq:cc  {#1-Default:} {#1}
    \cs_gset:cn    {#1:}        {\cs:w #1-Default:\cs_end: #2}
    \cs_undefine:c {#1}
    \cs_new_eq:cc  {#1}          {#1:}
}
\cs_new:Nn \UWMad_Hook_Append:Nn {
    \cs_new_eq:cN  {\string#1-Default:} #1
    \cs_gset:cn    {\string#1:}         {\cs:w\string#1-Default:\cs_end: #2}
    \cs_undefine:N  #1
    \cs_new_eq:Nc   #1          {\string#1:}
}
%    \end{macrocode}
%   \end{function}
%
%
%
%
%   \begin{function} {
%       \UWMad_Definition_Swap:Nn,
%       \UWMad_Definition_Reset:N,
%       \UWMad_Definition_Swap:cn,
%       \UWMad_Definition_Reset:c}
%
%       \begin{syntax}
%           \cs{UWMad_Definition_Swap:Nn} \meta{command} \Arg{replacement code}
%           \cs{UWMad_Definition_Reset:N} \meta{command}
%           \cs{UWMad_Definition_Swap:cn} \Arg{command name} \Arg{replacement code}
%           \cs{UWMad_Definition_Reset:c} \Arg{command name}
%       \end{syntax}
%
%   These commands \enquote{swap} in a new definition of a command and,
%   when called, reset it to it's default definition.
%
%    \begin{macrocode}
\cs_new:Nn \UWMad_Definition_Swap:Nn {
    \cs_if_exist:NTF #1 {
        \cs_new_eq:cN  {\string#1-Default:}  #1
        \cs_gset_eq:Nc  #1 {#2}
    } {
        \cs_new:Nn #1 {#2}
    }
}
\cs_new:Nn \UWMad_Definition_Reset:N {
    \cs_if_exist:cTF {\string#1-Default:} {
        \cs_gset_eq:Nc  #1              {\string#1-Default:}
        \cs_undefine:c  {\string#1-Default:}
    } { }
}
\cs_generate_variant:Nn \UWMad_Definition_Swap:Nn {cn}
\cs_generate_variant:Nn \UWMad_Definition_Reset:N {c}
%    \end{macrocode}
%   \end{function}
%
%
%
%
%
%
%   \begin{function}{
%       \UWMad_File_GetExtension:nNN}
%   \begin{syntax}
%   \cs{UWMad_File_GetExtension:nNN}\Arg{path}\meta{tl var 1}\meta{tl var 2}
%   \end{syntax}
%   Searches through the given \Arg{file path} for an extension identifier
%   (\texttt{.} by default) in the path.  If one is found, the path sans extension is
%   assigned to \meta{tl var 1} with the extension assigned to \meta{tl var 2}.
%
%   Initializations of variables and booleans used in the function
%    \begin{macrocode}
\tl_new:N   \g__UWMad_File_Path_tl
\tl_new:N   \g__UWMad_File_Extension_tl
\tl_new:N   \g__UWMad_File_Marker_Extension_tl
\tl_new:N   \g__UWMad_File_Marker_Directory_tl
\tl_gset:Nn \g__UWMad_File_Marker_Extension_tl {.}
\tl_gset:Nn \g__UWMad_File_Marker_Directory_tl {/}
\bool_new:N \g__UWMad_File_IsExtensionFound_bool
\bool_new:N \g__UWMad_File_IsDirectoryFound_bool
%    \end{macrocode}
%
%   Define the body of the function.
%    \begin{macrocode}
\cs_new:Nn \UWMad_File_GetExtension:nNN {

    \tl_gclear:N \g__UWMad_File_Path_tl
    \tl_gclear:N \g__UWMad_File_Extension_tl
    \bool_set_false:N \g__UWMad_File_IsExtensionFound_bool
    \bool_set_false:N \g__UWMad_File_IsDirectoryFound_bool

    \tl_set:Nx \l_tmpa_tl {
        \tl_reverse:V {#1}
    }

\tl_map_inline:Nn \l_tmpa_tl {

    \tl_set:Nn {\l_tmpb_tl}{##1}

    \bool_if:NTF \g__UWMad_File_IsExtensionFound_bool {
        \tl_gput_left:Nn \g__UWMad_File_Path_tl {##1}
    } {
        \bool_if:NTF \g__UWMad_File_IsDirectoryFound_bool {
            \tl_gput_left:Nn \g__UWMad_File_Path_tl {##1}
        } {
            \tl_if_eq:NNTF \l_tmpb_tl \g__UWMad_File_Marker_Extension_tl {
                \bool_set_true:N \g__UWMad_File_IsExtensionFound_bool
            } {
                \tl_if_eq:NNTF \l_tmpb_tl \g__UWMad_File_Marker_Directory_tl {
                    \bool_gset_true:N \g__UWMad_File_IsDirectoryFound_bool
                    \tl_gput_left:Nn \g__UWMad_File_Path_tl {##1}
                } {
                    \tl_gput_left:Nn \g__UWMad_File_Extension_tl {##1}
                }
            }
        }
    }
}

    \bool_if:NTF \g__UWMad_File_IsExtensionFound_bool { } {
        \tl_gput_right:NV \g__UWMad_File_Path_tl {
            \g__UWMad_File_Extension_tl
        }
        \tl_gclear:N \g__UWMad_File_Extension_tl
    }

    \tl_gset_eq:NN
        #2
        \g__UWMad_File_Path_tl
    \tl_gset_eq:NN
        #3
        \g__UWMad_File_Extension_tl
}
%    \end{macrocode}
%    \end{function}
%
%
%
%
%
%   \begin{function}{
%       \__UWMad_IfDefined:nnnnT,
%       \__UWMad_IfUndefined:nnnnT}
%
%   \begin{syntax}
%       \cs{__UWMad_IfDefined:nnnnT}\Arg{Prefix}\Arg{ID}\Arg{Suffix}\Arg{Type}\Arg{Code}
%       \cs{__UWMad_IfUndefined:nnnnT}\Arg{Prefix}\Arg{ID}\Arg{Suffix}\Arg{Type}\Arg{Code}
%   \end{syntax}
%
%   These commands accept a \marg{Prefix}, an \marg{ID}, a \marg{Suffix},
%   a \marg{Type}, and \marg{Code}.  It determines if a command named by the
%   concatenation of \marg{Prefix}, \marg{ID}, and \marg{Suffix}
%   is defined or not and executes \marg{Code} depending on the existence.
%
%    \begin{macrocode}
\cs_new:Nn \__UWMad_IfDefined:nnnnT{
    \cs_if_exist:cTF {#1#2#3} {
        #5
    }{
            \msg_error:nnnn
                {UWMadThesis}
                {Programming/Undefined}
                {#2}
                {#4}
    }
}
\cs_new:Nn \__UWMad_IfUndefined:nnnnT{
    \cs_if_free:cTF {#1#2#3} {
        #5
    }{
            \msg_warning:nnnn
                {UWMadThesis}
                {Programming/Defined}
                {#2}
                {#4}
    }
}
%    \end{macrocode}
%   \end{function}
%
%
%
%   \begin{function}{
%       \__UWMad_IfDefined:nT,
%       \__UWMad_IfUndefined:nT}
%   \begin{syntax}
%       \cs{__UWMad_IfDefined:nT}\Arg{CommandName}\Arg{TrueCode}
%       \cs{__UWMad_IfUndefined:nT}\Arg{CommandName}\Arg{TrueCode}
%   \end{syntax}
%   These commands are simplifications of the above commands and
%   that only take a \marg{CommandName} and \marg{TrueCode}.
%
%    \begin{macrocode}
\cs_new:Nn \__UWMad_IfDefined:nT{
    \_UWMad_IfDefined:nnnnT{}{#1}{}{command}{#2}
}
\cs_new:Nn \__UWMad_IfUndefined:nT{
    \_UWMad_IfUndefined:nnnnT{}{#1}{}{command}{#2}
}
%    \end{macrocode}
%   \end{function}
%
%
%
%
%
%^^A ======================================================================= %
%^^A                     Collection Creation Commands                        %
%^^A ======================================================================= %
%
%   \UWSubModule{Collections}
%   In the following subsections, commands that create and manipulate
%   various collection data types will be discussed.  The collections
%   currently implemented are stacks (LIFO),
%   queues (FIFO), deques (LIFO+FIFO), and hashes (key-value pairs).
%
%   All of the collection systems are thin abstractions of \LaTeXPL{}'s
%   \texttt{l3tl}, \texttt{l3seq}, and \texttt{l3prop} modules to avoid developing
%   one-shot systems while allowing more endeavoring authors access to the
%   features without learning \LaTeX3{} programming if they load the
%   abstractions.
%
%
%
%
%
%
%^^A ======================================================================= %
%^^A                          Stack Creation Commands                        %
%^^A ======================================================================= %
%
%   \UWSubSubModule{Stacks}
%   This set of commands is a simple system for creating and working with
%   stacks.  Stacks are a last-in first-out collection data type; this means
%   that the data element (in this any unexpanded token/token list) last
%   pushed on to the stack is the first popped.  Data elements can also be
%   walked (iterated over) with an inline callback in a LIFO sense.
%
%
%
%   \begin{function}{
%       \__UWMad_Stack_IfDefined:nT,
%       \__UWMad_Stack_IfUndefined:nT}
%       \begin{syntax}
%           \cs{__UWMad_Stack_IfDefined:nT}\Arg{stack name}\Arg{true code}
%           \cs{__UWMad_Stack_IfUndefined:nT}\Arg{stack name}\Arg{true code}
%       \end{syntax}
%   Shortcuts for the more general commands outlined above.
%
%    \begin{macrocode}
\cs_new:Nn \__UWMad_Stack_IfDefined:nT {
    \__UWMad_IfDefined:nnnnT{g__UWMad_Stack_}{#1}{}{Stack}{#2}
}
\cs_new:Nn \__UWMad_Stack_IfUndefined:nT{
    \__UWMad_IfUndefined:nnnnT{g__UWMad_Stack_}{#1}{}{Stack}{#2}
}
%    \end{macrocode}
%   \end{function}
%
%
%   \begin{function}{\UWMad_Stack_Define:n}
%   Define a new Stack.
%
%    \begin{macrocode}
\cs_new:Nn \UWMad_Stack_Define:n {
    \__UWMad_Stack_IfUndefined:nT {#1} {
        \tl_new:c {g__UWMad_Stack_#1}
    }
}
%    \end{macrocode}
%   \end{function}
%
%
%   \begin{function}{\UWMad_Stack_Clear:n}
%   Clear but do not undefine a defined Stack.
%
%    \begin{macrocode}
\cs_new:Nn \UWMad_Stack_Clear:n {
    \__UWMad_Stack_IfDefined:nT {#1} {
        \tl_gclear:c   {g__UWMad_Stack_#1}
    }
}
%    \end{macrocode}
%   \end{function}
%
%
%   \begin{function}{\UWMad_Stack_Delete:n}
%   Clear and undefine a defined Stack.
%
%    \begin{macrocode}
\cs_new:Nn \UWMad_Stack_Delete:n {
    \__UWMad_Stack_IfDefined:nT {#1} {
        \tl_gclear:c   {g__UWMad_Stack_#1}
        \cs_undefine:c {g__UWMad_Stack_#1}
    }
}
%    \end{macrocode}
%   \end{function}
%
%
%   \begin{function}{\UWMad_Stack_Push:nn}
%   Push a value on to a defined Stack.
%
%    \begin{macrocode}
\cs_new:Nn \UWMad_Stack_Push:nn {
    \__UWMad_Stack_IfDefined:nT {#1} {
        \tl_gput_left:cn {g__UWMad_Stack_#1} {#2}
    }
}
%
%
\cs_generate_variant:Nn \tl_head:N { c }
\cs_generate_variant:Nn \tl_tail:N { c }
%    \end{macrocode}
%   \end{function}
%
%
%   \begin{function}{\UWMad_Stack_Pop:n}
%   Pop a value off a defined Stack and place it in the
%   input stream.
%
%    \begin{macrocode}
\cs_new:Nn \UWMad_Stack_Pop:n {
    \__UWMad_Stack_IfDefined:nT {#1} {
        \tl_set:Nf \l_tmpa_tl          {\tl_head:c {g__UWMad_Stack_#1}}
        \tl_set:cf {g__UWMad_Stack_#1} {\tl_tail:c {g__UWMad_Stack_#1}}
        \tl_use:N \l_tmpa_tl
    }
}
%    \end{macrocode}
%   \end{function}
%
%
%   \begin{function}{\UWMad_Stack_Walk:nn}
%   Iterate of the elements of a defined Stack in a FILO sense
%   with supplied code.
%
%    \begin{macrocode}
\cs_new:Nn \UWMad_Stack_Walk:nn {
    \tl_map_inline:cn {g__UWMad_Stack_#1} {#2}
}
%    \end{macrocode}
%    \end{function}
%
%
%
%^^A ======================================================================= %
%^^A                          Queue Creation Commands                        %
%^^A ======================================================================= %
%
%   \UWSubSubModule{Queues}
%   This set of commands is a simple system for creating and working with
%   queue.  Queues are a first-in first-out collection data type; this means
%   that the data element (in this any unexpanded token/token list) first
%   pushed on to the queue is the first popped.  Data elements can also be
%   walked (iterated over) with an inline callback in a FIFO sense.
%
%
%
%   \begin{function} {
%       \__UWMad_Queue_IfDefined:nT,
%       \__UWMad_Queue_IfUndefined:nT}
%   Shortcuts for the more general commands outlined above.
%
%    \begin{macrocode}
\cs_new:Nn \__UWMad_Queue_IfDefined:nT {
    \__UWMad_IfDefined:nnnnT{g__UWMad_Queue_}{#1}{}{Queue}{#2}
}
\cs_new:Nn \__UWMad_Queue_IfUndefined:nT{
    \__UWMad_IfUndefined:nnnnT{g__UWMad_Queue_}{#1}{}{Queue}{#2}
}
%    \end{macrocode}
%   \end{function}
%
%
%   \begin{function}{\UWMad_Queue_Define:n}
%   Define a new Queue.
%
%    \begin{macrocode}
\cs_new:Nn \UWMad_Queue_Define:n {
    \__UWMad_Queue_IfUndefined:nT {#1} {
        \tl_new:c {g__UWMad_Queue_#1}
    }
}
%    \end{macrocode}
%   \end{function}
%
%
%   \begin{function}{\UWMad_Queue_Clear:n}
%   Clear but do not undefine a defined Queue.
%
%    \begin{macrocode}
\cs_new:Nn \UWMad_Queue_Clear:n {
    \__UWMad_Queue_IfDefined:nT {#1} {
        \tl_gclear:c {g__UWMad_Queue_#1}
    }
}
%    \end{macrocode}
%   \end{function}
%
%
%   \begin{function}{\UWMad_Queue_Delete:n}
%   Clear and undefine a defined Queue.
%
%    \begin{macrocode}
\cs_new:Nn \UWMad_Queue_Delete:n {
    \__UWMad_Queue_IfDefined:nT {#1} {
        \tl_gclear:c    {g__UWMad_Queue_#1}
         \cs_undefine:c {g__UWMad_Queue_#1}
    }
}
%    \end{macrocode}
%   \end{function}
%
%
%   \begin{function}{\UWMad_Queue_Pop:nn}
%   Push an item on to the start of a defined Queue.
%
%    \begin{macrocode}
\cs_new:Nn \UWMad_Queue_Push:nn {
    \__UWMad_Queue_IfDefined:nT {#1} {
        \tl_gput_left:cn {g__UWMad_Queue_#1} {{#2}}
    }
}
%
%
\cs_generate_variant:Nn \tl_head:N { c }
\cs_generate_variant:Nn \tl_tail:N { c }
%    \end{macrocode}
%   \end{function}
%
%
%   \begin{function}{\UWMad_Queue_Pop:n}
%   Pop an item from the end of a defined Queue
%   and place it in the input stream.
%
%    \begin{macrocode}
\cs_new:Nn \UWMad_Queue_Pop:n {
    \__UWMad_Queue_IfDefined:nT {#1} {
        \tl_reverse:c   {g__UWMad_Queue_#1}
        \tl_set:Nf \l_tmpa_tl
            {\tl_head:c {g__UWMad_Queue_#1}}
        \tl_set:cf      {g__UWMad_Queue_#1}
            {\tl_tail:c {g__UWMad_Queue_#1}}
        \tl_reverse:c   {g__UWMad_Queue_#1}
        \tl_use:N \l_tmpa_tl
    }
}
%    \end{macrocode}
%   \end{function}
%
%
%   \begin{function}{\UWMad_Queue_Walk:nn}
%   Iterate of the elements of a defined Queue in a FIFO sense
%   with supplied code.
%
%    \begin{macrocode}
\cs_new:Nn \UWMad_Queue_Walk:nn {
    \__UWMad_Queue_IfDefined:nT {#1} {
        \group_begin:
            \tl_reverse:c     {g__UWMad_Queue_#1}
            \tl_map_inline:cn {g__UWMad_Queue_#1} {#2}
        \group_end:
    }
}
%    \end{macrocode}
%   \end{function}
%
%
%   \begin{function}{\UWMad_Queue_IfEmpty:nTF}
%   Execute true/false code depending on the emptiness
%   of a defined Queue.
%
%    \begin{macrocode}
\cs_new:Nn \UWMad_Queue_IfEmpty:nTF {
    \__UWMad_Queue_IfDefined:nT {#1} {
        \tl_if_empty:cTF {g__UWMad_Queue_#1}{
            #2
        }{
            #3
        }
    }
}
%    \end{macrocode}
%    \end{function}
%
%
%^^A ======================================================================= %
%^^A                          Deque Creation Commands                        %
%^^A ======================================================================= %
%
%   \UWSubSubModule{Deques}
%   This set of commands is a simple system for creating and working with
%   double-ended queues (deques, pronounced \textit{deck}).  Deques are a
%   generalization of stacks and queues in that data can be pushed, popped,
%   and walked from either end of the list (i.e., LIFO+FIFO).
%
%
%
%   \begin{function} {
%       \__UWMad_Deque_IfDefined:nT,
%       \__UWMad_Deque_IfUndefined:nT}
%   Shortcuts for the more general  commands outlined above.
%
%    \begin{macrocode}
\cs_new:Nn \__UWMad_Deque_IfDefined:nT {
    \__UWMad_IfDefined:nnnnT{g__UWMad_Deque_}{#1}{}{Deque}{#2}
}
\cs_new:Nn \__UWMad_Deque_IfUndefined:nT{
    \__UWMad_IfUndefined:nnnnT{g__UWMad_Deque_}{#1}{}{Deque}{#2}
}
%    \end{macrocode}
%   \end{function}
%
%
%   \begin{function}{\UWMad_Deque_Define:n}
%   Define a new Deque.
%
%    \begin{macrocode}
\cs_new:Nn \UWMad_Deque_Define:n {
    \__UWMad_Deque_IfUndefined:nT {#1} {
        \seq_new:c {g__UWMad_Deque_#1}
    }
}
%    \end{macrocode}
%   \end{function}
%
%
%   \begin{function}{\UWMad_Deque_Clear:n}
%   Clear but do not undefine a defined Deque.
%
%    \begin{macrocode}
\cs_new:Nn \UWMad_Deque_Clear:n {
    \__UWMad_Deque_IfDefined:nT {#1} {
        \seq_gclear:c  {g__UWMad_Deque_#1}
    }
}
%    \end{macrocode}
%   \end{function}
%
%
%   \begin{function}{\UWMad_Deque_Delete:n}
%   Clear and undefine a defined Deque.
%
%    \begin{macrocode}
\cs_new:Nn \UWMad_Deque_Delete:n {
    \__UWMad_Deque_IfDefined:nT {#1} {
        \seq_gclear:c  {g__UWMad_Deque_#1}
        \cs_undefine:c {g__UWMad_Deque_#1}
    }
}
%    \end{macrocode}
%   \end{function}
%
%
%   \begin{function}{
%       \UWMad_Deque_PushLeft:nn,
%       \UWMad_Deque_PushRight:nn}
%   Push an element on to the left or right of a defined Deque.
%
%    \begin{macrocode}
\cs_new:Nn \UWMad_Deque_PushLeft:nn {
    \__UWMad_Deque_IfDefined:nT {#1} {
        \seq_gput_left:cn  {g__UWMad_Deque_#1} {#2}
    }
}
\cs_new:Nn \UWMad_Deque_PushRight:nn {
    \__UWMad_Deque_IfDefined:nT {#1} {
        \seq_gput_right:cn {g__UWMad_Deque_#1} {#2}
    }
}
%    \end{macrocode}
%   \end{function}
%
%
%   \begin{function}{
%       \UWMad_Deque_PopLeft:nn,
%       \UWMad_Deque_PopRight:nn}
%   Pop an element from the left or right of a defined Deque
%   and place it into the input stream.
%
%    \begin{macrocode}
\cs_new:Nn \UWMad_Deque_PopLeft:n {
    \__UWMad_Deque_IfDefined:nT {#1} {
        \seq_gpop_left:cN  {g__UWMad_Deque_#1} \l_tmpa_tl
        \tl_use:N \l_tmpa_tl
    }
}
\cs_new:Nn \UWMad_Deque_PopRight:n {
    \__UWMad_Deque_IfDefined:nT {#1} {
        \seq_gpop_right:cN {g__UWMad_Deque_#1} \l_tmpa_tl
        \tl_use:N \l_tmpa_tl
    }
}
%    \end{macrocode}
%   \end{function}
%
%
%   \begin{function}{
%       \UWMad_Deque_WalkLeftToRight:nn,
%       \UWMad_Deque_WalkRightToLeft:nn}
%   Iterate over the elements left-to-right or right-to-left
%   of a defined Deque with supplied code.
%
%    \begin{macrocode}
\cs_new:Nn \UWMad_Deque_WalkLeftToRight:nn {
    \__UWMad_Deque_IfDefined:nT {#1} {
        \seq_map_inline:cn {g__UWMad_Deque_#1} {#2}
    }
}
\cs_generate_variant:Nn \seq_reverse:N {c}
\cs_new:Nn \UWMad_Deque_WalkRightToLeft:nn {
    \__UWMad_Deque_IfDefined:nT {#1} {
        \group_begin:
            \seq_reverse:c     {g__UWMad_Deque_#1}
            \seq_map_inline:cn {g__UWMad_Deque_#1} {#2}
        \group_end:
    }
}
%    \end{macrocode}
%    \end{function}
%
%
%
%
%
%^^A =========================================================================== %
%^^A                 Hashes (Associative Arrays) with LaTeX3                     %
%^^A =========================================================================== %
%
%   \UWSubSubModule{Hashes}
%   This set of commands is a simple system for creating and working with
%   hashes (more often called associative arrays or dictionaries, but erring
%   on the side of usablility, Ruby's jargon will be used). Hashes are a
%   type of array that indexes values by (at least in \LaTeX{}) alphanumeric
%   keys instead of just integers.
%   Data can be set by key, retrieved by key, unset by key, deleted, and walked.
%
%   A hash walk, like the collection walks above, iterates through all of the
%   keys and values in the hash while applying a user supplied function.
%   However, unlike the collection walks, \textbf{a hash's walk order is not
%   gauranteed to be the set order}.  If walk order is needed to be
%   gauranteed, see the previous collection data types.
%
%   The system is a thin abstraction of \LaTeXPL's
%   \texttt{l3prop} module to avoid developing a one-shot system while allowing more
%   endeavoring authors access to the feature without learning \LaTeX3{}
%   programming.
%
%
%    \begin{macrocode}
\cs_generate_variant:Nn \prop_gput:Nnn   { c x n   }
\cs_generate_variant:Nn \prop_if_in:NnTF { c x TF  }
\cs_generate_variant:Nn \prop_if_in:NnTF { c f TF  }
\cs_generate_variant:Nn \prop_get:Nn     { c x     }
\cs_generate_variant:Nn \prop_get:Nn     { c f     }
\cs_generate_variant:Nn \prop_get:NnNTF  { c x N TF}
\cs_generate_variant:Nn \prop_gremove:Nn { c x     }
%    \end{macrocode}
%
%
%   \begin{function} {
%       \__UWMad_Hash_IfDefined:nT,
%       \__UWMad_Hash_IfUndefined:nT}
%   Shortcuts for the more general commands outlined above.
%
%    \begin{macrocode}
\cs_new:Nn \__UWMad_Hash_IfDefined:nT {
    \__UWMad_IfDefined:nnnnT{g__UWMad_Hash_}{#1}{}{Hash}{#2}
}
\cs_new:Nn \__UWMad_Hash_IfUndefined:nT{
    \__UWMad_IfUndefined:nnnnT{g__UWMad_Hash_}{#1}{}{Hash}{#2}
}
%    \end{macrocode}
%   \end{function}
%
%
%   \begin{function}{\UWMad_Hash_Define:n}
%   Define a new Hash.
%
%    \begin{macrocode}
\cs_new:Nn \UWMad_Hash_Define:n {
    \__UWMad_Hash_IfUndefined:nT {#1} {
        \prop_new:c {g__UWMad_Hash_#1}
    }
}
%    \end{macrocode}
%   \end{function}
%
%
%   \begin{function}{\UWMad_Hash_Set:nnn}
%   Set the value of a key of a defined Hash.
%
%   \begin{syntax}
%       \cs{UWMad_Hash_Set:nnn}\marg{HashID}\marg{Key}\marg{Value}
%   \end{syntax}
%
%    \begin{macrocode}
\cs_new:Nn \UWMad_Hash_Set:nnn {
    \__UWMad_Hash_IfDefined:nT {#1} {
        \prop_gput:cxn {g__UWMad_Hash_#1}{#2}{#3}
    }
}
%    \end{macrocode}
%   \end{function}
%
%
%   \begin{function}{\UWMad_Hash_Get:nn}
%   Get the value of a key of a defined Hash and place
%   it into the input stream.
%
%    \begin{macrocode}
\cs_generate_variant:Nn \prop_get:cn {cf}
\cs_new:Nn \UWMad_Hash_Get:nn {
    \__UWMad_Hash_IfDefined:nT {#1} {
        \prop_get:cf {g__UWMad_Hash_#1}{#2}
    }
}
%    \end{macrocode}
%   \end{function}
%
%
%   \begin{function}{\UWMad_Hash_Unset:nn}
%   Undefine a key-value pair in a defined Hash.
%
%    \begin{macrocode}
\cs_new:Nn \UWMad_Hash_Unset:nn {
    \__UWMad_Hash_IfDefined:nT {#1} {
        \prop_gremove:cx {g__UWMad_Hash_#1} {#2}
    }
}
%    \end{macrocode}
%   \end{function}
%
%
%   \begin{function}{\UWMad_Hash_IfKeySet:nnTF}
%   Execute true/false code depending on if a key is set in
%   a defined Hash.
%
%    \begin{macrocode}
\cs_generate_variant:Nn \tl_to_lowercase:n {f}
\cs_new:Nn \UWMad_Hash_IfKeySet:nnTF {
    \__UWMad_Hash_IfDefined:nT {#1} {
        \prop_if_in:cfTF {g__UWMad_Hash_#1} {\tl_trim_spaces:n{#2}} {
            #3
        }{
            #4
        }
    }
}
%    \end{macrocode}
%   \end{function}
%
%
%   \begin{function}{\UWMad_Hash_Walk:nn}
%   Iterate over the key-value pairs of a defined Hash with
%   supplied code. \textbf{No order is gauranteed.}
%
%    \begin{macrocode}
\cs_new:Nn \UWMad_Hash_Walk:nn {
    \__UWMad_Hash_IfDefined:nT {#1} {
        \prop_map_inline:cn {g__UWMad_Hash_#1} {#2}
    }
}
%    \end{macrocode}
%   \end{function}
%
%
%   \begin{function}{\UWMad_Hash_Delete:n}
%   Clear and undefine a defined Hash.
%
%    \begin{macrocode}
\cs_new:Nn \UWMad_Hash_Delete:n {
    \__UWMad_Hash_IfDefined:nT {#1} {
        \prop_gclear:c {g__UWMad_Hash_#1}
        \cs_undefine:c {g__UWMad_Hash_#1}
    }
}
%    \end{macrocode}
%    \end{function}
%
%
%
%
%^^A ==================================================================== %
%^^A                         LaTeX2e Abstractions                         %
%^^A ==================================================================== %
%
%   \UWSubModule{User-Level Abstractions}
%
%   The commands that follow are \LaTeXe{}-like commands that use the
%   \LaTeXPL{} as the underlying system.  \textbf{The commands are not loaded
%   by default; they must be invoked by calling the following command.}
%
%
%   \UWSubSubModule{Utility Commands}
%
%
%   \begin{function}{\IfCommandExists,\IfCommandDoesNotExist}
%   This command pair is used instead of \LaTeX{}'s \cs{@ifundefined}.
%   Since it is \eTeX{}, this command will allow for a switch to
%   \cs{@ifundefined} if problems arise from non-\eTeX{} users in the
%   future.
%
%   \begin{syntax}
%       \cs{IfCommandExists}\marg{Command Name}\marg{True}\marg{False}
%       \cs{IfCommandDoesNotExist}\marg{Command Name}\marg{True}\marg{False}
%   \end{syntax}
%
%    \begin{macrocode}
\DeclareDocumentCommand \IfCommandExistsTF { m +m +m } {
    \cs_if_exist:cTF {#1}{
        #2
    }{
        #3
    }
}
\DeclareDocumentCommand \IfCommandDoesNotExistTF { m +m +m } {
    \cs_if_free:cTF {#1}{
        #2
    }{
        #3
    }
}
%    \end{macrocode}
%   \end{function}
%
%
%
%   \begin{function}{\IfStringEmpty}
%   Checks if a given string is composed of no characters or just blank spaces.
%
%   \begin{syntax}
%       \cs{IfStringEmpty}\marg{String}\marg{True}\marg{False}
%   \end{syntax}
%
%    \begin{macrocode}
\cs_generate_variant:Nn \tl_if_blank:nTF {fTF}
\DeclareDocumentCommand \IfEmptyTF { m +m +m } {
    \tl_if_blank:fTF {#1}{
        #2
    }{
        #3
    }
}
%    \end{macrocode}
%   \end{function}
%
%
%
%   \begin{function}{\IfCommandEmpty}
%   Determines if a commands contains no or only space after one expansion.
%
%   \begin{syntax}
%       \cs{IfCommandEmpty}\marg{Command}\marg{True}\marg{False}
%   \end{syntax}
%
%    \begin{macrocode}
\DeclareDocumentCommand \IfCommandEmptyTF { m +m +m }{
    \tl_if_blank:oTF{#1}{
        #2
    }{
        #3
    }
}
%    \end{macrocode}
%   \end{function}
%
%
%
%
%
%
%^^A ==================================================================== %
%^^A                      Command Creator System                          %
%^^A ==================================================================== %
%
%   \UWSubSubModule{Command Creators}
%
%
%   \begin{function}{\MakeCommand,\ReMakeCommand}
%   If the requested command is not defined, \cs{MakeCommand} will create it;
%   however, if the requested command is already defined, \cs{MakeCommand} will
%   throw a warning and not make the command.
%   If the requested command is defined, \cs{ReMakeCommand} will redefine it;
%   however, if the requested command is not defined, \cs{ReMakeCommand} will
%   throw a warning and not make the command.
%
%   \begin{syntax}
%       \cs{MakeCommand}\marg{Command Name}\marg{Code}
%       \cs{ReMakeCommand}\marg{Command Name}\marg{Code}
%   \end{syntax}
%
%    \begin{macrocode}
\DeclareDocumentCommand \MakeCommand { O{} m +m } {
    \cs_if_free:cTF {#2} {
        \cs_set:cpn {#2} #1 {#3}
    }{
        \msg_warning:nnnn
            {UWMadThesis}{Programming/Defined}{#2}{command}
    }
}
\DeclareDocumentCommand \ReMakeCommand { O{} m +m }{
    \cs_if_exist:cTF {#2} {
        \cs_set:cpn {#2} #1 {#3}
    }{
        \msg_error:nnnn
            {UWMadThesis}{Programming/Undefined}{#2}{command}
    }
}
%    \end{macrocode}
%   \end{function}
%
%
%
%   \begin{function}{\MakeGlobalCommand}
%   Similar to \cs{MakeCommand} except the creation is made regardless of the
%   requested command's definition and the creation is global.
%
%   \begin{syntax}
%       \cs{MakeGlobalCommand}\marg{Command Name}\marg{Code}
%   \end{syntax}
%
%    \begin{macrocode}
\DeclareDocumentCommand \MakeGlobalCommand { O{} +m m } {
    \cs_gset:cpn {#2} #1 {#3}
}
%    \end{macrocode}
%   \end{function}
%
%
%   \begin{function}{\MakeCommandUndefined}
%   Globally undefines the command specified by \marg{Command Name}.
%
%   \begin{syntax}
%       \cs{MakeCommandUndefined}\marg{Command Name}
%   \end{syntax}
%
%    \begin{macrocode}
\DeclareDocumentCommand \MakeCommandUndefined { m } {
    \cs_undefine:c {#1}
}
%    \end{macrocode}
%   \end{function}
%
%
%
%   \begin{function}{\CopyCommand}
%   Copies the defintion of the command named \marg{Command Name 1} to
%   a new command named \marg{Command Name 2}.  If \marg{Command Name 2}
%   already has a definition, \cs{CopyCommand} will throw a warning
%   \emph{but} still make the copy.
%
%   \begin{syntax}
%       \cs{CopyCommand}\marg{Command Name 1}\marg{Command Name 2}
%   \end{syntax}
%
%    \begin{macrocode}
\DeclareDocumentCommand \CopyCommand { m m } {
    \cs_if_free:cTF {#1} {
        \cs_if_free:cTF {#2} {
            \cs_gset_eq:cc {#2}{#1}
        }{
            \msg_warning:nnnn
                {UWMadThesis}{Programming/Defined}{#2}{command}
        }
    }{
        \msg_warning:nnnn
            {UWMadThesis}{Programming/Defined}{#1}{command}
    }
}
%    \end{macrocode}
%   \end{function}
%
%
%
%
%
%   \UWSubSubModule{Types}
%
%   \begin{function}{
%       \CreateBoolean,
%       \CreateBooleanTrue,
%       \CreateBooleanFalse,
%       \SetBooleanTrue,
%       \SetBooleanFalse,
%       \IfBooleanTrueTF,
%       \IfBooleanFalseTF}
%   \LaTeXe{} version of the Boolean Type system above.
%    \begin{macrocode}
\DeclareDocumentCommand \CreateBoolean { m } {
    \bool_new:c {g__UWMad_Programming_API_#1_bool}
}
\DeclareDocumentCommand \CreateBooleanTrue { m } {
    \bool_new:c       {g__UWMad_Programming_API_#1_bool}
    \bool_gset_true:c {g__UWMad_Programming_API_#1_bool}
}
\DeclareDocumentCommand \CreateBooleanFalse { m } {
    \bool_new:c        {g__UWMad_Programming_API_#1_bool}
}
\DeclareDocumentCommand \SetBooleanTrue { m } {
    \bool_gset_true:c {g__UWMad_Programming_API_#1_bool}
}
\DeclareDocumentCommand \SetBooleanFalse { m } {
    \bool_gset_false:c {g__UWMad_Programming_API_#1_bool}
}
\DeclareDocumentCommand \IfBooleanTrueTF { m +m +m } {
    \bool_if:cTF {g__UWMad_Programming_API_#1_bool} {
        #2
    } {
        #3
    }
}
\DeclareDocumentCommand \IfBooleanFalseTF { m +m +m } {
    \bool_if:cTF {g__UWMad_Programming_API_#1_bool} {
        #3
    } {
        #2
    }
}
%    \end{macrocode}
%    \end{function}
%
%
%
%   \begin{function}{
%       \CreateLength,
%       \AddToLength,
%       \SetLength,
%       \ValueOfLength,
%       \IfLengthTF}
%   \LaTeXe{} version of the Boolean Type system above.
%    \begin{macrocode}
\DeclareDocumentCommand \CreateLength { m m } {
    \dim_new:c   {g__UWMad_Programming_API_#1_dim}
    \dim_gset:cn {g__UWMad_Programming_API_#1_dim} {#2}
}
\DeclareDocumentCommand \AddToLength { m m } {
    \dim_gadd:cn {g__UWMad_Programming_API_#1_dim} {#2}
}
\DeclareDocumentCommand \SetLength { m m } {
    \dim_gset:cn {g__UWMad_Programming_API_#1_dim} {#2}
}
\DeclareDocumentCommand \ValueOfLength { m } {
    \dim_use:c {g__UWMad_Programming_API_#1_dim}
}
\DeclareDocumentCommand \IfLengthTF { m m m +m +m } {
    \dim_compare:nNnTF {#1} #2 {#3} {
        #4
    } {
        #5
    }
}
%    \end{macrocode}
%    \end{function}
%
%
%
%   \begin{function}{
%       \CreateCounter,
%       \AddToCounter,
%       \StepCounter,
%       \SetCounter,
%       \ValueOfCounter,
%       \IfCounterTF,
%       \CounterToArabic,
%       \CounterToALPHA,
%       \CounterToAlpha,
%       \CounterToROMAN,
%       \CounterToRoman}
%   \LaTeXe{} version of the Counter Type system above.
%    \begin{macrocode}
\DeclareDocumentCommand \CreateCounter { m m } {
    \int_new:c   {g__UWMad_Programming_API_#1_int}
    \int_gset:cn {g__UWMad_Programming_API_#1_int} {#2}
}
\DeclareDocumentCommand \AddToCounter { m m } {
    \int_gadd:cn {g__UWMad_Programming_API_#1_int} {#2}
}
\DeclareDocumentCommand \StepCounter { m m } {
    \int_gincr:cn {g__UWMad_Programming_API_#1_int} {#2}
}
\DeclareDocumentCommand \SetCounter { m m } {
    \int_gset:cn {g__UWMad_Programming_API_#1_int} {#2}
}
\DeclareDocumentCommand \ValueOfCounter { m m } {
    \int_use:c {g__UWMad_Programming_API_#1_int}
}
\DeclareDocumentCommand \IfCounterTF { m m m +m +m } {
    \int_compare:nNnTF {#1} {#2} {#3} {
        #4
    } {
        #5
    }
}
\DeclareDocumentCommand \CounterToArabic { m } {
    \int_to_arabic:c {g__UWMad_Programming_API_#1_int}
}
\DeclareDocumentCommand \CounterToALPHA { m } {
    \int_to_Alph:c {g__UWMad_Programming_API_#1_int}
}
\DeclareDocumentCommand \CounterToAlpha { m } {
    \int_to_alph:c {g__UWMad_Programming_API_#1_int}
}
\DeclareDocumentCommand \CounterToROMAN { m } {
    \int_to_Roman:c {g__UWMad_Programming_API_#1_int}
}
\DeclareDocumentCommand \CounterToRoman { m } {
    \int_to_roman:c {g__UWMad_Programming_API_#1_int}
}
%    \end{macrocode}
%    \end{function}
%
%
%
%
%
%
%   \iffalse
%</Code>
%   \fi
%   \iffalse
%<*Code>
%   \fi
%
%
%   \UWModule{Layout And Styles}
%
%
%   \UWSubModule{Float Styles}
%
%   Make equation references of the form (\#).
%    \begin{macrocode}
\creflabelformat{equation}{#2#1#3}
%    \end{macrocode}
%
%   Default table style
%       \begin{macrocode}
\captionsetup [table] {
    format        = hang                ,
    labelsep      = colon               ,
    justification = justified           ,
    labelfont     = sc                  ,
    textfont      = sl                  ,
    font          = {normal,stretch=1.1},
    width         = 0.9\textwidth       ,
    position      = above               ,
    skip          = 0.50em
}
%    \end{macrocode}
%
%   Default figure style.
%    \begin{macrocode}
\captionsetup [figure] {
    format        = hang                ,
    labelsep      = colon               ,
    justification = justified           ,
    labelfont     = sc                  ,
    textfont      = sl                  ,
    font          = {normal,stretch=1.1},
    width         = 0.9\textwidth       ,
    position      = above               ,
    skip          = 0.5em
}
%    \end{macrocode}
%
%
%
%
%
%
%   \UWSubModule{Links}
%
%   Define a darker green than |green|.
%    \begin{macrocode}
\definecolor{UWMadGreen}{rgb}{0,0.7,0}
%    \end{macrocode}
%
%   Define the default colors for the (internal) links, cites, and URLs.
%    \begin{macrocode}
\tl_new:N  \l_UWMad_LayoutStyle_Color_Link_tl
\tl_set:Nn \l_UWMad_LayoutStyle_Color_Link_tl {blue}
\tl_new:N  \l_UWMad_LayoutStyle_Color_Cite_tl
\tl_set:Nn \l_UWMad_LayoutStyle_Color_Cite_tl {UWMadGreen}
\tl_new:N  \l_UWMad_LayoutStyle_Color_URL_tl
\tl_set:Nn \l_UWMad_LayoutStyle_Color_URL_tl  {violet}
%    \end{macrocode}
%
%   Define a new color and hyperlink defaults
%    \begin{macrocode}
\hypersetup {
    colorlinks         = true,
    linkcolor          = \l_UWMad_LayoutStyle_Color_Link_tl,
    citecolor          = \l_UWMad_LayoutStyle_Color_Cite_tl,
    urlcolor           = \l_UWMad_LayoutStyle_Color_URL_tl,
    pdfdisplaydoctitle = true,
    pdfview            = {FitH},
    pdfstartview       = {FitH},
    pdfpagelayout      = OneColumn,
    plainpages         = false,
    hypertexnames      = true,
    bookmarksopenlevel = 1,
    bookmarksopen      = true,
    unicode            = true
}
%    \end{macrocode}
%
%   Define a helper commands to redefine all of the \pkg{hyperref}
%   link colors using this class's color token lists.
%    \begin{macrocode}
\cs_new:Nn \UWMad_LayoutStyle_ResetLinkColors: {
    \hypersetup {
        colorlinks = true,
        linkcolor  = \l_UWMad_LayoutStyle_Color_Link_tl,
        citecolor  = \l_UWMad_LayoutStyle_Color_Cite_tl,
        urlcolor   = \l_UWMad_LayoutStyle_Color_URL_tl
    }
}
%    \end{macrocode}
%
%   Define user interfaces to setting link colors.
%    \begin{macrocode}
\DeclareDocumentCommand \MakeLinksTheseColors { m m m } {
    \tl_set:Nn \l_UWMad_LayoutStyle_Color_Link_tl {#1}
    \tl_set:Nn \l_UWMad_LayoutStyle_Color_Cite_tl {#2}
    \tl_set:Nn \l_UWMad_LayoutStyle_Color_URL_tl  {#3}
    \UWMad_LayoutStyle_ResetLinkColors:
}
\DeclareDocumentCommand \MakeLinksThisColor { m } {
    \tl_set:Nn \l_UWMad_LayoutStyle_Color_Link_tl {#1}
    \tl_set:Nn \l_UWMad_LayoutStyle_Color_Cite_tl {#1}
    \tl_set:Nn \l_UWMad_LayoutStyle_Color_URL_tl  {#1}
    \UWMad_LayoutStyle_ResetLinkColors:
}
%    \end{macrocode}
%
%   Define user interfaces to specialized color commands.
%    \begin{macrocode}
\DeclareDocumentCommand \MakeLinksBlack { } {
    \MakeLinksThisColor{black}
}
\DeclareDocumentCommand \MakeLinksBlue { } {
    \MakeLinksThisColor{blue}
}
\DeclareDocumentCommand \MakeLinksRed { } {
    \MakeLinksThisColor{red}
}
%    \end{macrocode}
%
%
%
%   \UWSubModule{Page Layout}
%
%   One inche magrins and letter (paper size) are set.
%    \begin{macrocode}
\geometry{
    includehead = true,
    margin      = 1.0in,
    paper       = letterpaper,
}
%    \end{macrocode}
%
%   Invoke `doublespacing' and set a warning in case any others invoke
%   the `not cool' commands according to the \UWMadShort{} Guidelines.
%    \begin{macrocode}
\doublespacing
\UWMad_Hook_Prepend:Nn \singlespacing {
    \__UWMad_FrontMatter_StyleWarning:n {
        University~guidelines~require~double-spacing.~
        If~this~is~for~temporary~use,~please~use~the~spacing~environment.
    }
}
\UWMad_Hook_Prepend:Nn \onehalfspacing {
    \__UWMad_FrontMatter_StyleWarning:n {
        University~guidelines~require~double-spacing.~
        If~this~is~for~temporary~use,~please~use~the~spacing~environment.
    }
}
%    \end{macrocode}
%
%   This setting puts the page numbers in the upper right-hand corner and
%   atleast one inch from the top and right sides of the page (per
%   the \UWMadShort{} guidelines).
%    \begin{macrocode}
\pagestyle{myheadings}
\setlength{\headsep}  {1.15em}
%    \end{macrocode}
%
%   Define user interface for defining different indentation styles.
%    \begin{macrocode}
\DeclareDocumentCommand \IndentParagraphs { } {
    \setlength{\parindent}{1.50em}
    \setlength{\parskip}  {0pt}

}
\DeclareDocumentCommand \PadParagraphs { } {
    \setlength{\parindent}{0pt}
    \setlength{\parskip}  {1em}
}
%    \end{macrocode}
%
%   Declare the default style of the \UWMadClass{} from the author:
%   no indentation with extra space between paragraphs.
%    \begin{macrocode}
\PadParagraphs
%    \end{macrocode}
%
%
%   \iffalse
%</Code>
%   \fi

%\iffalse
%<*Code>
%\fi
%
%
%   \UWModule{Sectioning}
%
%
%
%^^A  =========================================================================== %
%^^A                     Redefinition of Chapter Commands                         %
%^^A  =========================================================================== %
%   Prefix some code such that \cs{chapter} has the page number in the
%   upper right-hand corner and ensures that the  page numbering is
%   arabic before the first unnumbered chapter is used.
%    \begin{macrocode}
\UWMad_Hook_Prepend:Nn \@chapter {
    \thispagestyle{myheadings}
    \int_compare:nNnTF {\value{chapter}} = {0} {
        \pagenumbering{arabic}
    } { }
}
\UWMad_Hook_Prepend:Nn \@schapter {
    \thispagestyle{myheadings}
}
%    \end{macrocode}
%^^A  =========================================================================== %
%^^A                           New Appendix Command                               %
%^^A  =========================================================================== %
%
%   \UWSubModule{Appendix}
%   Here the \cs{appendix} command is redefined to act like the
%   \cs{chapter} command.
%   Before, \cs{appendix} simply changed the \texttt{chaptername} to
%   \enquote{Appendix}.
%
%   Define the appendix counter.
%    \begin{macrocode}
\int_new:N  \g__UWMad_Appendix_Counter_int
\int_set:Nn \g__UWMad_Appendix_Counter_int {0}
%    \end{macrocode}
%
%   This command is used when the first \cs{appendix} command
%   is used.  It sets the \texttt{chaptername} to \enquote{Appendix}
%   and sets the \cs{thechapter} to use the appendix counter
%   above.
%    \begin{macrocode}
\cs_new:Nn \__UWMad_Appendix_Initialize:{
    \par
    \setcounter{section}{0}
    \cs_gset_eq:NN \@chapapp \appendixname
    \cs_gset:Npn \thechapter {
        \int_to_Alph:n {
            \g__UWMad_Appendix_Counter_int
        }
    }
}
%    \end{macrocode}
%
%   Now, \cs{appendix} is undefined (to avoide a warning from
%   \pkg{xparse}) and redefined with standard \LaTeXe{}
%   sectioning arguments.
%    \begin{macrocode}
\cs_undefine:N \appendix
\DeclareDocumentCommand \appendix { s o m } {

    \int_compare:nNnTF {\g__UWMad_Appendix_Counter_int} = {0} {
        \__UWMad_Appendix_Initialize:
    } { }
    \int_gincr:N \g__UWMad_Appendix_Counter_int

    \IfBooleanTF { #1 } {
        \chapter*{#3}
    } {
        \IfNoValueTF { #2 } {
            \chapter[#3]{#3}
        } {
            \chapter[#2]{#3}
        }
    }
}
%    \end{macrocode}
%
%
%   \UWSubModule{Front Matter}
%
%   Front Matter commands (sometimes called preliminary pages)
%   are defined here.  These are the sections of the document
%   the precede the main body of the work.
%
%   Initialize a counter for the FrontMatter.
%    \begin{macrocode}%
\int_new:N \g__UWMad_FrontMatter_Counter_int
%    \end{macrocode}
%
%   This command enters the Front Matter with a given name
%   and section level into the Table of Contents.
%    \begin{macrocode}
\cs_new:Nn \__UWMad_FrontMatter_Register:nn {

    \int_compare:nNnTF {\g__UWMad_FrontMatter_Counter_int} = {0} {
        \pagenumbering{roman}
    } { }

    \int_gincr:N \g__UWMad_FrontMatter_Counter_int
    \addcontentsline
        {toc}
        {#1}
        {#2}
}
%    \end{macrocode}
%
%   These variables hold the default names of the Front Matter sections.
%    \begin{macrocode}
\tl_new:N   \g__UWMad_FrontMatter_Title_Dedications_tl
\tl_new:N   \g__UWMad_FrontMatter_Title_Acknowledgments_tl
\tl_new:N   \g__UWMad_FrontMatter_Title_Abstract_tl
\tl_new:N   \g__UWMad_FrontMatter_Title_UMIAbstract_tl
\tl_new:N   \g__UWMad_FrontMatter_Title_Preface_tl
%
\tl_gset:Nn \g__UWMad_FrontMatter_Title_Dedications_tl
    {Dedications}
\tl_gset:Nn \g__UWMad_FrontMatter_Title_Acknowledgments_tl
    {Acknowledgments}
\tl_gset:Nn \g__UWMad_FrontMatter_Title_Abstract_tl
    {Abstract}
\tl_gset:Nn \g__UWMad_FrontMatter_Title_UMIAbstract_tl
    {Abstract}
\tl_gset:Nn \g__UWMad_FrontMatter_Title_Preface_tl
    {Preface}
%    \end{macrocode}
%
%
%   First the \texttt{abstract} environment from the \LaTeX{} base class is
%   undefined, and the Front Matter commands as described in the User
%   Guide are defined.
%    \begin{macrocode}
\cs_undefine:N \abstract
\cs_undefine:N \endabstract

\DeclareDocumentCommand \FrontMatterSetSection { m m } {

    \tl_set_eq:Nc
        \l_tmpa_tl
        {g__UWMad_FrontMatter_Title_#2_tl}

    \IfNoValueTF { #1 } { } {
        \IfEmptyTF { #1 } { } {
            \tl_set:Nn \l_tmpa_tl {#1}
        }
    }

    \chapter*{\l_tmpa_tl}
    \__UWMad_FrontMatter_Register:nn {chapter} {
        \l_tmpa_tl
    }

}
\DeclareDocumentCommand \dedications { g } {
    \FrontMatterSetSection{#1}{Dedications}
}
\DeclareDocumentCommand \acknowledgments { g } {
    \FrontMatterSetSection{#1}{Acknowledgments}
}
\DeclareDocumentCommand \abstract { g } {
    \FrontMatterSetSection{#1}{Abstract}
}
\DeclareDocumentCommand \umiabstract { g } {
    \FrontMatterSetSection{#1}{Abstract}
}
\DeclareDocumentCommand \preface { g } {
    \FrontMatterSetSection{#1}{Preface}
}
%    \end{macrocode}
%
%
%   \UWSubModule{TOC Tweaks}
%
%   This section tweaks the Table of Contents, the List of Tables,
%   and the List of Figures commands to insert them into the
%   bookmark tree of the PDF.  Also, the commands for changing the
%   titles used for each of the commands' associated sections
%   are given.
%
%
%   First, store the original commands and then undefine them.
%    \begin{macrocode}
\cs_gset_eq:NN \TableOfContentsDefault \tableofcontents
\cs_gset_eq:NN \ListOfTablesDefault    \listoftables
\cs_gset_eq:NN \ListOfFiguresDefault   \listoffigures
\cs_undefine:N \tableofcontents
\cs_undefine:N \listoftables
\cs_undefine:N \listoffigures
%    \end{macrocode}
%
%   Now create token list variables to store the titles of the sections
%   and assign defaults.
%    \begin{macrocode}
\tl_new:N   \g__UWMad_TOC_Name_TOC_tl
\tl_new:N   \g__UWMad_TOC_Name_LOT_tl
\tl_new:N   \g__UWMad_TOC_Name_LOF_tl
\tl_gset:Nn \g__UWMad_TOC_Name_TOC_tl {Table~of~Contents}
\tl_gset:Nn \g__UWMad_TOC_Name_LOT_tl {List~of~Tables}
\tl_gset:Nn \g__UWMad_TOC_Name_LOF_tl {List~of~Figures}
%    \end{macrocode}
%
%   Define the new user-level commands.
%   Since these commands are technically Front Matter, they are
%   registered as such.
%    \begin{macrocode}
\DeclareDocumentCommand \tableofcontents { } {

    \tl_gset_eq:NN \contentsname \g__UWMad_TOC_Name_TOC_tl

    \group_begin:
        \setstretch{1.05}
        \phantomsection
        \ExplSyntaxOff
            \TableOfContentsDefault
        \ExplSyntaxOn
        \__UWMad_FrontMatter_Register:nn
            {chapter}
            {\contentsname}
        \clearpage
    \group_end:
}
\DeclareDocumentCommand \listoftables { } {

    \cs_set_eq:NN \listtablename \g__UWMad_TOC_Name_LOT_tl

    \group_begin:
        \setstretch{1.05}
        \ExplSyntaxOff
            \ListOfTablesDefault
        \ExplSyntaxOn
        \__UWMad_FrontMatter_Register:nn
            {chapter}
            {\listtablename}
        \clearpage
    \group_end:
}
\DeclareDocumentCommand \listoffigures { } {

    \cs_set_eq:NN \listfigurename \g__UWMad_TOC_Name_LOF_tl

    \group_begin:
        \setstretch{1.05}
        \ExplSyntaxOff
            \ListOfFiguresDefault
        \ExplSyntaxOn
        \__UWMad_FrontMatter_Register:nn
            {chapter}
            {\listfigurename}
        \clearpage
    \group_end:
}
%    \end{macrocode}
%
%   Camel-cased aliases.
%    \begin{macrocode}
\cs_set_eq:NN \TableOfContents \tableofcontents
\cs_set_eq:NN \ListOfTables    \listoftables
\cs_set_eq:NN \ListOfFigures   \listoffigures
%    \end{macrocode}
%
%   User-level commands to change the default names.
%    \begin{macrocode}
\DeclareDocumentCommand \TableOfContentsName { m } {
    \tl_gset:Nn \g__UWMad_TOC_Name_TOC_tl {#1}
}
\DeclareDocumentCommand \ListOfTablesName { m } {
    \tl_gset:Nn \g__UWMad_TOC_Name_LOT_tl {#1}
}
\DeclareDocumentCommand \ListOfFiguresName { m } {
    \tl_gset:Nn \g__UWMad_TOC_Name_LOF_tl {#1}
}
%    \end{macrocode}
%
%   \UWSubModule{Section-Level Commands}
%
%   These commands are used internally when needing to check if a
%   user-supplied \texttt{section} is a \LaTeXe{}-defined section and
%   also easily acquired/use the relationships among section levels
%   when needed.
%
%   These variables map a \texttt{section} to a level number and also serve
%   to define the existence of the level.
%    \begin{macrocode}
\tl_const:Nn \c__UWMad_SectionsLevel_part_tl            {-1}
\tl_const:Nn \c__UWMad_SectionsLevel_chapter_tl         {0}
\tl_const:Nn \c__UWMad_SectionsLevel_section_tl         {1}
\tl_const:Nn \c__UWMad_SectionsLevel_subsection_tl      {2}
\tl_const:Nn \c__UWMad_SectionsLevel_subsubsection_tl   {3}
\tl_const:Nn \c__UWMad_SectionsLevel_paragraph_tl       {4}
\tl_const:Nn \c__UWMad_SectionsLevel_subparagraph_tl    {5}
%    \end{macrocode}
%
%   Define a message to warn about an undefined section
%   and associated command to check if a section exists.
%    \begin{macrocode}
\msg_new:nnn { UWMadThesis } { Sectioning / UndefinedSection } {
    Undefined~section~'#1'~used.
}
\cs_new:Nn \UWMad_IfSectionExists:nT {
    \tl_if_exist:cTF {c__UWMad_SectionsLevel_ #1 _tl} {
        #2
    } {
        \msg_error:nnn
            { UWMadThesis }
            { Sectioning / UndefinedSection }
            {#1}
    }
}
%    \end{macrocode}
%
%   Variables that map a level number to a section.
%    \begin{macrocode}
\tl_const:cn {c__UWMad_LevelsSection_-1_tl} {part}
\tl_const:cn {c__UWMad_LevelsSection_ 0_tl} {chapter}
\tl_const:cn {c__UWMad_LevelsSection_ 1_tl} {section}
\tl_const:cn {c__UWMad_LevelsSection_ 2_tl} {subsection}
\tl_const:cn {c__UWMad_LevelsSection_ 3_tl} {subsubsection}
\tl_const:cn {c__UWMad_LevelsSection_ 4_tl} {paragraph}
\tl_const:cn {c__UWMad_LevelsSection_ 5_tl} {subparagraph}
%    \end{macrocode}
%
%   Variables that map a section to it's next lower one.
%    \begin{macrocode}
\tl_const:Nn \c__UWMad_NextSection_part_tl            {chapter}
\tl_const:Nn \c__UWMad_NextSection_chapter_tl         {section}
\tl_const:Nn \c__UWMad_NextSection_section_tl         {subsection}
\tl_const:Nn \c__UWMad_NextSection_subsection_tl      {subsubsection}
\tl_const:Nn \c__UWMad_NextSection_subsubsection_tl   {paragraph}
\tl_const:Nn \c__UWMad_NextSection_paragraph_tl       {subparagraph}
%    \end{macrocode}
%
%   Variables that map a section to it's next higher one.
%    \begin{macrocode}
\tl_const:Nn \c__UWMad_PreviousSection_chapter_tl         {part}
\tl_const:Nn \c__UWMad_PreviousSection_section_tl         {chapter}
\tl_const:Nn \c__UWMad_PreviousSection_subsection_tl      {section}
\tl_const:Nn \c__UWMad_PreviousSection_subsubsection_tl   {subsection}
\tl_const:Nn \c__UWMad_PreviousSection_paragraph_tl       {subsubsection}
\tl_const:Nn \c__UWMad_PreviousSection_subparagraph_tl    {paragraph}
%    \end{macrocode}
%
%   Given a section, acquire its level number.
%    \begin{macrocode}
\cs_new:Nn \UWMad_SectionToLevel:n {
    \UWMad_IfSectionExists:nT {#1} {
        \tl_use:c {c__UWMad_SectionsLevel_ #1 _tl}
    }
}
%    \end{macrocode}
%
%   Given a level number, acquire its section.
%    \begin{macrocode}
\cs_new:Nn \UWMad_LevelToSection:n {
    \UWMad_IfSectionExists:nT {#1} {
        \tl_use:c {c__UWMad_LevelsSection_ #1 _tl}
    }
}
%    \end{macrocode}
%
%   Given a section, acquire its next lower one.
%    \begin{macrocode}
\cs_new:Nn \UWMad_NextSection:n {
    \UWMad_IfSectionExists:nT {#1} {
        \tl_use:c {c__UWMad_NextSection_ #1 _tl}
    }
}
%    \end{macrocode}
%
%   Given a section, acquire its next higher one.
%    \begin{macrocode}
\cs_new:Nn \UWMad_PreviousSection:n {
    \UWMad_IfSectionExists:nT {#1} {
        \tl_use:c {c__UWMad_PreviousSection_ #1 _tl}
    }
}
%    \end{macrocode}
%
%
%   \iffalse
%</Code>
%   \fi
%   \iffalse
%<*Code>
%   \fi
%
%
%   \UWModule{Math}
%
%
%
%   We default the \cs{frac} command to a display style for all display
%   environments.
%    \begin{macrocode}
\tex_everydisplay:D \exp_after:wN {
    \tex_the:D \tex_everydisplay:D
    \cs_set_eq:NN \frac \dfrac
}
%    \end{macrocode}
%
%
%
%
%   \UWSubModule{Derivative Commands}
%   Define the token list variables for the three supported derivative types.
%    \begin{macrocode}
\tl_new:N   \g_UWMad_Math_derivSymbol_tl
\tl_gset:Nn \g_UWMad_Math_derivSymbol_tl   {\mathrm{d}}
\tl_new:N   \g_UWMad_Math_pderivSymbol_tl
\tl_gset:Nn \g_UWMad_Math_pderivSymbol_tl  {\partial}
\tl_new:N   \g_UWMad_Math_tderivSymbol_tl
\tl_gset:Nn \g_UWMad_Math_tderivSymbol_tl  {\mathrm{D}}
\tl_new:N   \g_UWMad_Math_DelimiterDefaultLeft_tl
\tl_gset:Nn \g_UWMad_Math_DelimiterDefaultLeft_tl  {[}
\tl_new:N   \g_UWMad_Math_DelimiterDefaultRight_tl
\tl_gset:Nn \g_UWMad_Math_DelimiterDefaultRight_tl {]}
\tl_new:N   \l_UWMad_Math_DelimiterLeft_tl
\tl_new:N   \l_UWMad_Math_DelimiterRight_tl
%    \end{macrocode}
%
%
%   Define the user interface accessors.
%    \begin{macrocode}
\DeclareDocumentCommand \derivSymbol { } {
    \g_UWMad_Math_derivSymbol_tl
}
\DeclareDocumentCommand \pderivSymbol { } {
    \g_UWMad_Math_pderivSymbol_tl
}
\DeclareDocumentCommand \tderivSymbol { } {
    \g_UWMad_Math_tderivSymbol_tl
}
%    \end{macrocode}
%
%
%   Define the user interface local mutators.
%    \begin{macrocode}
\DeclareDocumentCommand \derivSymbolChange { m } {
    \tl_set:Nn \g_UWMad_Math_derivSymbol_tl {#1}
}
\DeclareDocumentCommand \pderivSymbolChange { m } {
    \tl_set:Nn \g_UWMad_Math_pderivSymbol_tl {#1}
}
\DeclareDocumentCommand \tderivSymbolChange { m } {
    \tl_set:Nn \g_UWMad_Math_tderivSymbol_tl {#1}
}
%    \end{macrocode}
%
%
%   Define the user interface global mutators.
%    \begin{macrocode}
\DeclareDocumentCommand \derivSymbolChangeDefault { m } {
    \tl_gset:Nn \g_UWMad_Math_derivSymbol_tl {#1}
}
\DeclareDocumentCommand \pderivSymbolChangeDefault { m } {
    \tl_gset:Nn \g_UWMad_Math_pderivSymbol_tl {#1}
}
\DeclareDocumentCommand \tderivSymbolChangeDefault { m } {
    \tl_gset:Nn \g_UWMad_Math_tderivSymbol_tl {#1}
}
%    \end{macrocode}
%
%
%   Define the \cs{left} and \cs{right} delimiter global mutators.
%    \begin{macrocode}
\DeclareDocumentCommand \DelimiterChangeDefault { m m } {
    \tl_gset:Nn  \g_UWMad_Math_DelimiterDefaultLeft_tl  {#1}
    \tl_gset:Nn  \g_UWMad_Math_DelimiterDefaultRight_tl {#2}
}
%    \end{macrocode}
%
%
%   Define the generic regular and big derivative functions.
%    \begin{macrocode}
\DeclareDocumentCommand \DerivativeGeneral { +m +m m m } {
    \frac{ #4^{#3} #1      }
         { #4      #2^{#3} }
}
\DeclareDocumentCommand \DerivativeGeneralBig { +m +m m m m m} {

    \IfNoValueTF {#5} {
        \tl_set_eq:NN
            \l_UWMad_Math_DelimiterLeft_tl
            \g_UWMad_Math_DelimiterDefaultLeft_tl
    } {
        \tl_set:Nn \l_UWMad_Math_DelimiterLeft_tl {#5}
    }

    \IfNoValueTF {#6} {
        \tl_set_eq:NN
            \l_UWMad_Math_DelimiterRight_tl
            \g_UWMad_Math_DelimiterDefaultRight_tl
    } {
        \tl_set:Nn \l_UWMad_Math_DelimiterRight_tl {#6}
    }

    \frac{ #4^{#3}    }
         { #4 #2^{#3} }
    \!\!
    \left\l_UWMad_Math_DelimiterLeft_tl
        #1
    \right\l_UWMad_Math_DelimiterRight_tl
}
%    \end{macrocode}
%
%
%   Define the three supported derivative types' small forms.
%    \begin{macrocode}
\DeclareDocumentCommand \deriv { +m +m G{} } {
    \DerivativeGeneral
        {#1}{#2}{#3}{\derivSymbol}
}
\DeclareDocumentCommand \pderiv { +m +m G{} } {
    \DerivativeGeneral
        {#1}{#2}{#3}{\pderivSymbol}
}
\DeclareDocumentCommand \tderiv { +m +m G{} } {
    \DerivativeGeneral
        {#1}{#2}{#3}{\tderivSymbol}
}
%    \end{macrocode}
%
%
%   Define the three supported derivative types' big forms.
%    \begin{macrocode}
\DeclareDocumentCommand \derivbig { o +m o +m G{} } {
    \DerivativeGeneralBig
        {#2}{#4}{#5}{\derivSymbol}{#1}{#3}
}
\DeclareDocumentCommand \pderivbig { o +m o +m G{} } {
    \DerivativeGeneralBig
        {#2}{#4}{#5}{\pderivSymbol}{#1}{#3}
}
\DeclareDocumentCommand \tderivbig { o +m o +m G{} } {
    \DerivativeGeneralBig
        {#2}{#4}{#5}{\tderivSymbol}{#1}{#3}
}
%    \end{macrocode}
%
%   \UWSubModule{Operators and Functions}
%   Define all of the operators and function described in the user manual.
%    \begin{macrocode}
\DeclareMathOperator*{\Sup}    {Sup}
\DeclareMathOperator*{\Inf}    {Inf}
\DeclareMathOperator*{\Lim}    {Lim}
\DeclareMathOperator*{\Min}    {Min}
\DeclareMathOperator*{\Max}    {Max}
\DeclareMathOperator*{\ArgMin} {ArgMin}
\DeclareMathOperator*{\ArgMax} {ArgMax}
\DeclareMathOperator{\Abs}     {Abs}
\DeclareMathOperator{\Ln}      {Ln}
\DeclareMathOperator{\Log}     {Log}
\DeclareMathOperator{\Exp}     {Exp}
\DeclareMathOperator{\Cos}     {Cos}
\DeclareMathOperator{\Sin}     {Sin}
\DeclareMathOperator{\Tan}     {Tan}
\DeclareMathOperator{\Sec}     {Sec}
\DeclareMathOperator{\Csc}     {Csc}
\DeclareMathOperator{\Cot}     {Cot}
\DeclareMathOperator{\Cosh}    {Cosh}
\DeclareMathOperator{\Sinh}    {Sinh}
\DeclareMathOperator{\Tanh}    {Tanh}
\DeclareMathOperator{\Sech}    {Sech}
\DeclareMathOperator{\Csch}    {Csch}
\DeclareMathOperator{\Coth}    {Coth}
\DeclareMathOperator{\ArcCos}  {ArcCos}
\DeclareMathOperator{\ArcSin}  {ArcSin}
\DeclareMathOperator{\ArcTan}  {ArcTan}
\DeclareMathOperator{\ArcSec}  {ArcSec}
\DeclareMathOperator{\ArcCsc}  {ArcCsc}
\DeclareMathOperator{\ArcCot}  {ArcCot}
\DeclareMathOperator{\ArcCosh} {ArcCosh}
\DeclareMathOperator{\ArcSinh} {ArcSinh}
\DeclareMathOperator{\ArcTanh} {ArcTanh}
\DeclareMathOperator{\ArcSech} {ArcSech}
\DeclareMathOperator{\ArcCsch} {ArcCsch}
\DeclareMathOperator{\ArcCoth} {ArcCoth}
%    \end{macrocode}
%
%   \UWSubModule{Miscallaneous Functions}
%
%
%   Define the root function that has a tail.
%    \begin{macrocode}
\cs_new:Nn \UWMad_Math_RootWithTail:nn {

    \hbox_set:Nn \l_tmpa_box {
        $
            \mathchoice
                {\root #1 \of {#2\:\!}}
                {\root #1 \of {#2\:\!}}
                {\root #1 \of {#2\:\!}}
                {\root #1 \of {#2\:\!}}
        $
    }
    %
    \dim_set:Nn \l_tmpa_dim {\box_ht:N \l_tmpa_box}
    \dim_set:Nn \l_tmpb_dim {0.8\l_tmpa_dim}
    %
    \hbox_set:Nn \l_tmpb_box {
        \tex_vrule:D height \l_tmpa_dim depth -\l_tmpb_dim
    }
    %
    \box_use:N \l_tmpa_box
    \box_move_down:nn {0.40pt}{\box_use:N \l_tmpb_box}
}
\DeclareDocumentCommand \Sqrt { O{} m } {
    \UWMad_Math_RootWithTail:nn{#1}{#2}
}
%    \end{macrocode}
%
%
%   User interface math mode check.
%    \begin{macrocode}
\DeclareExpandableDocumentCommand \IfMathModeTF { +m +m } {
    \mode_if_math:TF {
        #1
    }{
        $#2$
    }
}
%    \end{macrocode}
%
%
%   Undefine the \cs{sups} commands defined by the IPA package.
%    \begin{macrocode}
\cs_gset_eq:NN \supsipa \sups
\cs_undefine:N \sups
%    \end{macrocode}
%
%
%   Then define the \cs{subs}, \cs{sups}, and \cs{subsups} commands as
%   described in the manual.
%    \begin{macrocode}
\ExplSyntaxOff
    \DeclareDocumentCommand \subs { O{} +m } {%
        \IfMathModeTF{%
            _{\!\!\:#1\text{\scriptsize #2}}%
        }{%
            _{\!#1\text{\scriptsize #2}}%
        }%
    }%
    \DeclareDocumentCommand \sups { O{} +m } {%
        \IfMathModeTF{%
            ^{#1\text{\scriptsize #2}}%
        }{%
            ^{#1\text{\scriptsize #2}}%
        }%
    }%
    \DeclareDocumentCommand \subsups { O{} +m O{} +m } {%
        \IfMathModeTF{%
            _{#1\text{\scriptsize #2}}^{\!\!\:#3\text{\scriptsize #4}}%
        }{%
            _{#1\text{\scriptsize #2}}^{\!\!\!#3\text{\scriptsize #4}}%
        }%
    }%
\ExplSyntaxOn
\cs_gset_eq:NN \supsubs \subsups
%    \end{macrocode}
%
%
%   The one-over functions discussed in the manual.
%    \begin{macrocode}
\DeclareDocumentCommand \OneOver { +m } {
    \frac{1}{#1}
}
\DeclareDocumentCommand \oneo { +m } {
    \OneOver{#1}
}
%    \end{macrocode}
%
%
%   The non-math `d' discussed in the manual.
%    \begin{macrocode}
\DeclareDocumentCommand \dd { m } {
    \mathrm{d}{#1}
}
%    \end{macrocode}
%
%
%   The prime commands discussed in the manual.
%    \begin{macrocode}
\DeclareDocumentCommand \dprime { } {
    {\prime\prime}
}
\DeclareDocumentCommand \tprime { } {
    {\prime\prime\prime}
}
%    \end{macrocode}
%
%
%   Two commands that were necessary for proper typesetting.
%    \begin{macrocode}
\DeclareDocumentCommand \LessThan        { } {<}
\DeclareDocumentCommand \GreaterThanThan { } {>}
%
%    \end{macrocode}
%   \iffalse
%</Code>
%   \fi
%   \iffalse
%<*Code>
%   \fi
%
%
%  \UWModule{ListOf}
%
%   The ListOf Module is a collection of commands that enables the easy
%   creation and typsetting of Lists.
%
%   Lists are taken to be any collection of entries that is to be typeset
%   with a particular style.  For example, a simple Nomenclature could be
%   considered a list of (symbol, description) entries to be typeset with a
%   fixed style for all entires.  The \texttt{ListOf} commands create a system
%   specifically for this scenario.
%
%   Of course, as the commands description will show, lists can be much more
%   complicated that two items.  For the \texttt{ListOf} system to function, an
%   author really only needs to define the \texttt{ListOf}, create a command to push
%   (enqueue) entries on to the \texttt{ListOf} queue, and at some point tell the
%   \texttt{ListOf} to typeset the entries it has stored (if display of the content
%   is desired).
%
%
%
%   \begin{macro}{\UWMad_ListOf_Define:n}
%   Define a new \texttt{ListOf} with \marg{ID}. This command creates the
%   commands to store the section commands and title for each group,
%   the booleans to indicate if the sections should be numbered and
%   if the sections should be included in the table of contentst
%   (regardless of numbering), a hash to hold of the user-defined
%   hooks for the \texttt{ListOf}, and a queue to store the entries for
%   typesetting.
%
%    \begin{macrocode}
\cs_new:Nn \UWMad_ListOf_Define:n {
    \tl_const:cn {c__UWMad_ListOf#1_IsDefined_tl}{}
%
    \tl_new:c {g__UWMad_ListOf#1_Section_Main_tl}
    \tl_new:c {g__UWMad_ListOf#1_Section_Group_tl}
    \tl_new:c {g__UWMad_ListOf#1_Section_Subgroup_tl}
%
    \tl_new:c {g__UWMad_ListOf#1_Title_Main_tl}
    \tl_new:c {g__UWMad_ListOf#1_Title_Group_tl}
    \tl_new:c {g__UWMad_ListOf#1_Title_Subgroup_tl}
%
    \bool_new:c       {g__UWMad_ListOf#1_ClearAfterPrint_bool}
    \bool_gset_true:c {g__UWMad_ListOf#1_ClearAfterPrint_bool}
    \bool_new:c       {g__UWMad_ListOf#1_IsNumbered_bool}
    \bool_gset_true:c {g__UWMad_ListOf#1_IsNumbered_bool}
    \bool_new:c       {g__UWMad_ListOf#1_IncludeInTOC_bool}
    \bool_gset_true:c {g__UWMad_ListOf#1_IncludeInTOC_bool}
    \UWMad_Queue_Define:n               {g__ListOf#1_EntryQueue}
    \UWMad_Hash_Define:n                {g__ListOf#1_Hook}
}
%    \end{macrocode}
%   \end{macro}
%
%
%
%   \begin{macro}{\UWMad_ListOf_Delete:n}
%   Simply undefines all of the commands created in the \texttt{Define} command
%   above for the given \marg{ID}.
%
%    \begin{macrocode}
\cs_new:Nn \UWMad_ListOf_Delete:n {
    \cs_undefine:c {c__UWMad_ListOf#1_IsDefined_tl}
    \cs_undefine:c {g__UWMad_ListOf#1_Section_Main_tl}
    \cs_undefine:c {g__UWMad_ListOf#1_Section_Group_tl}
    \cs_undefine:c {g__UWMad_ListOf#1_Section_Subgroup_tl}
    \cs_undefine:c {g__UWMad_ListOf#1_Title_Main_tl}
    \cs_undefine:c {g__UWMad_ListOf#1_Title_Group_tl}
    \cs_undefine:c {g__UWMad_ListOf#1_Title_Subgroup_tl}
    \cs_undefine:c {g__UWMad_ListOf#1_ClearAfterPrint_bool}
    \cs_undefine:c {g__UWMad_ListOf#1_IsNumbered_bool}
    \cs_undefine:c {g__UWMad_ListOf#1_IncludeInTOC_bool}
    \UWMad_Queue_Delete:n   {g__ListOf#1_EntryQueue}
    \UWMad_Hash_Delete:n    {g__ListOf#1_Hook}
}
%    \end{macrocode}
%   \end{macro}
%
%
%
%   \begin{macro}{\UWMad_ListOf_IfDefined:nT}
%   Checks to see if a \texttt{ListOf} with \marg{ID} has been created and
%   errors if not.
%
%    \begin{macrocode}
\cs_new:Nn \UWMad_ListOf_IfDefined:nT {
    \__UWMad_IfDefined:nnnnT
        {c__UWMad_ListOf}
        {#1}
        {_IsDefined_tl}
        {ListOf}
        {#2}
}
%    \end{macrocode}
%   \end{macro}
%
%
%
%   \begin{macro}{
%   \UWMad_ListOf_MakeNumbered:n,\UWMad_ListOf_MakeNotNumbered:n}
%   Makes the current section of the \texttt{ListOf} with \marg{ID} numbered
%   or unnumbered (i.e., a star version).
%
%    \begin{macrocode}
\cs_new:Nn \UWMad_ListOf_MakeNumbered:n {
    \UWMad_ListOf_IfDefined:nT {#1} {
        \bool_set_true:c {g__UWMad_ListOf#1_IsNumbered_bool}
    }
}
\cs_new:Nn \UWMad_ListOf_MakeNotNumbered:n {
    \UWMad_ListOf_IfDefined:nT {#1} {
        \bool_set_false:c {g__UWMad_ListOf#1_IsNumbered_bool}
    }
}
%    \end{macrocode}
%   \end{macro}
%
%
%
%   \begin{macro}{\UWMad_ListOf_IfNumbered:nTF}
%   Branches to \marg{True Code} or \marg{False Code} depending on whether
%   the \texttt{ListOf} with \marg{ID} is numbered or not.
%
%    \begin{macrocode}
\cs_new:Nn \UWMad_ListOf_IfNumbered:nTF {
    \UWMad_ListOf_IfDefined:nT {#1} {
        \bool_if:cTF {g__UWMad_ListOf#1_IsNumbered_bool} {
            #2
        }{
            #3
        }
    }
}
%    \end{macrocode}
%   \end{macro}
%
%
%
%   \begin{macro}{
%   \UWMad_ListOf_IncludeInTOC:n,\UWMad_ListOf_DoNotIncludeInTOC:n}
%   Makes the current section of the \texttt{ListOf} with \marg{ID} appear in
%   the Table of Contents (TOC)  or not, regardless of if it is
%   numbered/unnumbered.
%
%    \begin{macrocode}
\cs_new:Nn \UWMad_ListOf_IncludeInTOC:n {
    \UWMad_ListOf_IfDefined:nT {#1} {
        \bool_set_true:c {c__UWMad_ListOf#1_IncludeInTOC_bool}
    }
}
\cs_new:Nn \UWMad_ListOf_DoNotIncludeInTOC:n {
    \UWMad_ListOf_IfDefined:nT {#1} {
        \bool_set_false:c {c__UWMad_ListOf#1_IncludeInTOC_bool}
    }
}
%    \end{macrocode}
%   \end{macro}
%
%
%
%   \begin{macro}{\UWMad_ListOf_IfIncludeInTOC:n}
%   Branches to \marg{True Code} or \marg{False Code} depending on whether
%   the \texttt{ListOf} with \marg{ID} is to be included or not.
%
%    \begin{macrocode}
\cs_new:Nn \UWMad_ListOf_IfIncludeInTOC:nTF {
    \UWMad_ListOf_IfDefined:nT {#1} {
        \bool_if:cTF {c__UWMad_ListOf#1_IncludeInTOC_bool} {
            #2
        }{
            #3
        }
    }
}
%    \end{macrocode}
%   \end{macro}
%
%
%
%   \begin{function}{
%       \UWMad_ListOf_SetTitle_Main:nn,
%       \UWMad_ListOf_SetTitle_Group:nn,
%       \UWMad_ListOf_SetTitle_Subgroup:nn}
%   \begin{syntax}
%       \cs{UWMad_ListOf_SetTitle_Main:nn}\Arg{ID}\Arg{Title}
%       \cs{UWMad_ListOf_SetTitle_Group:nn}\Arg{ID}\Arg{Title}
%       \cs{UWMad_ListOf_SetTitle_Subgroup:nn}\Arg{ID}\Arg{Title}
%   \end{syntax}
%   Sets the value of the title of the sections to \marg{Title} for the
%   \texttt{ListOf} with \marg{ID}
%
%    \begin{macrocode}
\cs_new:Nn \UWMad_ListOf_SetTitle_Main:nn {
    \UWMad_ListOf_IfDefined:nT {#1} {
        \tl_set:cn {g__UWMad_ListOf#1_Title_Main_tl}{#2}
    }
}
\cs_new:Nn \UWMad_ListOf_SetTitle_Group:nn {
    \UWMad_ListOf_IfDefined:nT {#1} {
        \tl_set:cn {g__UWMad_ListOf#1_Title_Group_tl}{#2}
    }
}
\cs_new:Nn \UWMad_ListOf_SetTitle_Subgroup:nn {
    \UWMad_ListOf_IfDefined:nT {#1} {
        \tl_set:cn {g__UWMad_ListOf#1_Title_Subgroup_tl}{#2}
    }
}
%    \end{macrocode}
%   \end{function}
%
%
%
%   \begin{function}{
%       \UWMad_ListOf_GetTitle_Main:nn,
%       \UWMad_ListOf_GetTitle_Group:nn,
%       \UWMad_ListOf_GetTitle_Subgroup:nn}
%   \begin{syntax}
%       \cs{UWMad_ListOf_GetTitle_Main:n}    \Arg{ID}
%       \cs{UWMad_ListOf_GetTitle_Group:n}   \Arg{ID}
%       \cs{UWMad_ListOf_GetTitle_Subgroup:n}\Arg{ID}
%   \end{syntax}
%   Retrieces the value of the title of the section for the
%   \texttt{ListOf} with \marg{ID}.
%
%    \begin{macrocode}
\cs_new:Nn \UWMad_ListOf_GetTitle_Main:n {
    \UWMad_ListOf_IfDefined:nT {#1} {
        \tl_use:c {g__UWMad_ListOf#1_Title_Main_tl}
    }
}
\cs_new:Nn \UWMad_ListOf_GetTitle_Group:n {
    \UWMad_ListOf_IfDefined:nT {#1} {
        \tl_use:c {g__UWMad_ListOf#1_Title_Group_tl}
    }
}
\cs_new:Nn \UWMad_ListOf_GetTitle_Subgroup:n {
    \UWMad_ListOf_IfDefined:nT {#1} {
        \tl_use:c {g__UWMad_ListOf#1_Title_Subgroup_tl}
    }
}
%    \end{macrocode}
%   \end{function}
%
%
%
%   \begin{function}{
%       \UWMad_ListOf_SetSection_Main:nn,
%       \UWMad_ListOf_SetSection_Group:nn,
%       \UWMad_ListOf_SetSection_Subgroup:nn}
%   \begin{syntax}
%       \cs{UWMad_ListOf_SetSection_Main:nn}\Arg{ID}\Arg{Section}
%       \cs{UWMad_ListOf_SetSection_Group:nn}\Arg{ID}\Arg{Section}
%       \cs{UWMad_ListOf_SetSection_Subgroup:nn}\Arg{ID}\Arg{Section}
%   \end{syntax}
%   Sets the value of the section level and (currently) the sectioning
%   command for a particular group to \Arg{Section} of the \texttt{ListOf}
%   with \Arg{ID}.
%
%    \begin{macrocode}
\cs_new:Nn \UWMad_ListOf_SetSection_Main:nn {
    \UWMad_ListOf_IfDefined:nT {#1} {
        \tl_set:cn {g__UWMad_ListOf#1_Section_Main_tl}{#2}
    }
}
\cs_new:Nn \UWMad_ListOf_SetSection_Group:nn {
    \UWMad_ListOf_IfDefined:nT {#1} {
        \tl_set:cn {g__UWMad_ListOf#1_Section_Group_tl}{#2}
    }
}
\cs_new:Nn \UWMad_ListOf_SetSection_Subgroup:nn {
    \UWMad_ListOf_IfDefined:nT {#1} {
        \tl_set:cn {g__UWMad_ListOf#1_Section_Subgroup_tl}{#2}
    }
}
%    \end{macrocode}
%   \end{function}
%
%
%
%   \begin{function}{
%       \UWMad_ListOf_GetSection_Main:n,
%       \UWMad_ListOf_GetSection_Group:n,
%       \UWMad_ListOf_GetSection_Subgroup:n}
%   \begin{syntax}
%       \cs{UWMad_ListOf_GetSection_Main:n}\Arg{ID}
%       \cs{UWMad_ListOf_GetSection_Group:n}\Arg{ID}
%       \cs{UWMad_ListOf_GetSection_Subgroup:n}\Arg{ID}
%   \end{syntax}
%   Gets the value of the section level for a particular group of the
%   \texttt{ListOf} with \Arg{ID}.
%
%    \begin{macrocode}
\cs_new:Nn \UWMad_ListOf_GetSection_Main:n {
    \UWMad_ListOf_IfDefined:nT {#1} {
        \tl_use:c {g__UWMad_ListOf#1_Section_Main_tl}
    }
}
\cs_new:Nn \UWMad_ListOf_GetSection_Group:n {
    \UWMad_ListOf_IfDefined:nT {#1} {
        \tl_use:c {g__UWMad_ListOf#1_Section_Group_tl}
    }
}
\cs_new:Nn \UWMad_ListOf_GetSection_Subgroup:n {
    \UWMad_ListOf_IfDefined:nT {#1} {
        \tl_use:c {g__UWMad_ListOf#1_Section_Subgroup_tl}
    }
}
%    \end{macrocode}
%   \end{function}
%
%
%
%   \begin{function}{\UWMad_ListOf_SetHook:nnn}
%   \begin{syntax}
%       \cs{UWMad_ListOf_SetHook:nnn}\Arg{ID}\Arg{Hook name}\Arg{Hook code}
%   \end{syntax}
%   Sets \Arg{Hook name} to \Arg{Hook code} for the \texttt{ListOf}
%   with \Arg{ID}.
%   The current hooks used are: \texttt{PrePush}, \texttt{PostPush},
%   \texttt{PrePrint}, and \texttt{PostPrint}.
%
%    \begin{macrocode}
\cs_new:Nn \UWMad_ListOf_SetHook:nnn {
    \UWMad_Hash_Set:nnn{g__ListOf#1_Hook}{#2}{#3}
}
%    \end{macrocode}
%   \end{function}
%
%
%
%   \begin{function}{\UWMad_ListOf_PushEntry:nn}
%   \begin{syntax}
%       \cs{UWMad_ListOf_PushEntry:nn} \Arg{ID}\Arg{Entry}
%   \end{syntax}
%   Pushes \Arg{Entry} on to the entry queue of the \texttt{ListOf} with \Arg{ID}.
%
%    \begin{macrocode}
\cs_new:Nn \UWMad_ListOf_PushEntry:nn {
    \UWMad_Hash_Get:nn   {g__ListOf#1_Hook}{PrePush}
    \UWMad_Queue_Push:nn {g__ListOf#1_EntryQueue}{#2}
    \UWMad_Hash_Get:nn   {g__ListOf#1_Hook}{PostPush}
}
%    \end{macrocode}
%   \end{function}
%
%
%
%   \begin{function}{\UWMad_ListOf_PrintEntries:n}
%   \begin{syntax}
%       \cs{UWMad_ListOf_PrintEntries:n}\Arg{ID}
%   \end{syntax}
%   Prints all entries currently in the \texttt{ListOf} queue with \marg{ID} and
%   clears the queue.  The \texttt{PrePrint} and \texttt{PostPrint} hooks
%   are also called here.
%
%    \begin{macrocode}
\cs_new:Nn \UWMad_ListOf_PrintEntries:n {
    \UWMad_Hash_Get:nn   {g__ListOf#1_Hook}{PrePrint}
    \UWMad_Queue_Walk:nn {g__ListOf#1_EntryQueue}{##1}
    \UWMad_Queue_Clear:n {g__ListOf#1_EntryQueue}
    \UWMad_Hash_Get:nn   {g__ListOf#1_Hook}{PostPrint}
}
%    \end{macrocode}
%   \end{function}
%
%
%
%   \begin{function}{\UWMad_ListOf_PrintTitle:nn}
%   \begin{syntax}
%       \cs{UWMad_ListOf_PrintTitle:nn}\Arg{ID}\Arg{Group}
%   \end{syntax}
%   Prints the title for the \Arg{Group} of the \texttt{ListOf} with
%   \Arg{ID} at the section indicated by its associated token
%   list.
%   Numbering and table of contents adding is done according to the
%   current values of their respective booleans.
%
%    \begin{macrocode}
\cs_new:Nn \__UWMad_ListOf_CurrentSectioningCommmand:n {}
\cs_new:Nn \UWMad_ListOf_PrintTitle:nn {

    \cs_set_eq:Nc
        \__UWMad_ListOf_CurrentSectioningCommmand:n
        {\tl_use:c{g__UWMad_ListOf#1_Section_#2_tl}}

    \UWMad_ListOf_IfNumbered:nTF {#1} {

        \tl_if_eq:nnTF {#2} {Main} {
            \UWMad_ListOf_IfIncludeInTOC:nTF {#1} { } {
                \int_set_eq:NN \l_tmpa_int \c@tocdepth
                \setcounter{tocdepth}{-1}
            }
        } {
            \int_set_eq:NN \l_tmpa_int \c@tocdepth
            \setcounter{tocdepth}{-1}
        }


        \__UWMad_ListOf_CurrentSectioningCommmand:n
            {\tl_use:c {g__UWMad_ListOf#1_Title_#2_tl}}


        \tl_if_eq:nnTF #2 {Main} {
            \UWMad_ListOf_IfIncludeInTOC:nTF {#1} { } {
                \setcounter{tocdepth}{\l_tmpa_int}
            }
        } {
            \setcounter{tocdepth}{\l_tmpa_int}
        }

    } {

        \cs_generate_variant:Nn \tl_if_eq:nnTF {onTF}
        \tl_set:Nn \l_tmpa_tl {Main}

        \phantomsection
        \__UWMad_ListOf_CurrentSectioningCommmand:n*
        {\tl_use:c {g__UWMad_ListOf#1_Title_#2_tl}}
        \tl_if_eq:onTF {#2} {Main} {
            \UWMad_ListOf_IfIncludeInTOC:nTF {#1} {
                \addcontentsline
                    {toc}
                    {\tl_use:c {g__UWMad_ListOf#1_Section_#2_tl}}
                    {\tl_use:c {g__UWMad_ListOf#1_Title_#2_tl}}
            } { }
        } { }
    }

}
%    \end{macrocode}
%   \end{function}
%
%
%
%   \begin{function}{\UWMad_ListOf_StartGroup:nn}
%   \begin{syntax}
%       \cs{UWMad_ListOf_StartGroup:n}\Arg{ID}\Arg{Group}
%   \end{syntax}
%   A shortcut command that prints the entires in the current queue
%   and then starts the next section by printing the title.
%
%    \begin{macrocode}
\cs_new:Nn \UWMad_ListOf_StartGroup:nn {
    \UWMad_ListOf_PrintEntries:n{#1}
    \UWMad_ListOf_PrintTitle:nn {#1}{#2}
}
%    \end{macrocode}
%   \end{function}
%
%
%
%
%
%   \UWSubModule{Nomenclature}
%   Dimensions that are calculated are declared first.
%    \begin{macrocode}
\dim_new:N \l__UWMad_Nomenclature_WidestSymbol_dim
\dim_new:N \l__UWMad_Nomenclature_WidestUnit_dim
\dim_new:N \l__UWMad_Nomenclature_Entry_WidthSymbol_dim
\dim_new:N \l__UWMad_Nomenclature_Entry_WidthUnits_dim
\dim_new:N \l__UWMad_Nomenclature_Entry_WidthDescription_dim
%    \end{macrocode}
%
%   Then user-adjustable dimensions are declared.
%    \begin{macrocode}
\dim_new:N \l__UWMad_Nomenclature_TitleSkip_dim
\dim_new:N \l__UWMad_Nomenclature_PrintSkip_dim
\dim_new:N \l__UWMad_Nomenclature_Entry_MarginLeft_dim
\dim_new:N \l__UWMad_Nomenclature_Entry_MarginBottom_dim
\dim_new:N \l__UWMad_Nomenclature_Entry_Padding_dim
%    \end{macrocode}
%
%   The token lists that hold the section and title of the
%   groups are declared
%    \begin{macrocode}
\tl_new:N \l__UWMad_Nomenclature_Section_Main_tl
\tl_new:N \l__UWMad_Nomenclature_Section_Group_tl
\tl_new:N \l__UWMad_Nomenclature_Section_Subgroup_tl
\tl_new:N \l__UWMad_Nomenclature_Title_Main_tl
\tl_new:N \l__UWMad_Nomenclature_Title_Group_tl
\tl_new:N \l__UWMad_Nomenclature_Title_Subgroup_tl
%    \end{macrocode}
%
%   Now the keys for user-customization are defined:
%    \begin{macrocode}
\clist_new:N   \g__UWMad_Nomenclature_KeyValuePairs_clist
\clist_gset:Nn \g__UWMad_Nomenclature_KeyValuePairs_clist {
%    \end{macrocode}
%
%   Adjustable dimensions:
%    \begin{macrocode}
    title-skip .dim_set:N = \l__UWMad_Nomenclature_TitleSkip_dim,
    print-skip .dim_set:N = \l__UWMad_Nomenclature_PrintSkip_dim,
    entry-margin-left .dim_set:N =
        \l__UWMad_Nomenclature_Entry_MarginLeft_dim,
    entry-margin-bottom .dim_set:N =
        \l__UWMad_Nomenclature_Entry_MarginBottom_dim,
    entry-padding .dim_set:N =
        \l__UWMad_Nomenclature_Entry_Padding_dim,
%    \end{macrocode}
%
%   Adjustable dimension defaults:
%    \begin{macrocode}
    title-skip          .default:n  = {0.00pt},
    print-skip          .default:n  = {1.00em},
    entry-margin-left   .default:n  = {1.00em},
    entry-margin-bottom .default:n  = {0.25em},
    entry-padding       .default:n  = {0.75em},
%    \end{macrocode}
%
%   Group section adjustments:
%    \begin{macrocode}
    main-section .code:n = {
        \tl_set:Nn
            \l__UWMad_Nomenclature_Section_Main_tl {#1}
    },
    group-section .code:n = {
        \tl_set:Nn
            \l__UWMad_Nomenclature_Section_Group_tl {#1}
    },
    subgroup-section .code:n = {
        \tl_set:Nn
            \l__UWMad_Nomenclature_Section_Subgroup_tl {#1}
    },
%    \end{macrocode}
%
%   The default nomenclature section is chapter.
%   Since the other two groups of empty by default, the Nomenclature
%   environment will handle them.
%    \begin{macrocode}
    main-section .default:n = chapter,
%    \end{macrocode}
%
%   Group title adjustments:
%    \begin{macrocode}
    main-title .code:n = {
        \tl_set:Nn
            \l__UWMad_Nomenclature_Title_Main_tl {#1}
    },
    group-title .code:n = {
        \tl_set:Nn
            \l__UWMad_Nomenclature_Title_Group_tl {#1}
    },
    subgroup-title .code:n = {
        \tl_set:Nn
            \l__UWMad_Nomenclature_Title_Subgroup_tl {#1}
    },
%    \end{macrocode}
%
%   Group title default for main group only:
%    \begin{macrocode}
    main-title .default:n = Nomenclature,
%    \end{macrocode}
%
%   Miscellaneous options:
%    \begin{macrocode}
    numbered       .bool_gset:N =
        \g__UWMad_Nomenclature_IsNumbered_bool,
    include-in-toc .bool_gset:N =
        \g__UWMad_Nomenclature_IncludeInTOC_bool,
    with-units .bool_gset:N =
        \g__UWMad_Nomenclature_IncludeUnitsColumn_bool,
%    \end{macrocode}
%
%   Miscellaneous option defaults:
%    \begin{macrocode}
      numbered     .default:n = false,
    include-in-toc .default:n = true,
     with-units    .default:n = false
}
%    \end{macrocode}
%
%   Define the keys for the Nomenclature system by expanding the \texttt{clist}
%   created above.
%    \begin{macrocode}
\exp_args:Nnf
    \keys_define:nn
    { UWMadThesis / Nomenclature }
    {
        \clist_use:Nn \g__UWMad_Nomenclature_KeyValuePairs_clist {,}
    }
%    \end{macrocode}
%
%   And the defaults for all keys are now set.
%    \begin{macrocode}
\keys_set:nn { UWMadThesis / Nomenclature } {
    title-skip          ,
    print-skip          ,
    entry-margin-left   ,
    entry-margin-bottom ,
    entry-padding       ,
    numbered            ,
    include-in-toc      ,
    with-units          ,
    main-section        ,
    main-title
}
%    \end{macrocode}
%
%
%
%
%   \begin{function} {
%       \UWMad_Nomenclature_UpdateWidest:Nn,
%       \UWMad_Nomenclature_UpdateWidest_Symbol:n,
%       \UWMad_Nomenclature_UpdateWidest_Units:n,
%       }
%       \begin{syntax}
%           \cs{UWMad_Nomenclature_UpdateWidest:Nn}\meta{dim}\marg{object}
%           \cs{UWMad_Nomenclature_UpdateWidest_Symbol:n}\marg{symbol}
%           \cs{UWMad_Nomenclature_UpdateWidest_Units:n}\marg{units}
%       \end{syntax}
%       These commands update the widest symbol and widest unit lengths.
%    \begin{macrocode}
\cs_new:Nn \UWMad_Nomenclature_UpdateWidest:Nn {
    \hbox_set:Nn \l_tmpa_box {#2}
    \dim_set:Nn  \l_tmpa_dim {\box_wd:N \l_tmpa_box}
    \dim_compare:nNnTF {#1} < {\l_tmpa_dim} {
        \dim_set_eq:NN #1 \l_tmpa_dim
    } { }
}
\cs_new:Nn \UWMad_Nomenclature_UpdateWidest_Symbol:n {
    \UWMad_Nomenclature_UpdateWidest:Nn
        \l__UWMad_Nomenclature_WidestSymbol_dim {#1}
}
%
\cs_new:Nn \UWMad_Nomenclature_UpdateWidest_Units:n {
    \UWMad_Nomenclature_UpdateWidest:Nn
        \l__UWMad_Nomenclature_WidestUnit_dim {#1}
}
%    \end{macrocode}
%   \end{function}
%
%   And the defaults for all keys are now set.
%   \begin{function} {
%       \UWMad_Nomenclature_ZeroWidest_Symbol:,
%       \UWMad_Nomenclature_ZeroWidest_Unit:
%       }
%       \begin{syntax}
%           \cs{UWMad_Nomenclature_ZeroWidest_Symbol:}
%           \cs{UWMad_Nomenclature_ZeroWidest_Symbol:}
%       \end{syntax}
%       These commands set the widest symbol and unit lengths to 0pt.
%    \begin{macrocode}
\cs_new:Nn \UWMad_Nomenclature_ZeroWidest_Symbol: {
    \dim_set:Nn \l__UWMad_Nomenclature_WidestSymbol_dim {0pt}
}
\cs_new:Nn \UWMad_Nomenclature_ZeroWidest_Unit: {
    \dim_set:Nn \l__UWMad_Nomenclature_WidestUnit_dim {0pt}
}
%    \end{macrocode}
%   \end{function}
%
%
%   And the defaults for all keys are now set.
%   \begin{function} {
%       \UWMad_Nomenclature_SetEntryWidths_NoUnits:,
%       \UWMad_Nomenclature_SetEntryWidths_Units:
%       }
%       \begin{syntax}
%           \cs{UWMad_Nomenclature_SetEntryWidths_NoUnits:}
%           \cs{UWMad_Nomenclature_SetEntryWidths_Units:}
%       \end{syntax}
%       These commands sets the widths of the description, symbol, and (if
%       present) unit boxes for a particular entry.
%    \begin{macrocode}
\cs_new:Nn \UWMad_Nomenclature_SetEntryWidths_NoUnits: {
    \dim_set:Nn \l__UWMad_Nomenclature_Entry_WidthSymbol_dim {
        1.01\l__UWMad_Nomenclature_WidestSymbol_dim
    }
    \dim_set:Nn \l__UWMad_Nomenclature_Entry_WidthDescription_dim {
        0.995\textwidth -
        \l__UWMad_Nomenclature_Entry_MarginLeft_dim  -
        \l__UWMad_Nomenclature_Entry_WidthSymbol_dim -
        \l__UWMad_Nomenclature_Entry_Padding_dim
    }
}
\cs_new:Nn \UWMad_Nomenclature_SetEntryWidths_Units: {
    \dim_set:Nn \l__UWMad_Nomenclature_Entry_WidthSymbol_dim {
        1.01\l__UWMad_Nomenclature_WidestSymbol_dim
    }
    \dim_set:Nn \l__UWMad_Nomenclature_Entry_WidthUnit_dim {
        1.01\l__UWMad_Nomenclature_WidestUnit_dim
    }
    \dim_set:Nn \l__UWMad_Nomenclature_Entry_WidthDescription_dim {
        0.995\textwidth -
         \l__UWMad_Nomenclature_Entry_MarginLeft_dim  -
         \l__UWMad_Nomenclature_Entry_WidthSymbol_dim -
         \l__UWMad_Nomenclature_Entry_WidthUnit_dim   -
        2\l__UWMad_Nomenclature_Entry_Padding_dim
    }
}
%    \end{macrocode}
%   \end{function}
%
%
%   And the defaults for all keys are now set.
%   \begin{function} {
%       \UWMad_Nomenclature_SetEntryWidths:
%       }
%       \begin{syntax}
%           \cs{UWMad_Nomenclature_SetEntryWidths:}
%       \end{syntax}
%       This function calls one of the appropriate above setters.
%    \begin{macrocode}
\cs_new:Nn \UWMad_Nomenclature_SetEntryWidths: {
    \bool_if:NTF \g__UWMad_Nomenclature_IncludeUnitsColumn_bool {
        \UWMad_Nomenclature_SetEntryWidths_Units:
    } {
        \UWMad_Nomenclature_SetEntryWidths_NoUnits:
    }
}
%    \end{macrocode}
%   \end{function}
%
%
%   And the defaults for all keys are now set.
%   \begin{function} {
%       \UWMad_Nomenclature_SetEntry_NoUnits:nn,
%       \UWMad_Nomenclature_SetEntry_Units:nnn
%       }
%       \begin{syntax}
%           \cs{UWMad_Nomenclature_SetEntry_NoUnits:}
%               \Arg{symbol}\Arg{description}
%           \cs{UWMad_Nomenclature_SetEntry_Units:}
%               \Arg{symbol}\Arg{units}\Arg{description}
%       \end{syntax}
%       These functions typeset the contents passed into them.
%    \begin{macrocode}
\coffin_new:N \l_tmpc_coffin
\cs_new:Nn \UWMad_Nomenclature_SetEntry_NoUnits:nn {
    \vcoffin_set:Nnn
        \l_tmpa_coffin
        {\l__UWMad_Nomenclature_Entry_WidthSymbol_dim}
        {#1}
    \vcoffin_set:Nnn
        \l_tmpb_coffin
        {\l__UWMad_Nomenclature_Entry_WidthDescription_dim}
        {#2}
    \group_begin:
        \setstretch{1.1}
        \skip_horizontal:n {\l__UWMad_Nomenclature_Entry_MarginLeft_dim}
        \coffin_typeset:Nnnnn \l_tmpa_coffin {l}{t}{0pt}{0pt}
        \skip_horizontal:n {\l__UWMad_Nomenclature_Entry_Padding_dim}
        \coffin_typeset:Nnnnn \l_tmpb_coffin {l}{t}{0pt}{0pt}
        \skip_vertical:n {\l__UWMad_Nomenclature_Entry_MarginBottom_dim}
    \group_end:
}
\cs_new:Nn \UWMad_Nomenclature_SetEntry_Units:nnn {
    \vcoffin_set:Nnn
        \l_tmpa_coffin
        {\l__UWMad_Nomenclature_Entry_WidthSymbol_dim}
        {#1}
    \vcoffin_set:Nnn
        \l_tmpb_coffin
        {\l__UWMad_Nomenclature_Entry_WidthUnit_dim}
        {#2}
    \vcoffin_set:Nnn
        \l_tmpc_coffin
        {\l__UWMad_Nomenclature_Entry_WidthDescription_dim}
        {#3}
    \group_begin:
        \setstretch{1.1}
        \skip_horizontal:n {\l__UWMad_Nomenclature_Entry_MarginLeft_dim}
        \coffin_typeset:Nnnnn \l_tmpa_coffin {l}{t}{0pt}{0pt}
        \skip_horizontal:n {\l__UWMad_Nomenclature_Entry_Padding_dim}
        \coffin_typeset:Nnnnn \l_tmpb_coffin {l}{t}{0pt}{0pt}
        \skip_horizontal:n {\l__UWMad_Nomenclature_Entry_Padding_dim}
        \coffin_typeset:Nnnnn \l_tmpc_coffin {l}{t}{0pt}{0pt}
        \skip_vertical:n {\l__UWMad_Nomenclature_Entry_MarginBottom_dim}
    \group_end:
}
%    \end{macrocode}
%   \end{function}
%
%
%
%    \begin{macrocode}
\DeclareDocumentEnvironment {Nomenclature} { o g } {
%
%   Create the ListOf
    \UWMad_ListOf_Define:n {Nomenclature}
%
%
%   Check for an optional section declaration and
%   set Main section token list.
    \IfNoValueTF {#1} { } {
        \UWMad_IfSectionExists:nT {#1} { }
        \tl_set:Nn \l__UWMad_Nomenclature_Section_Main_tl {#1}
    }
%
%
%   Check for an optional section declaration and
%   set Main section token list.
    \IfNoValueTF {#2} { } {
        \tl_set:Nf
            \l__UWMad_Nomenclature_Title_Main_tl {#2}
    }
    \UWMad_ListOf_SetTitle_Main:nn {Nomenclature}
        {\l__UWMad_Nomenclature_Title_Main_tl}
%
%
   \bool_if:NTF \g__UWMad_Nomenclature_IsNumbered_bool {
        \UWMad_ListOf_MakeNumbered:n    {Nomenclature}
    }{
        \UWMad_ListOf_MakeNotNumbered:n {Nomenclature}
    }
%
%
%
%   If Group section token list is empty, set it to the following
%   section after Main in the LaTeX sectioning hierarchy.
%   Otherwise, take the value at its word.
    \tl_if_empty:NTF \l__UWMad_Nomenclature_Section_Group_tl {
        \tl_set:Nf \l__UWMad_Nomenclature_Section_Group_tl {
            \UWMad_NextSection:n {
                \l__UWMad_Nomenclature_Section_Main_tl
            }
        }
    } { }
%
%   If Subgroup section token list is empty, set it to the following
%   section after Group in the LaTeX sectioning hierarchy.
%   Otherwise, take the value at its word.
    \tl_if_empty:NTF \l__UWMad_Nomenclature_Section_Subgroup_tl {
        \tl_set:Nf \l__UWMad_Nomenclature_Section_Subgroup_tl {
            \UWMad_NextSection:n {
                \l__UWMad_Nomenclature_Section_Group_tl
            }
        }
    } { }
%
%   Set the sections with the Nomenclature ListOf instance
    \UWMad_ListOf_SetSection_Main:nn
        {Nomenclature} {\l__UWMad_Nomenclature_Section_Main_tl}
    \UWMad_ListOf_SetSection_Group:nn
        {Nomenclature} {\l__UWMad_Nomenclature_Section_Group_tl}
    \UWMad_ListOf_SetSection_Subgroup:nn
        {Nomenclature} {\l__UWMad_Nomenclature_Section_Subgroup_tl}
%
%
%   Determine if this nomenclature should be in the Table of Contents
    \bool_if:NTF \g__UWMad_Nomenclature_IncludeInTOC_bool {
        \UWMad_ListOf_IncludeInTOC:n {Nomenclature}
    } {
        \UWMad_ListOf_DoNotIncludeInTOC:n {Nomenclature}
    }
%
%   Set some hooks in the Nomenclature ListOf instance
    \UWMad_ListOf_SetHook:nnn {Nomenclature} {PrePrint} {
        \UWMad_Nomenclature_SetEntryWidths:
    }
    \UWMad_ListOf_SetHook:nnn {Nomenclature} {PostPrint} {
        \UWMad_Nomenclature_ZeroWidest_Symbol:
    }
%
%
%   User front-end for creating a Group
    \DeclareDocumentCommand \Group { G{} } {
        \IfNoValueTF {##1} { } {
            \tl_set:Nn
                \l__UWMad_Nomenclature_Title_Group_tl {##1}
        }
        \UWMad_ListOf_SetTitle_Group:nn {Nomenclature}
            {\l__UWMad_Nomenclature_Title_Group_tl}
        \UWMad_ListOf_StartGroup:nn{Nomenclature}{Group}
    }
%
%   User front-end for creating a Subgroup
    \DeclareDocumentCommand \Subgroup { G{} } {
        \IfNoValueTF {##1} { } {
            \tl_set:Nn
                \l__UWMad_Nomenclature_Title_Subgroup_tl {##1}
        }
        \UWMad_ListOf_SetTitle_Subgroup:nn {Nomenclature}
            {\l__UWMad_Nomenclature_Title_Subgroup_tl}
        \UWMad_ListOf_StartGroup:nn{Nomenclature}{Subgroup}
    }
%
%   User front-end for creating an entry
    \bool_if:NTF \g__UWMad_Nomenclature_IncludeUnitsColumn_bool {
        \DeclareDocumentCommand \Entry { m m m } {
            \UWMad_ListOf_PushEntry:nn {Nomenclature} {
                \UWMad_Nomenclature_SetEntry_Units:nnn
                    {##1} {##2} {##3}
            }
            \UWMad_Nomenclature_UpdateWidest_Symbol:n{##1}
        }
    } {
        \DeclareDocumentCommand \Entry { m m } {
            \UWMad_ListOf_PushEntry:nn {Nomenclature} {
                \UWMad_Nomenclature_SetEntry_NoUnits:nn
                    {##1} {##2}
            }
            \UWMad_Nomenclature_UpdateWidest_Symbol:n{##1}
        }
    }
%
%   User front-end for reseting the column width
    \DeclareDocumentCommand  \ResetColumnWidth { } {
        \UWMad_Nomenclature_ZeroWidest_Symbol:
        \UWMad_Nomenclature_ZeroWidest_Unit:
    }
%
%
%   Print the main section title
    \UWMad_ListOf_PrintTitle:nn {Nomenclature}{Main}
%
} {
%   Flush the remaining entries from the ListOf queue and
%   delete the Nomenclature ListOf instance.
    \UWMad_ListOf_PrintEntries:n {Nomenclature}
    \UWMad_ListOf_Delete:n{Nomenclature}
}
%
%
%
%
\DeclareDocumentEnvironment {Acronym} { o G{Acronym} } {

    \begin{Nomenclature}[#1]{#2}
%
%
    \UWMad_Hash_Define:n{Acronyms}
    \UWMad_Hash_Define:n{AcronymMeanings}
%
%
    \cs_undefine:N \Entry
    \DeclareDocumentCommand \Entry { o m m } {
        \IfNoValueTF {##1} {

            \UWMad_Hash_Set:nnn{Acronyms}       {##2}{##2}
            \UWMad_Hash_Set:nnn{AcronymMeanings}{##2}{##3}
            \bool_new:c {g__UWMad_Acronym_WasSet_##2_bool}
            %
            \UWMad_ListOf_PushEntry:nn {Nomenclature} {
                \hypertarget{Acronym:##2}{}
                \UWMad_Nomenclature_SetEntry_NoUnits:nn
                    {##2} {##3}
            }

        } {

            \UWMad_Hash_Set:nnn{Acronyms}       {##1}{##2}
            \UWMad_Hash_Set:nnn{AcronymMeanings}{##1}{##3}
            \bool_new:c {g__UWMad_Acronym_WasSet_##1_bool}
            %
            \UWMad_ListOf_PushEntry:nn {Nomenclature} {
                \hypertarget{Acronym:##1}{}
                \UWMad_Nomenclature_SetEntry_NoUnits:nn
                    {##2} {##3}
            }

        }
        \UWMad_Nomenclature_UpdateWidest_Symbol:n{##2}
    }
} {

    \end{Nomenclature}

}
%
%
%
\cs_new:Nn \UWMad_Acronym_CreateLink:n {
    \hyperlink{Acronym:#1}{
        \color{\g__UWMad_Acronym_LinkColor_tl}
        \UWMad_Hash_Get:nn{Acronyms}{#1}
    }
}
%
%
\DeclareDocumentCommand \Acro { m } {
    \UWMad_Hash_IfKeySet:nnTF {Acronyms} {#1} {
        \bool_if:cTF {g__UWMad_Acronym_WasSet_#1_bool} {
            \bool_if:NTF \g__UWMad_Acronym_UseLinks_bool {
                \UWMad_Acronym_CreateLink:n{#1}
            } {
                \UWMad_Hash_Get:nn{Acronyms}{#1}
            }
        } {
            \UWMad_Hash_Get:nn{AcronymMeanings}{#1}~
                (
                    \UWMad_Hash_Get:nn{Acronyms}{#1}
                )
            \bool_gset_true:c {g__UWMad_Acronym_WasSet_#1_bool}
        }
    } { }
}
%
%
%    \end{macrocode}
%
%   Define the keys for the Acronym system by expanding the \texttt{clist}
%   created for the Nomenclature system.
%    \begin{macrocode}
\exp_args:Nnf
    \keys_define:nn
    { UWMadThesis / Acronym }
    {
        \clist_use:Nn \g__UWMad_Nomenclature_KeyValuePairs_clist {,}
    }
\keys_define:nn { UWMadThesis / Acronym } {
    use-links .bool_gset:N = \g__UWMad_Acronym_UseLinks_bool,
    use-links .default:n = true,
    link-color .tl_gset:N = \g__UWMad_Acronym_LinkColor_tl,
    link-color .default:n = blue
}
%    \end{macrocode}
%
%   And the defaults for all keys are now set.
%    \begin{macrocode}
\keys_set:nn { UWMadThesis / Acronym } {
    title-skip          ,
    print-skip          ,
    entry-margin-left   ,
    entry-margin-bottom ,
    entry-padding       ,
    numbered            ,
    include-in-toc      ,
    with-units          ,
    main-section        ,
    main-title          ,
    use-links           ,
    link-color
}
%    \end{macrocode}
%
%
%
%
%
%   \iffalse
%</Code>
%   \fi
%   \iffalse
%<*Code>
%   \fi
%^^A ------------------------------------------------------------------------ %
%^^A                    Metadata Writing/Importing Routines                   %
%^^A ------------------------------------------------------------------------ %
%
%   \UWModule{Thesis and PDF Information}
%
%
%   \UWSubModule{Metadata clist and Aux Write}
%   Since the metadata (i.e., properties) of a PDF must be set in the
%   preamble but typically a user defines them in the document, these
%   routines write the supported metadata that a user may define to an
%   auxiliary file that is then imported upon recompilation.  It uses
%   the |expl3| |clist| commands to define and build the CSV list, and
%   then writes to the file.
%
%
%   Define the |clist|.
%    \begin{macrocode}
\clist_new:N \g__UWMad_MetaDataList_clist
%    \end{macrocode}
%
%   Define a command for pushing entries (with a brace guard) on to the
%   |clist|.
%    \begin{macrocode}
\cs_new:Nn \UWMad_MetaData_PushToList:nn {
   \clist_gput_right:Nn \g__UWMad_MetaDataList_clist {
        #1={#2}
    }
}
%    \end{macrocode}
%
%   Define to booleans: one to tell if a auxilary file is needed
%   and to tell if the |document| has begun.
%    \begin{macrocode}
\bool_new:N \g__UWMad_MetaData_GenerateAux_bool
\bool_new:N \g__UWMad_MetaData_IsDocument_bool
%    \end{macrocode}
%
%   Look for a auxilary file and load it if it exists.
%    \begin{macrocode}
\file_if_exist:nTF{\c_job_name_tl.UWMad.PDFMetaData.aux} {
    \file_input:n {\c_job_name_tl.UWMad.PDFMetaData.aux}
}{}
%    \end{macrocode}
%
%   At the beginning of the document, if data has been pushed to the list,
%   pass it to \cs{hypersetup} so the PDF gets it.
%   Also, set the |IsDocument| boolean true.
%    \begin{macrocode}
\AtBeginDocument{
    \clist_if_empty:NTF \g__UWMad_MetaDataList_clist { } {
        \exp_args:Nx \hypersetup {
            \clist_use:Nn\g__UWMad_MetaDataList_clist{,}
        }
    } { }
    \bool_gset_true:N \g__UWMad_MetaData_IsDocument_bool
}
%    \end{macrocode}
%
%   If thesis information of PDF metadata was used within |document|,
%   write that information to an auxilary file.
%    \begin{macrocode}
\AtEndDocument{
    \bool_if:NTF \g__UWMad_MetaData_GenerateAux_bool {
        \clist_if_empty:NTF \g__UWMad_MetaDataList_clist { } {
            \iow_new:N   \g__UWMad_PDFMetaData_HyperSetup_io
            \iow_open:Nn \g__UWMad_PDFMetaData_HyperSetup_io {
                \c_job_name_tl.UWMad.PDFMetaData.aux
            }
            \iow_now:Nx  \g__UWMad_PDFMetaData_HyperSetup_io {
                \noexpand\ExplSyntaxOff
                    \noexpand\hypersetup
                    {\clist_use:Nn\g__UWMad_MetaDataList_clist{,}}
                \noexpand\ExplSyntaxOn
            }
            \iow_close:N \g__UWMad_PDFMetaData_HyperSetup_io
        } { }
    } { }
}
%    \end{macrocode}
%
%
%
%
%
%
%
%   \UWSubModule{Thesis Information}
%
%
%
%   Declare the |ThesisInfo| token list variables.
%    \begin{macrocode}
\tl_new:N \g__UWMad_ThesisInfo_Title_tl
\tl_new:N \g__UWMad_ThesisInfo_Author_tl
\tl_new:N \g__UWMad_ThesisInfo_DefenseDate_tl
\tl_new:N \g__UWMad_ThesisInfo_Department_tl
\tl_new:N \g__UWMad_ThesisInfo_Program_tl
\tl_new:N \g__UWMad_ThesisInfo_Degree_tl
\tl_new:N \g__UWMad_ThesisInfo_DocumentType_tl
\tl_new:N \g__UWMad_ThesisInfo_AdvisorName_tl
\tl_new:N \g__UWMad_ThesisInfo_AdvisorPosition_tl
\tl_new:N \g__UWMad_ThesisInfo_AdvisorAssociation_tl
\tl_new:N \g__UWMad_ThesisInfo_AdvisorMarker_tl
\tl_new:N \g__UWMad_ThesisInfo_Institution_tl
%    \end{macrocode}
%
%   Set the document type default.
%    \begin{macrocode}
\tl_gset:Nn \g__UWMad_ThesisInfo_DocumentType_tl {report}
%    \end{macrocode}
%
%   Define some booleans for required information.
%    \begin{macrocode}
\bool_new:N \g__UWMad_ThesisInfo_IsSet_Title_bool
\bool_new:N \g__UWMad_ThesisInfo_IsSet_Author_bool
\bool_new:N \g__UWMad_ThesisInfo_IsSet_DefenseDate_bool
\bool_new:N \g__UWMad_ThesisInfo_IsSet_Program_bool
\bool_new:N \g__UWMad_ThesisInfo_IsSet_Degree_bool
\bool_new:N \g__UWMad_ThesisInfo_IsSet_Institution_bool
\bool_new:N \g__UWMad_ThesisInfo_IsSet_Advisor_bool
%    \end{macrocode}
%
%   Declare the user front-end for the title.
%    \begin{macrocode}
\DeclareDocumentCommand \Title { m } {
%    \end{macrocode}
%   Set the associated token list variable
%    \begin{macrocode}
    \tl_gset:Nn \g__UWMad_ThesisInfo_Title_tl {#1}
%    \end{macrocode}
%   Pass it to the default \LaTeX{} \cs{title} command.
%    \begin{macrocode}
    \title{#1}
%    \end{macrocode}
%   Push the value to the MetaData |clist|.
%    \begin{macrocode}
    \UWMad_MetaData_PushToList:nn{pdftitle}    {#1}
%    \end{macrocode}
%   If this command was used within the |document|, tell the class
%   to write an auxilary file.
%    \begin{macrocode}
    \bool_if:NTF \g__UWMad_MetaData_IsDocument_bool {
        \bool_gset_true:N \g__UWMad_MetaData_GenerateAux_bool
    } { }
%    \end{macrocode}
%   Tell the class this variable is now set.
%    \begin{macrocode}
    \bool_gset_true:N \g__UWMad_ThesisInfo_IsSet_Title_bool
}
%    \end{macrocode}
%
%   Similar flow to the \cs{Title} defintion.
%    \begin{macrocode}
\DeclareDocumentCommand \Author { m } {
    \tl_gset:Nn \g__UWMad_ThesisInfo_Author_tl {#1}
    \author{#1}
    \UWMad_MetaData_PushToList:nn{pdfauthor}   {#1}
    \bool_if:NTF \g__UWMad_MetaData_IsDocument_bool {
        \bool_gset_true:N \g__UWMad_MetaData_GenerateAux_bool
    } { }
    \bool_gset_true:N \g__UWMad_ThesisInfo_IsSet_Author_bool
}
%    \end{macrocode}
%
%   A simple setter command.
%    \begin{macrocode}
\DeclareDocumentCommand \Program { m } {
    \tl_gset:Nn \g__UWMad_ThesisInfo_Program_tl {#1}
    \bool_gset_true:N \g__UWMad_ThesisInfo_IsSet_Program_bool
}
%    \end{macrocode}
%
%   A simple setter command.
%    \begin{macrocode}
\DeclareDocumentCommand \Degree { m } {
    \tl_gset:Nn \g__UWMad_ThesisInfo_Degree_tl {#1}
    \bool_gset_true:N \g__UWMad_ThesisInfo_IsSet_Degree_bool
}
%    \end{macrocode}
%
%   Semantic names for the \cs{Degree} function.
%    \begin{macrocode}
\DeclareDocumentCommand \Doctorate { } {
    \tl_gset:Nn \g__UWMad_ThesisInfo_Degree_tl {Doctor~of~Philosophy}
    \bool_gset_true:N \g__UWMad_ThesisInfo_IsSet_Degree_bool
}
\DeclareDocumentCommand \Masters { } {
    \tl_gset:Nn \g__UWMad_ThesisInfo_Degree_tl {Master's}
    \bool_gset_true:N \g__UWMad_ThesisInfo_IsSet_Degree_bool
}
\DeclareDocumentCommand \Bachelors { } {
    \tl_gset:Nn \g__UWMad_ThesisInfo_Degree_tl {Bachelor's}
    \bool_gset_true:N \g__UWMad_ThesisInfo_IsSet_Degree_bool
}
%    \end{macrocode}
%
%   A simple setter command.
%    \begin{macrocode}
\DeclareDocumentCommand \DocumentType { m } {
    \tl_gset:Nn \g__UWMad_ThesisInfo_DocumentType_tl {#1}
}
%    \end{macrocode}
%
%   Semantic names for the \cs{DocumentType} function.
%    \begin{macrocode}
\DeclareDocumentCommand \Dissertation { } {
    \tl_gset:Nn \g__UWMad_ThesisInfo_DocumentType_tl {
        dissertation
    }
}
\DeclareDocumentCommand \DoctoralThesis { } {
    \tl_gset:Nn \g__UWMad_ThesisInfo_DocumentType_tl {
        doctoral~thesis
    }
}
\DeclareDocumentCommand \MastersThesis { } {
    \tl_gset:Nn \g__UWMad_ThesisInfo_DocumentType_tl {
        master's~thesis
    }
}
\DeclareDocumentCommand \Thesis { } {
    \tl_gset:Nn \g__UWMad_ThesisInfo_DocumentType_tl {
        thesis
    }
}
\DeclareDocumentCommand \Prelim { } {
    \tl_gset:Nn \g__UWMad_ThesisInfo_DocumentType_tl {
        preliminary~report
    }
}
%    \end{macrocode}
%
%   A simple setter command and aliases.
%    \begin{macrocode}
\DeclareDocumentCommand \DefenseDate { m } {
    \tl_gset:Nn \g__UWMad_ThesisInfo_DefenseDate_tl {#1}
    \bool_gset_true:N \g__UWMad_ThesisInfo_IsSet_DefenseDate_bool
}
\cs_gset_eq:NN \DefenceDate \DefenseDate
%    \end{macrocode}
%
%   A simple setter command and alias.
%    \begin{macrocode}
\DeclareDocumentCommand \Institution { m } {
    \tl_gset:Nn       \g__UWMad_ThesisInfo_Institution_tl {#1}
    \bool_gset_true:N \g__UWMad_ThesisInfo_IsSet_Institution_bool
}
\cs_set_eq:NN \University \Institution
%    \end{macrocode}
%
%   Define the optional user interface.
%    \begin{macrocode}
\DeclareDocumentCommand \Department { m } {
    \tl_gset:Nn \g__UWMad_ThesisInfo_Department_tl {#1}
}
%    \end{macrocode}
%
%   Define the Advisor and Adviser user interface.
%    \begin{macrocode}
\cs_new:Nn \UWMad_ThesisInfo_AdvisorInfo:nnn {
    \tl_gset:Nn \g__UWMad_ThesisInfo_AdvisorName_tl        {#1}
    \tl_gset:Nn \g__UWMad_ThesisInfo_AdvisorPosition_tl    {#2}
    \tl_gset:Nn \g__UWMad_ThesisInfo_AdvisorAssociation_tl {#3}
}
\DeclareDocumentCommand \Advisor { m m m } {
    \bool_gset_true:N \g__UWMad_ThesisInfo_IsSet_Advisor_bool
    \UWMad_ThesisInfo_AdvisorInfo:nnn{#1}{#2}{#3}
    \tl_gset:Nn \g__UWMad_ThesisInfo_AdvisorMarker_tl {Advisor}
 }
\DeclareDocumentCommand \Adviser { m m m } {
    \bool_gset_true:N \g__UWMad_ThesisInfo_IsSet_Advisor_bool
    \UWMad_ThesisInfo_AdvisorInfo:nnn{#1}{#2}{#3}
    \tl_gset:Nn \g__UWMad_ThesisInfo_AdvisorMarker_tl {Adviser}
}
%    \end{macrocode}
%
%   Define an author interface for determingin if required information
%   has been set.
%    \begin{macrocode}
\msg_new:nnn { UWMadThesis } { ThesisInfo / UnsetInformation } {
    The~required~information~for~the~#1~is~not~set.
}
\DeclareDocumentCommand \IfInfoIsSetT { m +m } {
    \bool_if:cTF {g__UWMad_ThesisInfo_IsSet_ #1 _bool} {
        #2
    } {
        \msg_error:nnn
            { UWMadThesis }
            { ThesisInfo / UnsetInformation }
            {#1}
    }
}
%    \end{macrocode}
%
%   Define user accessors for thesis info.
%    \begin{macrocode}
\DeclareDocumentCommand \TheTitle { } {
    \g__UWMad_ThesisInfo_Title_tl
}
\DeclareDocumentCommand \TheAuthor { } {
    \g__UWMad_ThesisInfo_Author_tl
}
\DeclareDocumentCommand \TheProgram { } {
    \g__UWMad_ThesisInfo_Program_tl
}
\DeclareDocumentCommand \TheDegree { } {
    \g__UWMad_ThesisInfo_Degree_tl
}
\DeclareDocumentCommand \TheDocumentType { } {
    \g__UWMad_ThesisInfo_DocumentType_tl
}
\DeclareDocumentCommand \TheDefenseDate { } {
    \g__UWMad_ThesisInfo_DefenseDate_tl
}
\cs_gset_eq:NN \TheDefenceDate \TheDefenseDate
\DeclareDocumentCommand \TheInstitution { } {
    \g__UWMad_ThesisInfo_Institution_tl
}
\cs_set_eq:NN \TheUniversity \TheInstitution
%
\DeclareDocumentCommand \TheDepartment { } {
    \g__UWMad_ThesisInfo_Department_tl
}
\DeclareDocumentCommand \TheAdvisor { } {
    \g__UWMad_ThesisInfo_AdvisorName_tl
}
%    \end{macrocode}
%
%
%
%
%
%   \UWSubModule{Committee Member List}
%
%
%   Define internals for the Committee member list: a separator, a count, and
%   and |ListOf|.
%    \begin{macrocode}
\tl_new:N   \g__UWMad_ThesisInfo_Committee_InfoSeparator_tl
\tl_gset:Nn \g__UWMad_ThesisInfo_Committee_InfoSeparator_tl {,}
\int_new:N \g__UWMad_ThesisInfo_CommitteeCount_int
\UWMad_ListOf_Define:n {CommitteeList}
%    \end{macrocode}
%
%   Define user interface for adding a person to the committee list.
%    \begin{macrocode}
\DeclareDocumentCommand \CommitteeMember { m m m } {
    \int_gincr:N \g__UWMad_ThesisInfo_CommitteeCount_int
    \UWMad_ListOf_PushEntry:nn {CommitteeList} {
        \tex_hbox:D{}
        \skip_horizontal:n{2em}
        #1
        \g__UWMad_ThesisInfo_Committee_InfoSeparator_tl{}
        \
        \textsl{#2}
        \g__UWMad_ThesisInfo_Committee_InfoSeparator_tl{}
        \
        \textsl{#3}
        \skip_vertical:n{-1em}
    }
}
%    \end{macrocode}
%
%   Define an author interface for printing the Committee member list.
%    \begin{macrocode}
\DeclareDocumentCommand \PrintCommitteeMemberList { } {
    {
    \setstretch{1}
    \bool_if:NTF \g__UWMad_ThesisInfo_IsSet_Advisor_bool {
        \tex_hbox:D{}
        \skip_horizontal:n{2em}
        \g__UWMad_ThesisInfo_AdvisorName_tl{}
        \g__UWMad_ThesisInfo_Committee_InfoSeparator_tl{}
        \
        \textsl{\g__UWMad_ThesisInfo_AdvisorPosition_tl{}}
        \g__UWMad_ThesisInfo_Committee_InfoSeparator_tl{}
        \
        \textsl{\g__UWMad_ThesisInfo_AdvisorAssociation_tl{}}
        \
        (\g__UWMad_ThesisInfo_AdvisorMarker_tl{})
        \skip_vertical:n{-1em}
    } { }
    \UWMad_ListOf_PrintEntries:n {CommitteeList}
    }
}
%    \end{macrocode}
%
%
%
%
%
%   \UWSubModule{PDF Metadata}
%
%
%
%   Define metadata internals.
%    \begin{macrocode}
\tl_new:N \g__UWMad_PDFMetaData_Subject_tl
\tl_new:N \g__UWMad_PDFMetaData_Keywords_tl
\tl_new:N \g__UWMad_PDFMetaData_Producer_tl
\tl_new:N \g__UWMad_PDFMetaData_Creator_tl
%    \end{macrocode}
%
%   Define user interface for setting metadata.
%    \begin{macrocode}
\DeclareDocumentCommand \Subject { m } {
    \tl_gset:Nn \g__UWMad_PDFMetaData_Subject_tl {#1}
    \UWMad_MetaData_PushToList:nn{pdfsubject}  {#1}
    \bool_if:NTF \g__UWMad_MetaData_IsDocument_bool {
        \bool_gset_true:N \g__UWMad_MetaData_GenerateAux_bool
    } { }
}
\DeclareDocumentCommand \Keywords { m } {
    \tl_gset:Nn \g__UWMad_PDFMetaData_Keywords_tl {#1}
    \UWMad_MetaData_PushToList:nn{pdfkeywords} {#1}
    \bool_if:NTF \g__UWMad_MetaData_IsDocument_bool {
        \bool_gset_true:N \g__UWMad_MetaData_GenerateAux_bool
    } { }
}
\DeclareDocumentCommand \Producer { m } {
    \tl_gset:Nn \g__UWMad_PDFMetaData_Producer_tl {#1}
    \UWMad_MetaData_PushToList:nn{pdfproducer}  {#1}
    \bool_if:NTF \g__UWMad_MetaData_IsDocument_bool {
        \bool_gset_true:N \g__UWMad_MetaData_GenerateAux_bool
    } { }
}
\DeclareDocumentCommand \Creator { m } {
    \tl_gset:Nn \g__UWMad_PDFMetaData_Creator_tl {#1}
    \UWMad_MetaData_PushToList:nn{pdfcreator} {#1}
    \bool_if:NTF \g__UWMad_MetaData_IsDocument_bool {
        \bool_gset_true:N \g__UWMad_MetaData_GenerateAux_bool
    } { }
}
%    \end{macrocode}
%
%   Define user interface for accessing metadata.
%    \begin{macrocode}
\DeclareDocumentCommand \TheSubject { } {
    \g__UWMad_PDFMetaData_Subject_tl
}
\DeclareDocumentCommand \TheKeywords { } {
    \g__UWMad_PDFMetaData_Keywords_tl
}
\DeclareDocumentCommand \TheProducer { } {
    \g__UWMad_PDFMetaData_Producer_tl
}
\DeclareDocumentCommand \TheCreator { } {
    \g__UWMad_PDFMetaData_Creator_tl
}
%
%    \end{macrocode}
%
%
%
%
%   \iffalse
%</Code>
%   \fi
%   \iffalse
%<*Code>
%   \fi
%
%
%^^A ====================================================================== %
%^^A                            Title Page                                  %
%^^A ====================================================================== %
%
%   \UWModule{Special Pages}
%
%   \UWSubModule{MakeTitlePage}\label{Imp/Special Pages/MakeTitlePage}
%
%    \begin{macrocode}
% That phrase that occurs on every title page design the class author has seen
\DeclareDocumentCommand \FulfillmentClause { } {
    {
    \setstretch{1.1}
    A~\TheDocumentType{}~submitted~in~partial~fulfillment~of~the~
    requirements~for~the~degree~of
    }
}

\DeclareDocumentCommand \TitlePageTitle { } {
    \IfInfoIsSetT {Title} {
        {
            \LARGE
            \textsc {\TheTitle}
        }
    }
}

\DeclareDocumentCommand \TitlePageAuthor { } {
    \IfInfoIsSetT {Author} {
        {
            \large
            by  \\[0.50em]
            \TheAuthor{}
        }
    }
}

\DeclareDocumentCommand \TitlePageFulFillment { } {
    \FulfillmentClause{}
}

\DeclareDocumentCommand \TitlePageDegree { } {
    \IfInfoIsSetT {Degree} {
        (\TheDegree{})
    }
}

\DeclareDocumentCommand \TitlePageProgram { } {
    \IfInfoIsSetT {Program} {
        \TheProgram{}
    }
}

\DeclareDocumentCommand \TitlePageInstitution { } {
    \IfInfoIsSetT {Institution} {
        at~the                      \\[0.50em]
        \textsc{\TheInstitution{}}  \\[0.50em]
         \the\year
    }
}

\DeclareDocumentCommand \TitlePageDefenseDate { } {
    \IfInfoIsSetT {DefenseDate} {
        Date~of~final~oral~examination:~\TheDefenseDate{}
    }
}


\DeclareDocumentCommand \MakeTitlePage { } {
    \clearpage
    \thispagestyle{empty}
    \begin{center}
        \TitlePageTitle{}       \\[1.0em]
        \TitlePageAuthor{}      \\[1.0em]
        \vfill
        \TitlePageFulFillment{} \\[1.0em]
        \TitlePageDegree{}      \\[1.0em]
        \TitlePageProgram{}     \\[1.0em]
        \vfill
        \TitlePageInstitution{}
        \vfill
    \end{center}
    \TitlePageDefenseDate{}\\[1.0em]
    \PrintCommitteeMemberList{}
    \cleardoublepage
}





%    \end{macrocode}
%   \UWSubModule{LicensePage}
%   First, the support code for defining \cs{Copyright} and
%   \cs{CreativeCommons} will be given.
%   Then the user front-end will be given through the |LicensePage|
%   environment.
%
%    \begin{macrocode}
\cs_new:Nn \__UWMad_LicensePage_StartPage: {
    \clearpage
    \thispagestyle{empty}
    \tex_hbox:D{}
    \tex_vfill:D
    \phantomsection
}
%
%    \end{macrocode}
%   \UWSubSubModule{Copyright}
%    \begin{macrocode}
\bool_new:N   \l__UWMad_Copyright_UseCopyright_bool
\cs_set_eq:NN \CopyrightSymbol \copyright

\cs_set:Nn \__UWMad_Copyright_LicenseText: {
    \begin{center}
        Copyright~\CopyrightSymbol{}~
        \l__UWMad_LicensePage_Year_tl{}~
        by~
        \l__UWMad_LicensePage_Owner_tl{}
    \end{center}
}
%
%
%
%
%    \end{macrocode}
%
%
%
%
%^^A ====================================================================== %
%^^A                       Creative Commons Licenses                        %
%^^A ====================================================================== %
%
%   \UWSubSubModule{Creative Commons}
%    \begin{macrocode}
%   Token lists
\tl_new:N    \l__UWMad_CCLicense_Porting_tl
\tl_new:N    \l__UWMad_CCLicense_Version_tl
\tl_new:N    \l__UWMad_CCLicense_TypeAbbreviation_tl
\tl_new:N    \l__UWMad_CCLicense_TypeWords_tl
\tl_new:N    \l__UWMad_CCLicense_URL_Front_tl
\tl_new:N    \l__UWMad_CCLicense_URL_Middle_tl
\tl_new:N    \l__UWMad_CCLicense_URL_Back_tl
\tl_new:N    \l__UWMad_CCLicense_URL_tl
\tl_new:N    \l__UWMad_CCLicense_http_tl
\tl_new:N    \l__UWMad_CCLicense_URLText_tl
%
%   Booleans
\bool_new:N \l__UWMad_CCLicense_UseCreativeCommons_bool
\bool_new:N \l__UWMad_CCLicense_UseAttribution_bool
\bool_new:N \l__UWMad_CCLicense_UseShareAlike_bool
\bool_new:N \l__UWMad_CCLicense_UseNoDerivatives_bool
\bool_new:N \l__UWMad_CCLicense_UseNonCommercial_bool
\bool_new:N \l__UWMad_CCLicense_IsValid_bool
\bool_set_true:N \l__UWMad_CCLicense_UseAttribution_bool
%
%   Valid license types
\cs_new:cn {l__UWMad_CCLicense_Valid_ by :}      {}
\cs_new:cn {l__UWMad_CCLicense_Valid_ by-sa :}   {}
\cs_new:cn {l__UWMad_CCLicense_Valid_ by-nd :}   {}
\cs_new:cn {l__UWMad_CCLicense_Valid_ by-nc :}   {}
\cs_new:cn {l__UWMad_CCLicense_Valid_ by-nc-sa :}{}
\cs_new:cn {l__UWMad_CCLicense_Valid_ by-nc-nd :}{}
%
%   Defaults
\tl_gset:Nn \l__UWMad_CCLicense_Porting_tl {
    International
}
\tl_gset:Nn \l__UWMad_CCLicense_Version_tl {
    4.0
}
%
%   URL definitions
\tl_set:Nn \l__UWMad_CCLicense_URL_Front_tl {
    creativecommons.org/licenses
}
\tl_set:Nn \l__UWMad_CCLicense_URL_Middle_tl {
    /\l__UWMad_CCLicense_TypeAbbreviation_tl
}
\tl_set:Nn \l__UWMad_CCLicense_URL_Back_tl {
    /\l__UWMad_CCLicense_Version_tl
}
\tl_set:Nn \l__UWMad_CCLicense_URL_tl {
    http://
    \l__UWMad_CCLicense_URL_Front_tl
    \l__UWMad_CCLicense_URL_Middle_tl
    \l__UWMad_CCLicense_URL_Back_tl
}
\tl_set:Nn \l__UWMad_CCLicense_http_tl {
    http://
}
%
%
\tl_set:Nn \l__UWMad_CCLicense_URLText_tl {
    Creative~Commons~
    \l__UWMad_CCLicense_TypeWords_tl{}~
    \l__UWMad_CCLicense_Version_tl{}~
    \l__UWMad_CCLicense_Porting_tl{}
}
%
%
%
%   Type Creator
\cs_new:Nn \__UWMad_CCLicense_CreateType: {

        \bool_if:NTF \l__UWMad_CCLicense_UseAttribution_bool {

            \tl_put_right:Nn \l__UWMad_CCLicense_TypeAbbreviation_tl {
                by
            }
            \tl_put_right:Nn \l__UWMad_CCLicense_TypeWords_tl {
                Attribution
            }

        } { }

        \bool_if:NTF \l__UWMad_CCLicense_UseNonCommercial_bool {

            \tl_put_right:Nn \l__UWMad_CCLicense_TypeAbbreviation_tl {
                -nc
            }
            \tl_put_right:Nn \l__UWMad_CCLicense_TypeWords_tl {
                -NonCommercial
            }

        } { }

        \bool_if:NTF \l__UWMad_CCLicense_UseShareAlike_bool {

            \tl_put_right:Nn \l__UWMad_CCLicense_TypeAbbreviation_tl {
                -sa
            }
            \tl_put_right:Nn \l__UWMad_CCLicense_TypeWords_tl {
                -ShareAlike
            }

        } { }

        \bool_if:NTF \l__UWMad_CCLicense_UseNoDerivatives_bool {

            \tl_put_right:Nn \l__UWMad_CCLicense_TypeAbbreviation_tl {
                -nd
            }
            \tl_put_right:Nn \l__UWMad_CCLicense_TypeWords_tl {
                -NoDerivatives
            }

        } { }
}
%
%
%
%   Type Validator
\cs_new:Nn \__UWMad_CCLicense_CheckTypeValidity: {
    \cs_if_exist:cTF {
        l__UWMad_CCLicense_Valid_
        \l__UWMad_CCLicense_TypeAbbreviation_tl :
    } {

        \bool_set_true:N \l__UWMad_CCLicense_IsValid_bool

    } {

        \msg_new:nnn {UWMadThesis} {CCLicense / InvalidLicenseType} {
            The~license~type~`\l__UWMad_CCLicense_TypeAbbreviation_tl'~
            is~not~a~valid~Creative~Commons~license.
        }
        \msg_error:nn {UWMadThesis} {CCLicense / InvalidLicenseType}

    }
}
%
%
%
%   Page Printer
\cs_new:Nn \__UWMad_CCLicense_LicenseText: {
    \begin{center}
        \setstretch{1.05}
        This~work~is~released~under~a~
        \href {\l__UWMad_CCLicense_URL_tl} {
            \l__UWMad_CCLicense_URLText_tl
        }~
        license.\\[0.1em]
        \l__UWMad_LicensePage_Owner_tl{},~
        \l__UWMad_LicensePage_Year_tl{}
    \end{center}
}
%
%    \end{macrocode}
%^^A ====================================================================== %
%^^A                            License Page                                %
%^^A ====================================================================== %
%
%   \UWSubSubModule{LicensePage Proper}
%
%
%    \begin{macrocode}
%
    \tl_new:N  \l__UWMad_LicensePage_Year_tl
    \tl_new:N  \l__UWMad_LicensePage_Owner_tl
%
    \tl_set:Nn \l__UWMad_LicensePage_Owner_tl {
        \g__UWMad_ThesisInfo_Author_tl
    }
    \tl_set:Nn \l__UWMad_LicensePage_Year_tl {
        \the\year
    }
%
%
%
\DeclareDocumentEnvironment {LicensePage} { } {
%
%
%
    \DeclareDocumentCommand \LicenseOwner { m } {
        \tl_set:Nn \l__UWMad_LicensePage_Owner_tl {
            ##1
        }
    }
    \DeclareDocumentCommand \TheLicenseOwner { } {
        \l__UWMad_LicensePage_Owner_tl
    }
%
    \DeclareDocumentCommand \LicenseYear { m } {
        \tl_set:Nn \l__UWMad_LicensePage_Year_tl {
            ##1
        }
    }
    \DeclareDocumentCommand \TheLicenseYear { } {
        \l__UWMad_LicensePage_Year_tl
    }
%
%
%
\DeclareDocumentCommand \Copyright { } {
    \bool_set_true:N \l__UWMad_Copyright_UseCopyright_bool
}
\cs_set_eq:NN \AllRightsReserved \Copyright
%
%
%
%   User front ends
\DeclareDocumentCommand \CreativeCommons { } {
    \bool_set_true:N \l__UWMad_CCLicense_UseCreativeCommons_bool
}
\DeclareDocumentCommand \Attribution { } {
    \bool_set_true:N \l__UWMad_CCLicense_UseAttribution_bool
}
\DeclareDocumentCommand \NonCommercial { } {
    \bool_set_true:N \l__UWMad_CCLicense_UseNonCommercial_bool
}
\DeclareDocumentCommand \ShareAlike { } {
    \bool_set_true:N \l__UWMad_CCLicense_UseShareAlike_bool
}
\DeclareDocumentCommand \NoDerivs { } {
    \bool_set_true:N \l__UWMad_CCLicense_UseNoDerivatives_bool
}
%
%
\DeclareDocumentCommand \CCVersion { m } {
    \tl_set:Nn \l__UWMad_CCLicense_Version_tl {##1}
}
%
\DeclareDocumentCommand \CCPorting { m } {
    \tl_set:Nn \l__UWMad_CCLicense_Porting_tl {##1}
}
%
\DeclareDocumentCommand \CCURL { m } {
    \tl_set:Nn \l__UWMad_CCLicense_URL_Front_tl  {##1}
    \tl_set:Nn \l__UWMad_CCLicense_URL_Middle_tl {/.}
    \tl_set:Nn \l__UWMad_CCLicense_URL_Back_tl   {}
}
%
\DeclareDocumentCommand \CCURLText { m } {
    \tl_set:Nn \l__UWMad_CCLicense_URLText_tl {##1}
}
%
%
} {

    \bool_if:nTF {
        \l__UWMad_CCLicense_UseCreativeCommons_bool &&
        \l__UWMad_Copyright_UseCopyright_bool
    } {
        \msg_new:nnn { UWMadThesis } { SpecialPages / MultipleLicenses } {
            Both~Creative~Commons~and~Copyright~have~been~declared.~
            Please,~pick~one.
        }
        \msg_error:nn { UWMadThesis } { SpecialPages / MultipleLicenses }
    } { }



    \bool_if:NTF \l__UWMad_CCLicense_UseCreativeCommons_bool {

        \__UWMad_CCLicense_CreateType:
        \__UWMad_CCLicense_CheckTypeValidity:
        \bool_if:NTF \l__UWMad_CCLicense_IsValid_bool {
            \cs_new_eq:NN
                \__UWMad_LicensePage_LicenseText:
                \__UWMad_CCLicense_LicenseText:
        } { }

    } { }



    \bool_if:NTF \l__UWMad_Copyright_UseCopyright_bool {
        \cs_new_eq:NN
            \__UWMad_LicensePage_LicenseText:
            \__UWMad_Copyright_LicenseText:
    } { }



    \cs_if_exist:NTF \__UWMad_LicensePage_LicenseText: {
        \__UWMad_LicensePage_StartPage:
        \vbox_to_ht:nn {0.3333\textheight} {
            \__UWMad_LicensePage_LicenseText:
        }
    } { }


}
%
%    \end{macrocode}
%
%   \iffalse
%</Code>
%   \fi
%   \iffalse
%<*Code>
%   \fi
%
% \UWModule{Relative Directory Input}
%
%
%   \UWSubModule{Declarations and Initializations}
%   Variable declarations and default initializations for Chapter directories.
%    \begin{macrocode}
\int_new:N  \g__UWMad_RelativeDirectory_Chapter_Count_int
\tl_new:N   \g__UWMad_RelativeDirectory_Chapter_Prefix_tl
\tl_new:N   \g__UWMad_RelativeDirectory_Chapter_Suffix_tl
\tl_new:N   \g__UWMad_RelativeDirectory_Chapter_CurrentPath_tl
\tl_new:N   \g__UWMad_RelativeDirectory_Chapter_CurrentName_tl
\tl_new:N   \g__UWMad_RelativeDirectory_Chapter_ParentPath_tl
\tl_gset:Nn \g__UWMad_RelativeDirectory_Chapter_ParentPath_tl {}
%    \end{macrocode}
%
%   Variable declarations and default initializations for Section directories.
%    \begin{macrocode}
\int_new:N  \g__UWMad_RelativeDirectory_Section_Count_int
\tl_new:N   \g__UWMad_RelativeDirectory_Section_Prefix_tl
\tl_new:N   \g__UWMad_RelativeDirectory_Section_Suffix_tl
\tl_new:N   \g__UWMad_RelativeDirectory_Section_CurrentPath_tl
\tl_new:N   \g__UWMad_RelativeDirectory_Section_CurrentName_tl
\tl_new:N   \g__UWMad_RelativeDirectory_Section_ParentPath_tl
\tl_gset:Nn \g__UWMad_RelativeDirectory_Section_ParentPath_tl {
    \g__UWMad_RelativeDirectory_Chapter_CurrentPath_tl/
}
%    \end{macrocode}
%
%   Variable declarations and default initializations for Subsection
%   directories.
%    \begin{macrocode}
\int_new:N  \g__UWMad_RelativeDirectory_Subsection_Count_int
\tl_new:N   \g__UWMad_RelativeDirectory_Subsection_Prefix_tl
\tl_new:N   \g__UWMad_RelativeDirectory_Subsection_Suffix_tl
\tl_new:N   \g__UWMad_RelativeDirectory_Subsection_CurrentPath_tl
\tl_new:N   \g__UWMad_RelativeDirectory_Subsection_CurrentName_tl
\tl_new:N   \g__UWMad_RelativeDirectory_Subsection_ParentPath_tl
\tl_gset:Nn \g__UWMad_RelativeDirectory_Subsection_ParentPath_tl {
    \g__UWMad_RelativeDirectory_Section_CurrentPath_tl/
}
%    \end{macrocode}
%
%   Variable declaration for graphics inclusion
%    \begin{macrocode}
\tl_new:N   \g__UWMad_RelativeDirectory_Graphics_DirectoryName_tl
\tl_new:N   \g__UWMad_RelativeDirectory_Graphics_Extension_tl
\tl_new:N   \g__UWMad_RelativeDirectory_Graphics_BaseName_tl
%    \end{macrocode}
%
%   Variable declarations for search options.
%    \begin{macrocode}
\bool_new:N \g__UWMad_RelativeDirectory_CycleThrough_Graphics_bool
\bool_new:N \g__UWMad_RelativeDirectory_CycleThrough_Files_bool
%    \end{macrocode}
%
%   Miscellaneous variable initializations for the system
%    \begin{macrocode}
\tl_new:N    \g__UWMad_RelativeDirectory_File_CurrentName_tl
\tl_new:N    \g__UWMad_RelativeDirectory_OptionalPath_tl
\seq_new:N   \g__UWMad_RelativeDirectory_PathStack_Files_seq
\seq_new:N   \g__UWMad_RelativeDirectory_PathStack_Graphics_seq
\bool_new:N  \g__UWMad_RelativeDirectory_IsFileFound_bool
%    \end{macrocode}
%
%
%
%   Miscellaneous control sequence initializations for the system.
%    \begin{macrocode}
\cs_new:Nn \UWMad_RelativeDirectory_Chapter_SetName:    {}
\cs_new:Nn \UWMad_RelativeDirectory_Section_SetName:    {}
\cs_new:Nn \UWMad_RelativeDirectory_Subsection_SetName: {}
%    \end{macrocode}
%
%
%   \UWSubModule{Back End Code}
%   All of the underlying \LaTeXPL{} code for this module is in this section.
%
%
%   \UWSubSubModule{File Inclusion}
%
%   Special hooks for the automatic naming function below.
%    \begin{macrocode}
\cs_new:Nn \UWMad_RelativeDirectory_SetName_Increment_Hook_Chapter: {
    \int_gset:cn {g__UWMad_RelativeDirectory_Section_Count_int}{0}
    \int_gset:cn {g__UWMad_RelativeDirectory_Subsection_Count_int}{0}
}
\cs_new:Nn \UWMad_RelativeDirectory_SetName_Increment_Hook_Section: {
    \int_gset:cn {g__UWMad_RelativeDirectory_Subsection_Count_int}{0}
}
\cs_new:Nn \UWMad_RelativeDirectory_SetName_Increment_Hook_Subsection: {}
%    \end{macrocode}
%
%   Directory name-setting functions.
%    \begin{macrocode}
\cs_new:Nn \UWMad_RelativeDirectory_SetName_None:n {
    \tl_gset:cx   {g__UWMad_RelativeDirectory_ #1 _CurrentName_tl} {
        \tl_use:c {g__UWMad_RelativeDirectory_ #1 _Prefix_tl}
        \tl_use:c {g__UWMad_RelativeDirectory_ #1 _Suffix_tl}
    }
}
\cs_new:Nn \UWMad_RelativeDirectory_SetName_Increment:n {
    \use:c{UWMad_RelativeDirectory_SetName_Increment_Hook_ #1 :}
    \int_gincr:c  {g__UWMad_RelativeDirectory_ #1 _Count_int}
    \tl_gset:cx   {g__UWMad_RelativeDirectory_ #1 _CurrentName_tl} {
        \tl_use:c {g__UWMad_RelativeDirectory_ #1 _Prefix_tl}
        \int_to_arabic:n{
            \int_use:c{g__UWMad_RelativeDirectory_ #1 _Count_int}
        }
        \tl_use:c {g__UWMad_RelativeDirectory_ #1 _Suffix_tl}
    }
}
\cs_new:Nn \UWMad_RelativeDirectory_SetName_Same:n {
    \tl_gset:cx   {g__UWMad_RelativeDirectory_ #1 _CurrentName_tl} {
        \tl_use:c {g__UWMad_RelativeDirectory_ #1 _Prefix_tl}
        \g__UWMad_RelativeDirectory_File_CurrentName_tl
        \tl_use:c {g__UWMad_RelativeDirectory_ #1 _Suffix_tl}
    }
}
%    \end{macrocode}
%
%
%   Name and path setter.
%    \begin{macrocode}
\cs_new:Nn \UWMad_RelativeDirectory_SetNameAndPath:n {

    \tl_gclear:c {g__UWMad_RelativeDirectory_ #1 _CurrentName_tl}
    \tl_gclear:c {g__UWMad_RelativeDirectory_ #1 _CurrentPath_tl}

    \tl_if_blank:VTF {\g__UWMad_RelativeDirectory_OptionalPath_tl} {
        \use:c {UWMad_RelativeDirectory_ #1 _SetName:}
    } {
        \tl_gset_eq:cN
            {g__UWMad_RelativeDirectory_ #1 _CurrentName_tl}
            \g__UWMad_RelativeDirectory_OptionalPath_tl
    }
    \tl_gset:cx   {g__UWMad_RelativeDirectory_ #1 _CurrentPath_tl} {
        \tl_use:c {g__UWMad_RelativeDirectory_ #1 _ParentPath_tl}
        \tl_use:c {g__UWMad_RelativeDirectory_ #1 _CurrentName_tl}
    }
}
%    \end{macrocode}
%
%
%   The default push function pushes to both the file and graphics stacks.
%   However, if the user defines a single (the only) graphics folder, a
%   files-only push function is also defined that will be used when that
%   option is set.
%    \begin{macrocode}
\cs_new:Nn \__UWMad_RelativeDirectory_StackPush_Default:n {
    \tl_gset_eq:Nc
        \g_tmpa_tl
        {g__UWMad_RelativeDirectory_ #1 _CurrentName_tl}
    \tl_if_blank:VTF {\g_tmpa_tl} { } {
        \seq_gpush:Nx \g__UWMad_RelativeDirectory_PathStack_Files_seq {
            \tl_use:c {g__UWMad_RelativeDirectory_ #1 _CurrentPath_tl}
        }
        \seq_gpush:Nx \g__UWMad_RelativeDirectory_PathStack_Graphics_seq {
            \tl_use:c {g__UWMad_RelativeDirectory_ #1 _CurrentPath_tl}
        }
    }
}
\cs_new:Nn \__UWMad_RelativeDirectory_StackPush_Files:n {
    \tl_gset_eq:Nc
        \g_tmpa_tl
        {g__UWMad_RelativeDirectory_ #1 _CurrentName_tl}
    \tl_if_blank:VTF {\g_tmpa_tl} { } {
        \seq_gpush:Nx \g__UWMad_RelativeDirectory_PathStack_Files_seq {
            \tl_use:c {g__UWMad_RelativeDirectory_ #1 _CurrentPath_tl}
        }
    }
}
%    \end{macrocode}
%
%
%   The default push function uses the default function above.
%   If the user sets a graphics directory name (in which there may be multiple
%   graphics directories in all subdirectories), this will be re-defined.
%    \begin{macrocode}
\cs_new:Nn \__UWMad_RelativeDirectory_StackPush:n {
    \__UWMad_RelativeDirectory_StackPush_Default:n{#1}
}
%    \end{macrocode}
%
%
%   Pre-stack update functions for the supported sections.
%    \begin{macrocode}
\cs_new:Nn \UWMad_RelativeDirectory_UpdateStack_Chapter_PreHook: {
    \seq_gclear:N \g__UWMad_RelativeDirectory_PathStack_Files_seq
    \seq_gclear:N \g__UWMad_RelativeDirectory_PathStack_Graphics_seq
}
\cs_new:Nn \UWMad_RelativeDirectory_UpdateStack_Section_PreHook: {}
\cs_new:Nn \UWMad_RelativeDirectory_UpdateStack_Subsection_PreHook: {}
%    \end{macrocode}
%   This function updates the current name, path, and stack(s).
%   Chapters inclusions always clear the stacks.
%    \begin{macrocode}
\cs_new:Nn \UWMad_RelativeDirectory_UpdateStack:n {
    \use:c {UWMad_RelativeDirectory_UpdateStack_ #1 _PreHook:}
    \UWMad_RelativeDirectory_SetNameAndPath:n{#1}
    \__UWMad_RelativeDirectory_StackPush:n{#1}
}
%    \end{macrocode}
%
%
%   Two file inputers: one cycles through the current path stack searching
%   for the file from deepest to highest and the other only searches the
%   deepest (i.e., current) directory.
%    \begin{macrocode}
\cs_new:Nn \UWMad_RelativeDirectory_IncludeFile_CycleThrough: {
    \seq_map_inline:Nn \g__UWMad_RelativeDirectory_PathStack_Files_seq {
        \tl_gset:Nx \g_tmpa_tl {
            ./##1/
            \g__UWMad_RelativeDirectory_File_CurrentName_tl
        }
        \bool_if:NTF \g__UWMad_RelativeDirectory_IsFileFound_bool { } {
            \file_if_exist:nTF { \g_tmpa_tl } {
                \file_input:n{ \g_tmpa_tl }
                \bool_gset_true:N \g__UWMad_RelativeDirectory_IsFileFound_bool
                \seq_map_break:
            } { }
        }
    }
}
\cs_new:Nn \UWMad_RelativeDirectory_IncludeFile_CheckDeepest: {

    \seq_get:NN
        \g__UWMad_RelativeDirectory_PathStack_Files_seq
        \g_tmpa_tl
    \tl_gset:Nx \g_tmpa_tl {
            ./\g_tmpa_tl/
            \g__UWMad_RelativeDirectory_File_CurrentName_tl
    }
    \file_if_exist:nTF {\g_tmpa_tl} {
        \file_input:n{\g_tmpa_tl}
        \bool_gset_true:N \g__UWMad_RelativeDirectory_IsFileFound_bool
    } { }
}
%    \end{macrocode}
%
%   This is a wrapper function for the above two functions with two additional
%   behaviors: if the file is not found from the search stack, it will check
%   the topmost \TeX{} directory for the file and issue a warning if it is not
%   found.
%    \begin{macrocode}
\msg_new:nnn { UWMadThesis }{ RelativeDirectory / FileNotFound } {
    The~requested~file~'#1'~was~not~found~in~the~current~search~stack~nor~the~
    main~LaTeX~directory~for~the~job~'\c_job_name_tl'.
}
\cs_new:Nn \UWMad_RelativeDirectory_IncludeFile: {
    \bool_gset_false:N \g__UWMad_RelativeDirectory_IsFileFound_bool

    \bool_if:NTF \g__UWMad_RelativeDirectory_CycleThrough_Files_bool {
        \UWMad_RelativeDirectory_IncludeFile_CycleThrough:
    } {
        \UWMad_RelativeDirectory_IncludeFile_CheckDeepest:
    }
    \bool_if:NTF \g__UWMad_RelativeDirectory_IsFileFound_bool { } {
        \file_if_exist:nTF {\g__UWMad_RelativeDirectory_File_CurrentName_tl} {
            \file_input:n{ \g__UWMad_RelativeDirectory_File_CurrentName_tl }
            \bool_gset_true:N \g__UWMad_RelativeDirectory_IsFileFound_bool
        } {
            \msg_warning:nnx
                { UWMadThesis }
                { RelativeDirectory / FileNotFound }
                { \g__UWMad_RelativeDirectory_File_CurrentName_tl }
        }
    }
}
%    \end{macrocode}
%
%
%
%
%
%
%
%
%
%
%
%
%   \UWSubSubModule{Graphics Inclusion}
%
%   This code copies the existing \cs{includegraphics} command such that it
%   can be used in a compatible way with the \LaTeXe{} system.
%   This technically breaks the \LaTeXPL{} naming convention since an |n|
%   argument specifier is not a for double square braces, but it is deemed
%   good enough.
%    \begin{macrocode}
\cs_new_eq:NN
    \__UWMad_RelativeDirectory_IncludeGraphics_Original:nn
    \includegraphics
\cs_undefine:N
    \includegraphics
%    \end{macrocode}
%
%   This function defines the push procedure when a graphics directory name is
%   given.
%   This function will replace the default stack push if the user defines
%   a graphics directory.
%    \begin{macrocode}
\cs_new:Nn \__UWMad_RelativeDirectory_StackPush_FilesAndGraphics:n {
    \tl_gset_eq:Nc
        \g_tmpa_tl
        {g__UWMad_RelativeDirectory_ #1 _CurrentName_tl}
    \tl_if_blank:VTF {\g_tmpa_tl} { } {
        \seq_gpush:Nx \g__UWMad_RelativeDirectory_PathStack_Files_seq {
            \tl_use:c {g__UWMad_RelativeDirectory_ #1 _CurrentPath_tl}
        }
        \seq_gpush:Nx \g__UWMad_RelativeDirectory_PathStack_Graphics_seq {
            \tl_use:c {g__UWMad_RelativeDirectory_ #1 _CurrentPath_tl}
        }
        \seq_gpush:Nx \g__UWMad_RelativeDirectory_PathStack_Graphics_seq {
            \tl_use:c {g__UWMad_RelativeDirectory_ #1 _CurrentPath_tl}/
            \g__UWMad_RelativeDirectory_Graphics_DirectoryName_tl
        }
    }
}
%    \end{macrocode}
%
%
%   Two graphics includers: one cycles through the current path stack searching
%   for the file from deepest to highest and the other only searches the
%   deepest (i.e., current graphic's) directory.
%    \begin{macrocode}
\cs_new:Nn \UWMad_RelativeDirectory_IncludeGraphics_CycleThrough:n {

    \UWMad_File_GetExtension:nNN
        {\g__UWMad_RelativeDirectory_File_CurrentName_tl}
        \g__UWMad_RelativeDirectory_Graphics_BaseName_tl
        \g__UWMad_RelativeDirectory_Graphics_Extension_tl

    \seq_map_inline:Nn \g__UWMad_RelativeDirectory_PathStack_Graphics_seq {

        \tl_gset:Nx \g_tmpa_tl {
            ./##1/
            \g__UWMad_RelativeDirectory_File_CurrentName_tl
        }

        \bool_if:NTF \g__UWMad_RelativeDirectory_IsFileFound_bool { } {
            \file_if_exist:nTF { \g_tmpa_tl } {
                \tl_gset:Nx \g_tmpa_tl {
                    ./##1/
                    \g__UWMad_RelativeDirectory_Graphics_BaseName_tl
                }
                \__UWMad_RelativeDirectory_IncludeGraphics_Original:nn
                    [ #1 ] {\g_tmpa_tl}
                \bool_gset_true:N \g__UWMad_RelativeDirectory_IsFileFound_bool
                \seq_map_break:
            } { }
        }
    }
}
\cs_new:Nn \UWMad_RelativeDirectory_IncludeGraphics_CheckDeepest:n {

    \seq_get:NN
        \g__UWMad_RelativeDirectory_PathStack_Graphics_seq
        \g_tmpa_tl

    \tl_gset:Nx \g_tmpb_tl {
            ./\g_tmpa_tl/
            \g__UWMad_RelativeDirectory_File_CurrentName_tl
    }

    \UWMad_File_GetExtension:nNN {\g_tmpb_tl}
        \g__UWMad_RelativeDirectory_Graphics_BaseName_tl
        \g__UWMad_RelativeDirectory_Graphics_Extension_tl

    \file_if_exist:nTF { \g_tmpb_tl } {
        \__UWMad_RelativeDirectory_IncludeGraphics_Original:nn
            [ #1 ]
            { \g__UWMad_RelativeDirectory_Graphics_BaseName_tl }
        \bool_gset_true:N \g__UWMad_RelativeDirectory_IsFileFound_bool
    } { }
}
%    \end{macrocode}
%
%   This is a wrapper function for the above two functions with two additional
%   behaviors: if the graphic is not found from the search stack, it will check
%   the topmost \TeX{} directory  and issue a warning if it is still not
%   found.
%    \begin{macrocode}
\msg_new:nnn { UWMadThesis }{ RelativeDirectory / GraphicNotFound } {
    The~requested~graphic~'#1'~was~not~found~in~the~current~search~stack~nor~
    the~main~LaTeX~directory~for~the~job~'\c_job_name_tl'.
}
\cs_new:Nn \UWMad_RelativeDirectory_IncludeGraphics:n {
    \bool_gset_false:N \g__UWMad_RelativeDirectory_IsFileFound_bool
    \bool_if:NTF \g__UWMad_RelativeDirectory_CycleThrough_Graphics_bool {
        \UWMad_RelativeDirectory_IncludeGraphics_CycleThrough:n{#1}
    } {
        \UWMad_RelativeDirectory_IncludeGraphics_CheckDeepest:n{#1}
    }
    \bool_if:NTF \g__UWMad_RelativeDirectory_IsFileFound_bool { } {
        \file_if_exist:nTF {\g__UWMad_RelativeDirectory_File_CurrentName_tl} {
            \__UWMad_RelativeDirectory_IncludeGraphics_Original:nn
            [ #1 ]
            { \g__UWMad_RelativeDirectory_File_CurrentName_tl }
            \bool_gset_true:N \g__UWMad_RelativeDirectory_IsFileFound_bool
        } {
            \msg_warning:nnx
                { UWMadThesis }
                { RelativeDirectory / GraphicNotFound }
                { \g__UWMad_RelativeDirectory_File_CurrentName_tl }
        }
    }
}
%    \end{macrocode}
%
%
%
%
%
%
%
%
%   \UWSubSubModule{Key-Value Option Definitions}
%
%
%
%   Being the key definitions
%    \begin{macrocode}
\keys_define:nn { UWMadThesis / RelativeDirectory } {
%    \end{macrocode}
%
%
%
%   Chapter prefix and suffix keys.
%    \begin{macrocode}
    chapter-directory-prefix    .tl_gset:N =
        \g__UWMad_RelativeDirectory_Chapter_Prefix_tl,
    chapter-directory-prefix    .default:n =,
    chapter-directory-suffix    .tl_gset:N =
        \g__UWMad_RelativeDirectory_Chapter_Suffix_tl,
    chapter-directory-suffix    .default:n =,
%    \end{macrocode}
%
%
%
%   Chapter naming conventions
%    \begin{macrocode}
    chapter-directory-name      .choice:,
    chapter-directory-name / none .code:n = {
        \cs_gset:Nn \UWMad_RelativeDirectory_Chapter_SetName: {
            \UWMad_RelativeDirectory_SetName_None:n{Chapter}
        }
    },
    chapter-directory-name / same .code:n = {
        \cs_gset:Nn \UWMad_RelativeDirectory_Chapter_SetName: {
            \UWMad_RelativeDirectory_SetName_Same:n{Chapter}
        }
    },
    chapter-directory-name / increment .code:n = {
        \cs_gset:Nn \UWMad_RelativeDirectory_Chapter_SetName: {
            \UWMad_RelativeDirectory_SetName_Increment:n{Chapter}
        }
    },
    chapter-directory-name      .default:n = none,
%    \end{macrocode}
%
%
%
%   Section prefix and suffix keys.
%    \begin{macrocode}
    section-directory-prefix    .tl_gset:N =
        \g__UWMad_RelativeDirectory_Section_Prefix_tl,
    section-directory-prefix    .default:n =,
    section-directory-suffix    .tl_gset:N =
        \g__UWMad_RelativeDirectory_Section_Suffix_tl,
    section-directory-suffix    .default:n =,
%    \end{macrocode}
%
%
%
%   Section naming conventions
%    \begin{macrocode}
    section-directory-name      .choice:,
    section-directory-name / none .code:n = {
        \cs_gset:Nn \UWMad_RelativeDirectory_Section_SetName: {
            \UWMad_RelativeDirectory_SetName_None:n{Section}
        }
    },
    section-directory-name / same .code:n = {
        \cs_gset:Nn \UWMad_RelativeDirectory_Section_SetName: {
            \UWMad_RelativeDirectory_SetName_Same:n{Section}
        }
    },
    section-directory-name / increment .code:n = {
        \cs_gset:Nn \UWMad_RelativeDirectory_Section_SetName: {
            \UWMad_RelativeDirectory_SetName_Increment:n{Section}
        }
    },
    section-directory-name      .default:n = none,
%    \end{macrocode}
%
%
%
%   Subsection prefix and suffix keys.
%    \begin{macrocode}
    subsection-directory-prefix    .tl_gset:N =
        \g__UWMad_RelativeDirectory_Subsection_Prefix_tl,
    subsection-directory-prefix    .default:n =,
    subsection-directory-suffix    .tl_gset:N =
        \g__UWMad_RelativeDirectory_Subsection_Suffix_tl,
    subsection-directory-suffix    .default:n =,
%    \end{macrocode}
%
%
%
%   Subsection naming conventions
%    \begin{macrocode}
    subsection-directory-name      .choice:,
    subsection-directory-name / none .code:n = {
        \cs_gset:Nn \UWMad_RelativeDirectory_Subsection_SetName: {
            \UWMad_RelativeDirectory_SetName_None:n{Subsection}
        }
    },
    subsection-directory-name / same .code:n = {
        \cs_gset:Nn \UWMad_RelativeDirectory_Subsection_SetName: {
            \UWMad_RelativeDirectory_SetName_Same:n{Subsection}
        }
    },
    subsection-directory-name / increment .code:n = {
        \cs_gset:Nn \UWMad_RelativeDirectory_Subsection_SetName: {
            \UWMad_RelativeDirectory_SetName_Increment:n{Subsection}
        }
    },
    subsection-directory-name      .default:n = none,
%    \end{macrocode}
%
%
%   Graphics directory keys.
%    \begin{macrocode}
    graphics-directory-name .code:n = {
        \tl_gset:Nn \g__UWMad_RelativeDirectory_Graphics_DirectoryName_tl {
            #1
        }
        \tl_if_blank:nTF { #1 } {
            \cs_gset:Nn \__UWMad_RelativeDirectory_StackPush:n {
                \__UWMad_RelativeDirectory_StackPush_Default:n{##1}
            }
        } {
            \cs_gset:Nn \__UWMad_RelativeDirectory_StackPush:n {
                \__UWMad_RelativeDirectory_StackPush_FilesAndGraphics:n{##1}
            }
        }
    },
    the-only-graphics-directory .code:n = {
        \bool_set_false:N
            \g__UWMad_RelativeDirectory_CycleThrough_Graphics_bool
        \seq_gclear:N \g__UWMad_RelativeDirectory_PathStack_Graphics_seq
        \seq_gpush:Nn \g__UWMad_RelativeDirectory_PathStack_Graphics_seq {
            #1
        }
        \cs_gset:Nn \UWMad_RelativeDirectory_UpdateStack_Chapter_PreHook: {
            \seq_gclear:N \g__UWMad_RelativeDirectory_PathStack_Files_seq
        }
        \cs_gset:Nn \__UWMad_RelativeDirectory_StackPush:n {
            \__UWMad_RelativeDirectory_StackPush_Files:n{##1}
        }
    },
%    \end{macrocode}
%
%   Path search keys.
%    \begin{macrocode}
    cycle-file-paths .bool_gset:N =
        \g__UWMad_RelativeDirectory_CycleThrough_Files_bool,
    cycle-file-paths .default:n = false,
    cycle-graphic-paths .bool_gset:N =
        \g__UWMad_RelativeDirectory_CycleThrough_Graphics_bool,
    cycle-graphic-paths .default:n = true
}
%    \end{macrocode}
%
%   Set the default values for the keys.
%    \begin{macrocode}
\keys_set:nn { UWMadThesis / RelativeDirectory } {
    chapter-directory-prefix,
    chapter-directory-suffix,
    section-directory-prefix,
    section-directory-suffix,
    subsection-directory-prefix,
    subsection-directory-suffix,
    chapter-directory-name,
    section-directory-name,
    subsection-directory-name,
    cycle-file-paths,
    cycle-graphic-paths
}
%    \end{macrocode}
%
%   \UWSubModule{User Front Ends}
%    \begin{macrocode}
\DeclareDocumentCommand \IncludeChapter { o m } {
    \IfValueTF { #1 } {
        \tl_gset:Nn \g__UWMad_RelativeDirectory_OptionalPath_tl {#1}
    } { }
    \tl_gset:Nn \g__UWMad_RelativeDirectory_File_CurrentName_tl {#2}
    \UWMad_RelativeDirectory_UpdateStack:n{Chapter}
    \UWMad_RelativeDirectory_IncludeFile:
    \tl_gclear:N \g__UWMad_RelativeDirectory_OptionalPath_tl
}
\DeclareDocumentCommand \IncludeSection { o m } {
    \IfValueTF { #1 } {
        \tl_gset:Nn \g__UWMad_RelativeDirectory_OptionalPath_tl {#1}
    } { }
    \tl_gset:Nn \g__UWMad_RelativeDirectory_File_CurrentName_tl {#2}
    \UWMad_RelativeDirectory_UpdateStack:n{Section}
    \UWMad_RelativeDirectory_IncludeFile:
    \tl_gclear:N \g__UWMad_RelativeDirectory_OptionalPath_tl
}
\DeclareDocumentCommand \IncludeSubsection { o m } {
    \IfValueTF { #1 } {
        \tl_gset:Nn \g__UWMad_RelativeDirectory_OptionalPath_tl {#1}
    } { }
    \tl_gset:Nn \g__UWMad_RelativeDirectory_File_CurrentName_tl {#2}
    \UWMad_RelativeDirectory_UpdateStack:n{Subsection}
    \UWMad_RelativeDirectory_IncludeFile:
    \tl_gclear:N \g__UWMad_RelativeDirectory_OptionalPath_tl
}
\DeclareDocumentCommand \IncludeGraphics { o m } {
    \tl_gset:Nn \g__UWMad_RelativeDirectory_File_CurrentName_tl {#2}
    \IfValueTF { #1 } {
        \UWMad_RelativeDirectory_IncludeGraphics:n{#1}
    } {
        \UWMad_RelativeDirectory_IncludeGraphics:n{}
    }
}
\cs_new_eq:NN
    \includegraphics
    \IncludeGraphics
%    \end{macrocode}
%
%   \iffalse
%</Code>
%   \fi
%    \begin{macrocode}
\ExplSyntaxOff
%    \end{macrocode}
%   \Finale
\endinput
