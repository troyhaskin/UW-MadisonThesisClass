%<*UserGuide>

\UWFeature{Thesis and PDF Information}

In order for the \RefSubFeature{Title Page} to function properly, a certain amount of information about the thesis must be given.
The \UWMadClass{} has a set of commands to provide both the thesis information and PDF metadata to \LaTeX{}.

It is highly encouraged to use all of these commands in the preamble such that any PDF metadata can be directly set before the document begins.
If the commands are used within the |document| environment, it will require another \LaTeX{} compilation to include the metadata since \UWMadClass{} will automatically write the information to an external file.

\UWSubFeature{Required}
    These commands are required for the document to be typeset properly.
    It is encouraged to use these commands in the preamble of the document.

    \begin{function} {
        \Title,
        \Author,
        \Program,
        \Degree
    }
        \begin{syntax}
            \cs{Title}   \marg{title}
            \cs{Author}  \marg{author name}
            \cs{Program} \marg{program}
            \cs{Degree}  \marg{degree}
        \end{syntax}
        Each of these commands must be used once; if not, their respective variables be empty while being typeset.
        They can, of course, be used more than once, but the additional usages would only redefine the value of the associated variable.
    \end{function}

    \begin{function} {
        \DefenseDate,
        \DefenceDate
    }
        \begin{syntax}
            \cs{DefenseDate} \marg{defense date}
            \cs{DefenceDate} \marg{defense date}
        \end{syntax}
        Only one of these commands is needed since they all point to the same variable \marg{defense date}.
        The aliases were created for personal preference only.

        Since \marg{defense date} has no parsing performed on it, it may be entered any which way and will be typeset as-entered.
    \end{function}

    \begin{function} {
        \Institution,
        \University
    }
        \begin{syntax}
            \cs{Institution} \marg{institution name}
            \cs{University}  \marg{institution name}
        \end{syntax}
        Only one of these commands is needed since they both point to the same variable \marg{institution name}.
        The aliases were created for personal preference only.
    \end{function}

    \begin{function} {\CommitteeMember}
        \begin{syntax}
            \cs{CommitteeMember} \marg{member name} \marg{member position}
        \end{syntax}
        \cs{CommitteeMember} can be used as many times as required.
        However, if the list of members becomes too large, formatting of the \RefSubFeature[title page]{Title Page} will suffer.
        This may be fixed in the future but would require a much more sophisticated template for the title page (possibly using |expl3| |coffins|).
    \end{function}


\UWSubFeature{Optional}
These commands are not required for the document to be typeset properly.
However, they do provide metadata for the PDF (e.g., keywords and document subject) that is convenient for searching and categorization.
It is encouraged to use these commands in the preamble of the document.

\begin{function} {
    \DocumentType,
    }
    \begin{syntax}
        \cs{DocumentType} \marg{document type}
    \end{syntax}
    By default, the \cs{MakeTitlePage} command prints the phrase \enquote{A \marg{document type} submitted in partial fulfillment of the requirements for the degree of'' on the title page}.
    The default \marg{document type} is \enquote{report}.
    This command sets the value to any valid text.
    However, facilitate good semantic mark-up, some prepared commands to set the document type are given below.
\end{function}

\begin{function} {
    \Dissertation,
    \DoctoralThesis,
    \MastersThesis,
    \Thesis,
    \Prelim
    }
    \begin{syntax}
        \cs{Dissertation}
        \cs{DoctoralThesis}
        \cs{MastersThesis}
        \cs{Thesis}
        \cs{Prelim}
    \end{syntax}
    These commands set the value of \marg{document type} to a value similar to their command name:
    \begin{itemize}
        \item{\cs{Dissertation} sets \marg{document type} to \enquote{dissertation}}
        \item{\cs{DoctoralThesis} sets \marg{document type} to \enquote{doctoral thesis}}
        \item{\cs{MastersThesis} sets \marg{document type} to \enquote{master's thesis}}
        \item{\cs{Thesis} sets \marg{document type} to \enquote{thesis}}
        \item{\cs{Prelim} sets \marg{document type} to \enquote{preliminary report}}
    \end{itemize}
\end{function}

\begin{function} {
    \Advisor,
    \Adviser
    }
    \begin{syntax}
        \cs{Advisor} \marg{advisor name} \marg{advisor position}
        \cs{Adviser} \marg{advisor name} \marg{advisor position}
    \end{syntax}
    Using either of these commands automatically adds the advisor/adviser to the top of the committee list created by \cs{CommitteeMember}.
    Also, on the title page's committee list, the advisor/adviser is marked as such by ``(Advisor)'' or ``(Adviser)''.
    This is a rare exception where the choice of alias has a side-effect.
\end{function}



\begin{function} {
    \Subject,
    \Keywords
    }
    \begin{syntax}
        \cs{Subject}  \marg{document subject}
        \cs{Keywords} \marg{list of keywords}
    \end{syntax}
    These commands set the subject and keyword portions of the PDF metadata.
    The \marg{document subject} is typically a one-ish line description of the document.
    The \marg{list of keywords} can be a long, punctuation-delimited list (e.g., comma or semicolon) of keywords.
\end{function}



\begin{function} {
    \Producer,
    \Creator
    }
    \begin{syntax}
        \cs{Producer} \marg{pdf producer}
        \cs{Creator}  \marg{pdf creator}
    \end{syntax}
    These commands set the PDF Producer and PDF Creator fields of the metadata.
    These fields are a little confusing in their intended usage.
    The best explanation found is
    \begin{description}
        \item[Creator]  {The application used to create the original document which became the PDF.}
        \item[Producer] {The application used to convert the original document into the PDF.}
    \end{description}
    These are very thin distinctions and complicated by the typical workflow of a \LaTeX{} document: installing a \TeX{} distribution, editing a text file in \TeX{}/\LaTeX{} editor, and running the document through a \TeX{} engine with the \LaTeX{} format.
    In order to give credit at all levels (while maintaining proper separation of the processes involved), it is recommended to state the editor and \TeX{} format used as the creator and state the engine and distribution used as the producer.
    For example, this document would declare the following:
    \begin{verbatim}
        \Creator{TeXnicCenter 2.02, LaTeX2e+}
        \Producer{pdfTeX 1.40.14, MiKTeX 2.9}
    \end{verbatim}
    But as stated before, this is all optional.
\end{function}



\UWSubFeature{Accessors}

\begin{function} {
    \TheTitle,
    \TheAuthor,
    \TheProgram,
    \TheDegree,
    \TheDefenseDate,
    \TheDefenceDate,
    \TheInstitution,
    \TheDocumentType,
    \TheAdvisor,
    \TheSubject,
    \TheKeywords,
    \TheProducer,
    \TheCreator
    }
If, for any reason, the thesis information or metadata registered with the document is required, these accessor commands exist to retrieve the stored value.
\end{function}


%</UserGuide>
%
%
%
%
%
%
%<*Implementation>
%<<Verbatim
%   \iffalse
%<*Code>
%   \fi
%^^A ------------------------------------------------------------------------ %
%^^A                    Metadata Writing/Importing Routines                   %
%^^A ------------------------------------------------------------------------ %
%
%   \UWModule{Thesis and PDF Information}
%
%
%   \UWSubModule{Metadata clist and Aux Write}
%   Since the metadata (i.e., properties) of a PDF must be set in the 
%   preamble but typically a user defines them in the document, these
%   routines write the supported metadata that a user may define to an
%   auxiliary file that is then imported upon recompilation.  It uses
%   the |expl3| |clist| commands to define and build the CSV list, and
%   then writes to the file.
%
%
%   Define the |clist|.
%    \begin{macrocode}
\clist_new:N \g__UWMad_MetaDataList_clist
%    \end{macrocode}
%
%   Define a command for pushing entries (with a brace guard) on to the
%   |clist|.
%    \begin{macrocode}
\cs_new:Nn \UWMad_MetaData_PushToList:nn {
   \clist_gput_right:Nn \g__UWMad_MetaDataList_clist {
        #1={#2}
    }
}
%    \end{macrocode}
%
%   Define to booleans: one to tell if a auxilary file is needed
%   and to tell if the |document| has begun.
%    \begin{macrocode}
\bool_new:N \g__UWMad_MetaData_GenerateAux_bool
\bool_new:N \g__UWMad_MetaData_IsDocument_bool
%    \end{macrocode}
%
%   Look for a auxilary file and load it if it exists.
%    \begin{macrocode}
\file_if_exist:nTF{\c_job_name_tl.UWMad.PDFMetaData.aux} {
    \file_input:n {\c_job_name_tl.UWMad.PDFMetaData.aux}
}{}
%    \end{macrocode}
%
%   At the beginning of the document, if data has been pushed to the list,
%   pass it to \cs{hypersetup} so the PDF gets it.
%   Also, set the |IsDocument| boolean true.
%    \begin{macrocode}
\AtBeginDocument{
    \clist_if_empty:NTF \g__UWMad_MetaDataList_clist { } {
        \exp_args:Nx \hypersetup {
            \clist_use:Nn\g__UWMad_MetaDataList_clist{,}
        }
    } { }
    \bool_gset_true:N \g__UWMad_MetaData_IsDocument_bool
}
%    \end{macrocode}
%
%   If thesis information of PDF metadata was used within |document|,
%   write that information to an auxilary file.
%    \begin{macrocode}
\AtEndDocument{
    \bool_if:NTF \g__UWMad_MetaData_GenerateAux_bool {
        \clist_if_empty:NTF \g__UWMad_MetaDataList_clist { } {
            \iow_new:N   \g__UWMad_PDFMetaData_HyperSetup_io
            \iow_open:Nn \g__UWMad_PDFMetaData_HyperSetup_io {
                \c_job_name_tl.UWMad.PDFMetaData.aux
            }
            \iow_now:Nx  \g__UWMad_PDFMetaData_HyperSetup_io {
                \noexpand\ExplSyntaxOff
                    \noexpand\hypersetup
                    {\clist_use:Nn\g__UWMad_MetaDataList_clist{,}}
                \noexpand\ExplSyntaxOn
            }
            \iow_close:N \g__UWMad_PDFMetaData_HyperSetup_io
        } { }
    } { }
}
%    \end{macrocode}
%
%   \UWSubModule{Thesis Information}
%
%   Declare the |ThesisInfo| token list variables.
%    \begin{macrocode}
\tl_new:N \g__UWMad_ThesisInfo_Title_tl
\tl_new:N \g__UWMad_ThesisInfo_Author_tl
\tl_new:N \g__UWMad_ThesisInfo_DefenseDate_tl
\tl_new:N \g__UWMad_ThesisInfo_Department_tl
\tl_new:N \g__UWMad_ThesisInfo_Program_tl
\tl_new:N \g__UWMad_ThesisInfo_Degree_tl
\tl_new:N \g__UWMad_ThesisInfo_DocumentType_tl
\tl_new:N \g__UWMad_ThesisInfo_AdvisorName_tl
\tl_new:N \g__UWMad_ThesisInfo_AdvisorPosition_tl
\tl_new:N \g__UWMad_ThesisInfo_AdvisorAssociation_tl
\tl_new:N \g__UWMad_ThesisInfo_AdvisorMarker_tl
\tl_new:N \g__UWMad_ThesisInfo_Institution_tl
%    \end{macrocode}
%
%   Set the document type default.
%    \begin{macrocode}
\tl_gset:Nn \g__UWMad_ThesisInfo_DocumentType_tl {report}
%    \end{macrocode}
%
%   Define some booleans for required information/
%    \begin{macrocode}
\bool_new:N \g__UWMad_ThesisInfo_IsSet_Title_bool
\bool_new:N \g__UWMad_ThesisInfo_IsSet_Author_bool
\bool_new:N \g__UWMad_ThesisInfo_IsSet_DefenseDate_bool
\bool_new:N \g__UWMad_ThesisInfo_IsSet_Program_bool
\bool_new:N \g__UWMad_ThesisInfo_IsSet_Degree_bool
\bool_new:N \g__UWMad_ThesisInfo_IsSet_Institution_bool
\bool_new:N \g__UWMad_ThesisInfo_IsSet_Advisor_bool
%    \end{macrocode}
%
%   Declare the user front-end for the title.
%    \begin{macrocode}
\DeclareDocumentCommand \Title { m } {
%    \end{macrocode}
%   Set the associated token list variable
%    \begin{macrocode}
    \tl_gset:Nn \g__UWMad_ThesisInfo_Title_tl {#1}
%    \end{macrocode}
%   Pass it to the default \LaTeX{} \cs{title} command.
%    \begin{macrocode}
    \title{#1}
%    \end{macrocode}
%   Push the value to the MetaData |clist|.
%    \begin{macrocode}
    \UWMad_MetaData_PushToList:nn{pdftitle}    {#1}
%    \end{macrocode}
%   If this command was used within the |document|, tell the class
%   to write an auxilary file.
%    \begin{macrocode}
    \bool_if:NTF \g__UWMad_MetaData_IsDocument_bool {
        \bool_gset_true:N \g__UWMad_MetaData_GenerateAux_bool
    } { }
%    \end{macrocode}
%   Tell the class this variable is now set.
%    \begin{macrocode}
    \bool_gset_true:N \g__UWMad_ThesisInfo_IsSet_Title_bool
}
%    \end{macrocode}
%
%   Similar flow to the \cs{Title} defintion.
%    \begin{macrocode}
\DeclareDocumentCommand \Author { m } {
    \tl_gset:Nn \g__UWMad_ThesisInfo_Author_tl {#1}
    \author{#1}
    \UWMad_MetaData_PushToList:nn{pdfauthor}   {#1}
    \bool_if:NTF \g__UWMad_MetaData_IsDocument_bool {
        \bool_gset_true:N \g__UWMad_MetaData_GenerateAux_bool
    } { }
    \bool_gset_true:N \g__UWMad_ThesisInfo_IsSet_Author_bool
}
%    \end{macrocode}
%
%   A simple setter command.
%    \begin{macrocode}
\DeclareDocumentCommand \Program { m } {
    \tl_gset:Nn \g__UWMad_ThesisInfo_Program_tl {#1}
    \bool_gset_true:N \g__UWMad_ThesisInfo_IsSet_Program_bool
}
%    \end{macrocode}
%
%   A simple setter command.
%    \begin{macrocode}
\DeclareDocumentCommand \Degree { m } {
    \tl_gset:Nn \g__UWMad_ThesisInfo_Degree_tl {#1}
    \bool_gset_true:N \g__UWMad_ThesisInfo_IsSet_Degree_bool
}
%    \end{macrocode}
%
%   A simple setter command.
%    \begin{macrocode}
\DeclareDocumentCommand \DocumentType { m } {
    \tl_gset:Nn \g__UWMad_ThesisInfo_DocumentType_tl {#1}
}
%    \end{macrocode}
%
%   A semantic setter command.
%    \begin{macrocode}
\DeclareDocumentCommand \Dissertation { } {
    \tl_gset:Nn \g__UWMad_ThesisInfo_DocumentType_tl {
        dissertation
    }
}
%    \end{macrocode}
%
%   A semantic setter command.
%    \begin{macrocode}
\DeclareDocumentCommand \DoctoralThesis { } {
    \tl_gset:Nn \g__UWMad_ThesisInfo_DocumentType_tl {
        doctoral~thesis
    }
}
%    \end{macrocode}
%
%   A semantic setter command.
%    \begin{macrocode}
\DeclareDocumentCommand \MastersThesis { } {
    \tl_gset:Nn \g__UWMad_ThesisInfo_DocumentType_tl {
        master's~thesis
    }
}
%    \end{macrocode}
%
%   A semantic setter command.
%    \begin{macrocode}
\DeclareDocumentCommand \Thesis { } {
    \tl_gset:Nn \g__UWMad_ThesisInfo_DocumentType_tl {
        thesis
    }
}
%    \end{macrocode}
%
%   A semantic setter command.
%    \begin{macrocode}
\DeclareDocumentCommand \Prelim { } {
    \tl_gset:Nn \g__UWMad_ThesisInfo_DocumentType_tl {
        preliminary~report
    }
}
%    \end{macrocode}
%
%   A simple setter command and aliases.
%    \begin{macrocode}
\DeclareDocumentCommand \DefenseDate { m } {
    \tl_gset:Nn \g__UWMad_ThesisInfo_DefenseDate_tl {#1}
    \bool_gset_true:N \g__UWMad_ThesisInfo_IsSet_DefenseDate_bool
}
\cs_gset_eq:NN \DefenceDate \DefenseDate
%    \end{macrocode}
%
%   A simple setter command and aliases.
%    \begin{macrocode}
\DeclareDocumentCommand \Institution { m } {
    \tl_gset:Nn       \g__UWMad_ThesisInfo_Institution_tl {#1}
    \bool_gset_true:N \g__UWMad_ThesisInfo_IsSet_Institution_bool
}
\cs_set_eq:NN \University \Institution
%
%
%
%
%
%   User front-ends (Optional)
\DeclareDocumentCommand \Department { m } {
    \tl_gset:Nn \g__UWMad_ThesisInfo_Department_tl {#1}
}
\DeclareDocumentCommand \Advisor { m m m } {
    \bool_gset_true:N \g__UWMad_ThesisInfo_IsSet_Advisor_bool
    \tl_gset:Nn \g__UWMad_ThesisInfo_AdvisorName_tl        {#1}
    \tl_gset:Nn \g__UWMad_ThesisInfo_AdvisorPosition_tl    {#2}
    \tl_gset:Nn \g__UWMad_ThesisInfo_AdvisorAssociation_tl {#3}
    \tl_gset:Nn \g__UWMad_ThesisInfo_AdvisorMarker_tl      {Advisor}
 }
\DeclareDocumentCommand \Adviser { m m m } {
    \bool_gset_true:N \g__UWMad_ThesisInfo_IsSet_Advisor_bool
    \tl_gset:Nn \g__UWMad_ThesisInfo_AdvisorName_tl        {#1}
    \tl_gset:Nn \g__UWMad_ThesisInfo_AdvisorPosition_tl    {#2}
    \tl_gset:Nn \g__UWMad_ThesisInfo_AdvisorAssociation_tl {#3}
    \tl_gset:Nn \g__UWMad_ThesisInfo_AdvisorMarker_tl      {Adviser}
}
%
%
%
\msg_new:nnn { UWMadThesis } { ThesisInfo / UnsetInformation } {
    The~required~information~for~the~#1~is~not~set.
}
\DeclareDocumentCommand \IfInfoIsSetT { m +m } {
    \bool_if:cTF {g__UWMad_ThesisInfo_IsSet_ #1 _bool} {
        #2
    } {
        \msg_error:nnn
            { UWMadThesis }
            { ThesisInfo / UnsetInformation }
            {#1}
    }
}
%
%
%
%
%   User front-end accessors
\DeclareDocumentCommand \TheTitle { } {
    \g__UWMad_ThesisInfo_Title_tl
}
\DeclareDocumentCommand \TheAuthor { } {
    \g__UWMad_ThesisInfo_Author_tl
}
\DeclareDocumentCommand \TheProgram { } {
    \g__UWMad_ThesisInfo_Program_tl
}
\DeclareDocumentCommand \TheDegree { } {
    \g__UWMad_ThesisInfo_Degree_tl
}
\DeclareDocumentCommand \TheDocumentType { } {
    \g__UWMad_ThesisInfo_DocumentType_tl
}
\DeclareDocumentCommand \TheDefenseDate { } {
    \g__UWMad_ThesisInfo_DefenseDate_tl
}
\cs_gset_eq:NN \TheDefenceDate \TheDefenseDate
\DeclareDocumentCommand \TheInstitution { } {
    \g__UWMad_ThesisInfo_Institution_tl
}
\cs_set_eq:NN \TheUniversity \TheInstitution
%
\DeclareDocumentCommand \TheDepartment { } {
    \g__UWMad_ThesisInfo_Department_tl
}
\DeclareDocumentCommand \TheAdvisor { } {
    \g__UWMad_ThesisInfo_AdvisorName_tl
}
%
%
%
%
%
\tl_new:N   \g__UWMad_ThesisInfo_Committee_InfoSeparator_tl
\tl_gset:Nn \g__UWMad_ThesisInfo_Committee_InfoSeparator_tl {,}
%
\int_new:N \l__UWMad_ThesisInfo_CommitteeCount_int
\UWMad_ListOf_Define:n {CommitteeList}
\DeclareDocumentCommand \CommitteeMember { m m m } {
    \int_incr:N \l__UWMad_ThesisInfo_CommitteeCount_int
    \UWMad_ListOf_PushEntry:nn {CommitteeList} {
        #1
        \g__UWMad_ThesisInfo_Committee_InfoSeparator_tl
        ~
        \textsl{#2}
        \g__UWMad_ThesisInfo_Committee_InfoSeparator_tl
        ~
        \textsl{#3}
        \skip_vertical:n{-1em}
    }
}
\DeclareDocumentCommand \PrintCommitteeMemberList { } {
    \bool_if:NTF \g__UWMad_ThesisInfo_IsSet_Advisor_bool {
        \g__UWMad_ThesisInfo_AdvisorName_tl{}
        \g__UWMad_ThesisInfo_Committee_InfoSeparator_tl
        ~
        \textsl{\g__UWMad_ThesisInfo_AdvisorPosition_tl{}}
        \g__UWMad_ThesisInfo_Committee_InfoSeparator_tl
        ~
        \textsl{\g__UWMad_ThesisInfo_AdvisorAssociation_tl{}}
        ~
        (\g__UWMad_ThesisInfo_AdvisorMarker_tl{})
        \skip_vertical:n{-1em}
    } { }
    \UWMad_ListOf_PrintEntries:n {CommitteeList}
}
%
%
%
%
%
\tl_new:N \g__UWMad_PDFMetaData_Subject_tl
\tl_new:N \g__UWMad_PDFMetaData_Keywords_tl
\tl_new:N \g__UWMad_PDFMetaData_Producer_tl
\tl_new:N \g__UWMad_PDFMetaData_Creator_tl
%
%
%
%   User front-end (Optional)
\DeclareDocumentCommand \Subject { m } {
    \tl_gset:Nn \g__UWMad_PDFMetaData_Subject_tl {#1}
    \UWMad_MetaData_PushToList:nn{pdfsubject}  {#1}
    \bool_if:NTF \g__UWMad_MetaData_IsDocument_bool {
        \bool_gset_true:N \g__UWMad_MetaData_GenerateAux_bool
    } { }
}
\DeclareDocumentCommand \Keywords { m } {
    \tl_gset:Nn \g__UWMad_PDFMetaData_Keywords_tl {#1}
    \UWMad_MetaData_PushToList:nn{pdfkeywords} {#1}
    \bool_if:NTF \g__UWMad_MetaData_IsDocument_bool {
        \bool_gset_true:N \g__UWMad_MetaData_GenerateAux_bool
    } { }
}
\DeclareDocumentCommand \Producer { m } {
    \tl_gset:Nn \g__UWMad_PDFMetaData_Producer_tl {#1}
    \UWMad_MetaData_PushToList:nn{pdfproducer}  {#1}
    \bool_if:NTF \g__UWMad_MetaData_IsDocument_bool {
        \bool_gset_true:N \g__UWMad_MetaData_GenerateAux_bool
    } { }
}
\DeclareDocumentCommand \Creator { m } {
    \tl_gset:Nn \g__UWMad_PDFMetaData_Creator_tl {#1}
    \UWMad_MetaData_PushToList:nn{pdfcreator} {#1}
    \bool_if:NTF \g__UWMad_MetaData_IsDocument_bool {
        \bool_gset_true:N \g__UWMad_MetaData_GenerateAux_bool
    } { }
}
%
%
%
%   User front-end accessors.
\DeclareDocumentCommand \TheSubject { } {
    \g__UWMad_PDFMetaData_Subject_tl
}
\DeclareDocumentCommand \TheKeywords { } {
    \g__UWMad_PDFMetaData_Keywords_tl
}
\DeclareDocumentCommand \TheProducer { } {
    \g__UWMad_PDFMetaData_Producer_tl
}
\DeclareDocumentCommand \TheCreator { } {
    \g__UWMad_PDFMetaData_Creator_tl
}
%
%    \end{macrocode}
%
%
%
%
%   \iffalse
%</Code>
%   \fi
%Verbatim
%</Implementation>