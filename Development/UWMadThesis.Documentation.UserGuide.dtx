\UWFeature{Programming}
The Programming Module has no immediate user-facing features.
The Implementation section for this module outlines the programming layer
and is intended for average use.

\UWFeature{Math}\label{UG:Math}
As the feature name may suggest, all of the commands in this section deal with mathematical typesetting.

\UWSubFeature{Derivative Commands}
These command set deal with quick and easy typesetting of derivatives.

\begin{function}{\deriv , \pderiv , \tderiv}
    \begin{syntax}
        \cs{deriv}  \Arg{function} \Arg{variable} \Arg{order}    \\
        \cs{pderiv} \Arg{function} \Arg{variable} \Arg{order}    \\
        \cs{tderiv} \Arg{function} \Arg{variable} \Arg{order}
    \end{syntax}
    This function set is meant to typeset three different kinds of derivatives: ordinary, partial, and total (i.e., material or Lagragian).
    The only difference between them is the differential symbol: \cs{deriv} uses `$\mathrm{d}$', \cs{pderiv} uses `$\partial$', and \cs{tderiv} used `$\mathrm{D}$'.

    These commands typeset the derivative of a given \Arg{function} with respect to \Arg{variable} of $n$-th \Arg{order} using Leibniz's notation.
    The \Arg{order} is optional and defaults to empty (first derivative).
    For example, the input
    \begin{verbatim}
        \begin{align}
            \deriv{y}{x}{2} + \deriv{y}{x} + y(x)         &= 0    \\[0.50em]
            \pderiv{T}{t} - \alpha \pderiv{T}{z}{2}       &= 0    \\[0.50em]
            \tderiv{\rho{u}}{t} + \pderiv{P}{z}  - \rho g &= 0
        \end{align}
    \end{verbatim}
    and is typeset as
        \begin{align}
            \deriv{y}{x}{2} + \deriv{y}{x} + y(x)         &= 0    \\[0.50em]
            \pderiv{T}{t} - \alpha \pderiv{T}{z}{2}       &= 0    \\[0.50em]
            \tderiv{(\rho{u})}{t} + \pderiv{P}{z}  - \rho g &= 0
        \end{align}
\end{function}

\begin{function}{\derivbig,\pderivbig,\tderivbig}
    \begin{syntax}
        \cs{derivbig}   \oarg{left delim} \Arg{function} \oarg{right delim} \Arg{variable} \Arg{order} \\
        \cs{pderivbig}  \oarg{left delim} \Arg{function} \oarg{right delim} \Arg{variable} \Arg{order} \\
        \cs{tderivbig}  \oarg{left delim} \Arg{function} \oarg{right delim} \Arg{variable} \Arg{order}
    \end{syntax}
    This function set is identical to the non-|big| versions above, except that \Arg{function} is placed to the right of the derivative operator and wrapped by |\left| and |\right|.
    The default delimiters for the stretch commands are `[' and ']', and either can be individually overridden via the two optional arguments.
    For example, the input
    \begin{verbatim}
        \begin{align}
            -\derivbig{ p(x) \deriv{y}{x} }{x} +
                    q(x) (1 - \lambda) y(x)  &= 0 \\[0.50em]
            \tderivbig{ \rho{i} + \frac{1}{2} \rho u^2 }[(]{t} -
                    \pderivbig[\lvert]{ \kappa \pderiv{T}{z} }{z} &= 0
        \end{align}
    \end{verbatim}
    and is typeset as
        \begin{align}
            -\derivbig{ p(x) \deriv{y}{x} }{x} +
                    q(x) (1 - \lambda) y(x)  &= 0 \\[0.50em]
            \tderivbig{ \rho{i} + \frac{1}{2} \rho u^2 }[(]{t} -
                    \pderivbig[\lvert]{ \kappa \pderiv{T}{z} }{z} &= 0
        \end{align}
\end{function}

\begin{function}{\DerivativeGeneral,\DerivativeGeneralBig}
    \begin{syntax}
        \cs{DerivativeGeneral}     \Arg{function} \Arg{variable} \Arg{order} \Arg{symbol}
        \cs{DerivativeGeneralBig}  \Arg{function} \Arg{variable} \Arg{order} \Arg{symbol} \Arg{left delim} \Arg{right delim}
    \end{syntax}
    These commands are lower-level commands used by the |deriv| family above.
    All of the arguments are mandatory.
    If a change to the general style of the derivatives or another version of the |deriv| family is desire, these commands are available for usage.
\end{function}

\begin{function}{\derivSymbol,\pderivSymbol,\tderivSymbol}
    \begin{syntax}
        \cs{derivSymbol}
    \end{syntax}
    These commands take no arguments and expand to the current symbol used for the associated |deriv| command.
    The defaults require math mode to be typeset.
    Therefore, |$\pderivSymbol$| will be appear as $\pderivSymbol$.
\end{function}

\begin{function}{\derivSymbolChange,\pderivSymbolChange,\tderivSymbolChange}
    \begin{syntax}
        \cs{derivSymbolChange} \Arg{symbol} \\[0.50em]
    \end{syntax}
    These commands will \textsc{temporarily} change the symbol used by the associated |deriv| commands.
    The symbol will revert back to the original, default value after leaving the \TeX{} group where the switch was made (more often than not for \LaTeX{} users, this means ``upon exiting an environment'').
    For example:
    \begin{verbatim}
        \begin{equation}
            \deriv{U}{t} =
            \derivSymbolChange{\delta}
            \deriv{Q}{t} - \deriv{W}{t}
        \end{equation}
    \end{verbatim}
    typesets as
    \begin{equation}
            \deriv{U}{t} =
            \derivSymbolChange{\delta}
            \deriv{Q}{t} - \deriv{W}{t}
    \end{equation}
    and now, after the environment, the \cs{derivSymbol} is once again `$\derivSymbol$'.
\end{function}

\begin{function}{\derivSymbolChangeDefault,\pderivSymbolChangeDefault,\tderivSymbolChangeDefault}
    \begin{syntax}
        \cs{derivSymbolChangeDefault} \Arg{symbol} \\[0.50em]
    \end{syntax}
    These commands will \textsc{permanently} change the symbol used by the associated |deriv| commands.
    For example:
    \begin{verbatim}
        \begin{equation}
            \deriv{U}{t} =
            \derivSymbolChangeDefault{\delta}
            \deriv{Q}{t} - \deriv{W}{t}
        \end{equation}
    \end{verbatim}
    typesets as
    \begin{equation}
            \deriv{U}{t} =
            \derivSymbolChangeDefault{\delta}
            \deriv{Q}{t} - \deriv{W}{t}
    \end{equation}
    and now, after the environment, the \cs{derivSymbol} is `$\derivSymbol$'.
\end{function}

\begin{function}{\DelimiterChangeDefault}
    \begin{syntax}
        \cs{DelimiterChangeDefault} \Arg{left delim} \Arg{right delim}
    \end{syntax}
    This command changes the default delimiters used by the |big| commands above.
    Any valid delimiters can be used.
    For example:
    \begin{verbatim}
        \DelimiterChangeDefault{(}{)}
        \begin{equation}
            -\derivbig{ p(x) \deriv{y}{x} }{x} +
                    q(x) (1 - \lambda) y(x) = 0 \\[0.50em]
        \end{equation}
    \end{verbatim}
    and is typeset as
    \DelimiterChangeDefault{(}{)}
    \begin{equation}
        -\derivbig{ p(x) \deriv{y}{x} }{x} +
                q(x) (1 - \lambda) y(x) = 0 \\[0.50em]
    \end{equation}
    and notice that the \cs{derivSymbol} is still $\derivSymbol$.
\end{function}

\UWSubFeature{Operators}
These operators are added to the standard set using the \AmS{} operator system.
Some are new while others are simply in a camel-cased versions of the standard ones.

\begin{function}{\Sup,\Inf}
    Supremum and Infinum operators using the math operator system.
    For example, the input
    \begin{verbatim}
        \begin{align}
            \Inf_{x \in \mathbb{R}} \{0 < x  < 1\} &= 0 \\[0.50em]
            \Sup_{x \in \mathbb{R}} \{0 < x  < 1\} &= 1
        \end{align}
    \end{verbatim}
    is typeset as
    \begin{align}
        \Inf_{x \in \mathbb{R}}
            \{0 \LessThan x \LessThan 1\} &= 0 \\[0.50em]
        \Sup_{x \in \mathbb{R}}
            \{0 \LessThan x \LessThan 1\} &= 1
    \end{align}
\end{function}

\begin{function}{\Lim}
    The limit operator:
    \begin{verbatim}
        \begin{equation}
            \Lim_{n \rightarrow \infty} \left(1 + \frac{1}{n}\right)^n = \mathrm{e}
        \end{equation}
    \end{verbatim}
    is typeset as
        \begin{equation}
            \Lim_{n \rightarrow \infty} \left(1 + \frac{1}{n}\right)^n = \mathrm{e}
        \end{equation}
\end{function}

\begin{function}{\Min,\Max}
    The maximum and minimum value operators
    \begin{verbatim}
        \begin{equation}
            \begin{align}
                \Min_{x \in \mathbb{R}} \Sin(x) &= -1 \\[0.50em]
                \Max_{x \in \mathbb{R}} \Sin(x) &= +1
            \end{align}
        \end{equation}
    \end{verbatim}
    is typeset as
    \begin{align}
        \Min_{x \in \mathbb{R}} \Sin(x) &= -1 \\[0.50em]
        \Max_{x \in \mathbb{R}} \Sin(x) &= +1
    \end{align}
\end{function}

\begin{function}{\ArgMin,\ArgMax}
    The maximum and minimum argument operators
    \begin{verbatim}
        \begin{equation}
            \begin{align}
                \ArgMin_{x \in \mathbb{R}} \Sin(x) &= \frac{3\pi}{2} + 2 \pi n \\[0.50em]
                \ArgMax_{x \in \mathbb{R}} \Sin(x) &= \frac{\pi}{2} + 2 \pi n
            \end{align}
        \end{equation}
    \end{verbatim}
    is typeset as
    \begin{align}
                \ArgMin_{x \in \mathbb{R}} \Sin(x) &= \frac{3\pi}{2} + 2 \pi n \\[0.50em]
                \ArgMax_{x \in \mathbb{R}} \Sin(x) &= \frac{\pi}{2} + 2 \pi n
    \end{align}
\end{function}

\begin{function}{\Abs,\Ln,\Log,\Exp}
    Common set of operators in uppercase form.
\end{function}

\begin{function}{\Cos,\Sin,\Tan,\Sec,\Csc,\Cot}
    Standard trigonometric functions and their reciprocals.
\end{function}
\begin{function}{\Cosh,\Sinh,\Tanh,\Sech,\Csch,\Coth}
    Hyperbolic trigonometric functions and their reciprocals.
\end{function}
\begin{function}{\ArcCos,\ArcSin,\ArcTan,\ArcSec,\ArcCsc,\ArcCot}
    Standard inverse trigonometric functions and their reciprocals.
\end{function}
\begin{function}{\ArcCosh,\ArcSinh,\ArcTanh,\ArcSech,\ArcCsch,\ArcCoth}
    Hyperbolic inverse trigonometric functions and their reciprocals.
\end{function}

\UWSubFeature{Miscellaneous Commands}

\begin{function}{\Sqrt}
    \begin{syntax}
        \cs{Sqrt} \oarg{n} \Arg{argument}
    \end{syntax}
    This command typesets the \oarg{n}-th root of a given \Arg{argument} with a closing tail.
    This command differs from the default \cs{sqrt} in appearance only:
    \begin{equation}
        \sqrt[3]{\frac{f(x)}{g(x)}} = \Sqrt[3]{\frac{f(x)}{g(x)}}
    \end{equation}
\end{function}

\begin{function}{\IfMathModeTF}
    \begin{syntax}
        \cs{IfMathModeTF} \Arg{math mode code} \Arg{text mode code}
    \end{syntax}
    This is an abstraction of |expl3|'s |\mode_if_math:TF| function.
    It was added to give more control on the following \cs{subs} and \cs{sups} commands since |expl3|'s syntax is disabled to make |_| a subscript shift and not a letter.
\end{function}

\begin{function}{\subs,\sups,\subsups}
    \begin{syntax}
        \cs{subs}    \oarg{space} \Arg{text subscript} \\[0.50em]
        \cs{sups}    \oarg{space} \Arg{text superscript} \\[0.50em]
        \cs{subsups} \oarg{subscript space} \Arg{text subscript} \oarg{superscript space} \Arg{text superscript}
    \end{syntax}
    These command typeset a subscript or superscript \textsc{in text mode}.
    They are useful if the subscript or superscript are not variable, and therefore should be in non-math text, or for making subscripts or superscripts in text mode.
    The optional argument \oarg{space} is meant for adjusting the spacing of the scripts and exists in \textsc{in math mode}, so technically, any valid math statement can be used.
    However, it is encouraged to only use this argument for spacing.
    For example, the input |`T\subs{P}, $T\subs{P}$, $T_P$'| is typeset as `T\subs{P}, $T\subs{P}$, $T_P$', and the input |`T\subs[\!]{P}, T\subs[\:]{P}'| is typeset as `T\subs[\!]{P}, T\subs[\:]{P}'.
    T\sups{P}
\end{function}

\begin{function}{\OneOver,\oneo}
    \begin{syntax}
        \cs{OneOver}  \Arg{denominator}
    \end{syntax}
    A simple command the typesets a fraction whose numerator is always one.
    For example, the input
    \begin{verbatim}
        \begin{equation}
            \OneOver{\Sqrt{x^2 + 1}}
        \end{equation}
    \end{verbatim}
    is typeset as
        \begin{equation}
            \OneOver{\Sqrt{x^2 + 1}}
        \end{equation}
\end{function}

\begin{function}{\dd}
    \begin{syntax}
        \cs{dd}  \Arg{variable}
    \end{syntax}
    A simple command the typesets a non-math `d' in math mode and is meant to be used for differentials.
    For example, the input
    \begin{verbatim}
        \derivSymbolChangeDefault{\mathrm{d}}
        \begin{equation}
            f(b) - f(a) = \int_a^b \deriv{f}{t} \dd{t}
        \end{equation}
    \end{verbatim}
    is typeset as
        \derivSymbolChangeDefault{\mathrm{d}}
        \begin{equation}
            f(b) - f(a) = \int_a^b \deriv{f}{t} \dd{t}
        \end{equation}
\end{function}

\begin{function}{\dprime,\tprime}
    \begin{syntax}
        \cs{dprime}
    \end{syntax}
    These commands take no arguments and simply mean `double prime' and `triple prime'.
    For example, the input
    \begin{verbatim}
        \begin{equation}
            q^\prime = q^\dprime 2\pi{R} = q^\tprime \pi{R^2}
        \end{equation}
    \end{verbatim}
    is typeset as
        \begin{equation}
            q^\prime = q^\dprime 2\pi{R} = q^\tprime \pi{R^2}
        \end{equation}
\end{function}


\UWFeature{List Environments}
\UWSubFeature{Nomenclature}
\UWSubFeature{Acronym}


\UWFeature{Thesis and PDF Information}

In order for the \RefSubFeature{Title Page} to function properly, a certain amount of information about the thesis must be given.
The \UWMadClass{} has a set of commands to provide both the thesis information and PDF metadata to \LaTeX{}.

It is highly encouraged to use all of these commands in the preamble such that any PDF metadata can be directly set before typesetting begins.
If the commands are used within the |document| environment, it will require another \LaTeX{} compilation to include the metadata since \UWMadClass{} will automatically right the information to a an external file.

\UWSubFeature{Required}
These commands are required for the document to be typeset properly.
It is encouraged to use these commands in the preamble of the document.

\begin{function} {
    \Title,
    \Author,
    \Degree,
    \Program,
    \DefenseDate,
    \Date,
    \Institution,
    \University}
    \begin{syntax}
        \vspace*{3pt}
        \setstretch{1.30}
        \cs{Title}          \marg{title}
        \cs{Author}         \marg{author name}
        \cs{Degree}         \marg{degree}
        \cs{Program}        \marg{program}
        \cs{DefenseDate}    \marg{defense date}
        \cs{Date}           \marg{defense date}
        \cs{Institution}    \marg{institution name}
        \cs{University}     \marg{institution name}
    \end{syntax}
    The commands \cs{DefenseDate} and \cs{Date} are aliases to the same variable.
    Also, \cs{Institution} and \cs{University}  are aliases to the same variable.

    Since \marg{defense date} has no parsing performed on it, it may be entered any which way and will be typeset as-entered.
\end{function}

\UWSubFeature{Optional}


\UWFeature{Special Pages}

\UWSubFeature{Title Page}
This is a replace for the default \cs{maketitlepage}.
Per the example provided by the \UWMadLong{} Graduate School's sample, the sample page flows (in order): thesis title, author by-line, partial fulfillment clause, degree, program, university identification, oral defense date, and oral committee list.
The styles can be re-worked by redefining the commands as presented in the \RefSubModule{MakeTitlePage} implementation.
The formatting of the commands is standard \LaTeXe{} to facilitate customization.

\textsc{Note:} The \cs{MakeTitlePage} command needs the required thesis information from the commands described in the \RefSubFeature[Required subsection]{Required}.

\UWSubFeature{License Page}

\begin{LicensePage}
    \CreativeCommons
    \NonCommercial
    \ShareAlike
\end{LicensePage}

